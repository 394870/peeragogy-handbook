\subsubsection{Main actor: Simone}

Simone is a young media department graduate, who followed the adventures
of the journalist Jorge Luis. Jorge Luis was transforming the newspaper
operation into a kind of collective learning project, turning the
newsroom into a platform for discussion and learning, and inciting the
developers to provide an API for external coders. Simone wrote a paper
about all this in her last year at the media department.\textbf{~}She
also runs a blog about tools which empower people to participate in
politics (local, nation-wide and international).

\subsubsection{Main success scenario}

\begin{enumerate}
\item
  Simone loves her blog. She believes verticals and specialization are
  the future in blogging. However, she needs money to live, and to pay
  back the debts she made to finance her studies. Her media department
  was moderately interesting, but nobody ever thought of organizing a
  course ``entrepreneurial blogging/journalism''.
\item
  Posting every day about collaborative online tools such as wikis,
  forums, blogs, mindmaps, synchronous sessions, social bookmarks,
  visualization tools, Simone decides to reach out and look online for
  others who are experiencing the same challenges.
\item
  As she encounters various other people, they start curating stuff
  about blogging business models and best practices. They find lots of
  useful stuff for free at Robin Good's website, and they manage to get
  access to online resources at a strange group which seems to
  specialize in ``mind amplifying tools'' and ``literacies of
  cooperation''. They also discover that ``entrepreneurial journalism''
  is taught at various colleges, and invariably the professors and most
  of the students there indulge in blogging and publishing about their
  insights and experiments. All that material is being discussed on the
  collaborative platform Simone built.
\item
  Simone uses the discussions to blog about her experience. After all,
  issues about financing media who empower people in order to broaden
  and deepen the democracy is something which is rather on topic for her
  own blogging practice. Also, because of her reaching out, her contacts
  increased considerably. She works together with someone to share a
  virtual co-working space, and people start noticing her. Some ask her
  for customized expert advice about collaborative tools and
  collaboration methodologies. The city council expresses some vague
  interest and considers hiring her as a consultant.
\item
  Even though she gets several gigs, Simone realizes it's not easy to
  earn a living as a blogger. But it seems to open other doors\ldots{}
  however, she continues her investigation about business models for
  collaborative media. As yet we don't know whether Simone's blog will
  be profitable in itself, but we do see a network around her projects,
  exchanging insights but also valuable business information and opening
  more doors.
\end{enumerate}

\subsubsection{Thoughts}

I had the opportunity to give some seminars at media departments here in
Belgium. In my experience, the students were not familiar with curation
practices or infotention strategies. They also lack courses in
entrepreneurial journalism. In other words, they're still educated for
the big media companies, but they're not prepared to start the next
TechCrunch or Huffington Post. Often the students asked me, after the
seminar, ``how can we learn all this? they won't teach us these things
here''. I think there is a need for P2P learning about not only
curation, infotention, social dashboards, communities and governance of
common pool recourses, but also about publishing strategies, social
media workflows and business models.
