It is very useful to have an up-to-date public roadmap for the project,
someplace where it can be discussed and maintained.~ This
helps~\textbf{newcomers} know where they can jump in.~ It also gives a
sense of the accomplishments to date, and any major challenges that lie
ahead.~ Remember, the Roadmap exists as an artifact with which to share
current, but never complete, understanding of the space.~ Never stop
learning!

\subsubsection{Examples}

\textbf{}In the Peeragogy project, now that the outline is fairly
mature, we can use it as a roadmap, by marking the sections that are
``finished'' (at least in draft), marking the sections where editing is
currently taking place, and marking the stubs (possible starting points
for future contributors).~ While this does not provide a complete
roadmap for all aspects of the project, it does give editors a sense of
what is going on.~ The Free Technology Guild provides one
\href{http://campus.ftacademy.org/wiki/index.php/Free\_Technology\_Guild\_Roadmap}{example}.

\subsubsection{And also}

Note that a shared roadmap is very similar to a
\href{http://peeragogy.org/to-peeragogy/personal-learning-plan/}{Personal
Learning Plan}, or ``paragogical profile''.~ We've made some
\href{http://campus.ftacademy.org/wiki/index.php/Free\_Technology\_Guild\#Learning\_design}{examples}
of these as we got started working on the Free Technology Guild. ~
