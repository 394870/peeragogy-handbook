\subsection{Summary}

All collaborative work is managed in some way. Methods of managing
projects, including learning projects, are ranking from the more formal
and structured to the less formal and unstructured.

\subsection{Participation in business-oriented projects}

When we think about project management in an organization, we often
relate to well-established tools and processes. For example, we will use
the \href{http://www.pmi.org/PMBOK-Guide-and-Standards.aspx}{Project
Management Body of Knowledge}~(PMBOK)~as a standard. For the Project
Management Institute (PMI) and most workers, those standards are the key
to project success. In classical project management, tasks and deadlines
are clearly defined. We will, for example, use
\href{http://en.wikipedia.org/wiki/PERT}{Program Evaluation and Review
Technic} (PERT)~to analyze and represent tasks. We often represent the
project schedule using
a~\href{http://en.wikipedia.org/wiki/Gantt\_chart}{Gantt chart}. Those
are just two of the project management tools that illustrate how project
management rests firmly on its engineering background. In those very
structured projects, each actor is expected to work exactly as planned
and to deliver his part of the work on time; every individual delay
potentially leading to a collective delay.

\subsection{Participation in educational projects}

If we look for analogies between project management and education, we
can find some similarities in models pedagogy of. In a paper called
"\href{http://www-distance.syr.edu/andraggy.html}{Moving from Pedagogy
to Andragogy}"~by Hiemstra and Sisco, we see how students hold a passive
role (on a cognitive level) in the pedagogy model. They are following a
plan or syllabus that has been designed by the instructor and that won't
change during the session. Students will have to complete all their
tasks on time; in other words, return their exercices to the teacher
before the due date. In a peeragogy project, whose roots lie closer to
andragogy than in pedagogy, participation to the project is less
regulated (see \href{http://peeragogy.org/to-peeragogy/}{From peer
learning to peeragogy})

As peeragogy projects members expect to break
the~\href{http://en.wikipedia.org/wiki/1\%25\_rule\_\%28Internet\_culture\%29}{90/9/1
rule}~and bring on board more than 1\% of creators and 9\% of editors,
they also keep in mind
the~\href{http://en.wikipedia.org/wiki/Long\_Tail}{Long Tail}~rule.
``The term Long Tail has gained popularity in recent times as describing
the retailing strategy of selling a large number of unique items with
relatively small quantities sold of each.'' In other words, people
working in peeragogy should accept that some participants only
contribute few ideas (or may be even just one!). Going further, people
may even be allowed to just watch a peeragogy project going on without
creating or editing, in order to understand its culture before feeling
ready to jump in and contribute more actively.

In general, a peeragogy community will constantly adjust as it seeks an
equilibrium between order and chaos, allowing everyone to collaborate at
their own pace without loosing focus, and in such a manner that the
collective can deliver - whether that's a product or a learning
experience!.

\subsection{How to deal with participation in a peeragogy project}

\begin{itemize}
\item
  Accept that some people want to watch what is going on before jumping
  in. This doesn't mean you have to keep them forever. After a while you
  may un-enroll people who don't add any value to the community.~ In our
  Peeragogy project, we've asked people to re-sign up several times (at
  any given juncture, some proprotion prefer to leave).
\item
  Accept that people may only contribute a little: if this contribution
  is good it will add value to the whole
\item
  Understand that you can not impose strict deadlines to volunteers
\item
  Let your work be ``open'' in a sense inspired by
  Wikipedia's~\href{http://en.wikipedia.org/wiki/Wikipedia:Neutral\_point\_of\_view}{Neutral
  Point of View}~policy
\item
  Give roles to participants and define some ``energy centers'' who will
  take the lead on specific items in the project
\item
  Organize regular face to face or online meetings to talk about
  progress and what's needed in upcoming days/weeks
\item
  Ask participants to be clear about when they will be ready to deliver
  their contributions
\item
  Have clear deadlines, but allow contributions that come in after the
  deadline -- in general, be flexible
\item
  Add a~newcomer section~on your online platform to help newbies to get
  started
\end{itemize}
