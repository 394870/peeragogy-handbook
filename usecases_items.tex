\subsubsection{Main actor}

Julian, an enthusiastic convert to the power of peer-learning

\subsubsection{Main success scenario}

\begin{enumerate}
\item
  Reflecting on the success of
  \href{http://socialmediaclassroom.com/host/peeragogy/forum/patterns-and-use-cases\#comment-1749}{Strategy
  as Learning}, Julian notes that other housing associations might
  benefit from this process. He also notes that as most housing
  association boards are made up of volunteers like himself, there is a
  very wide variation in background, knowledge and skills, and therefore
  not only a need for low cost (free) learning opportunities, but a
  range of skills available to enable them.
\item
  Julian sets up a peer learning resource on the web, drawing on the
  experiences in implementing
  \href{http://socialmediaclassroom.com/host/peeragogy/forum/patterns-and-use-cases\#comment-1749}{Strategy
  as Learning}, and promotes it through industry-specific web forums. He
  draws attention from an online journalist writing in the housing field
  who writes a positive article, and as a result a growing number of
  collaborators come forward.
\item
  Over a period of a year or so, the core team of active users
  collaborate to create standards and exemplars in relation to different
  aspects of housing association governance that become a de facto
  standard in the sector.
\end{enumerate}
\subsubsection{Thoughts}

\begin{enumerate}
\item
  Obviously a very specific use case that could easily be generalised
\item
  Possible patterns to extract? Seeding Peer Communities, Emergent
  Standards, Emergent Assessment ???
\end{enumerate}
\subsubsection{Main Actors}

Pierre and Marie - recently married.

\subsubsection{Main Success Scenario}

1. They furnished off an apartment from a Sears \& Roebuck sale. Their
coolerator was crammed with TV dinners and ginger ale. (She couldn't
cook.) 2. But when Pierre found work, the little money coming in worked
out well. They got a hi-fi phono, and boy, did they let it blast --
Seven hundred little EPs, all rock, rhythm and jazz. 3. When the sun
went down, the rapid tempo of the music sort of fell (for various
reasons). 4. They bought a souped up Mercedes -- a cherry red '53 -- and
drove it down to New Orleans to celebrate their anniversary. 5. ``C'est
la vie,'' say the old folks, ``It goes to show you never can tell!''
(Apres Chuck Berry.)

\subsubsection{Thoughts}

I tried to use the familiar song to suggest that pæragogy works in
personal relationships, too. Compare the above story with this quote
from Leopold von Sacher-Massoch\ldots{}: "That woman, as nature has
created her and as man is at present educating her, is his enemy. She
can only be his slave or his despot, but\emph{never his companion}. This
she can become only when she has the same rights as he, and is his equal
in education and work." I don't know if Sacher-Massoch is particularly
reliable as a feminist. But it \emph{is} interesting to look at
``companionship'' (along with membership in the same age cohort) as a
criterion for a peer-like and working relationship in the story. It's
unclear as to whether Pierre \& Marie have ``equal'' roles (he found
work, but it's not in any way implied that she was working\ldots{} so
how did she spend her time? Etc.).

\subsubsection{Main Actor}

Kim, a Ph. D. student in Geography.

\subsubsection{Main success scenario}

\begin{enumerate}
\item
  Kim has 5 different people on her supervision team: some in her field,
  others from geology. They all have somewhat different ideas about what
  she should be doing with her thesis work. None of them are co-located.
  This situation can be quite frustrating.
\item
  Kim decides to go spend a few weeks working in close proximity to the
  one member of the team who she has the most rapport with. This will
  also give her a chance to be in touch with other students in her
  field.
\item
  In the mean time, she establishes contact with yet another researcher
  whose work is quite closely related to hers. Although he does not have
  any formal responsibilities or ties to her project, they are already
  colleagues in an academic sense, and they have more congruent views on
  what her project is about. After she visits her favorite supervisor,
  she may plan to spend a month or so visiting this other researcher in
  his home country.
\end{enumerate}
\subsubsection{Main Actor}

Madeleine, a student who is trying to learn real analysis.

\subsubsection{Main success scenario}

\begin{enumerate}
\item
  Madeleine has been using a peer-learning website for mathematics for a
  while now. When she gets stuck, she asks for help in context, and her
  request is brought to the attention of the appropriate community
  member, who improves the pedagogic quality of the material. This help
  enables her to solve math problems very effectively.
\item
  Now, however, the system's software is being updated. Instead of being
  solely a ``Web 2.0'' system for communicating about the subject, the
  system can keep track of new concepts that Madeleine is using in the
  problems she solves and the questions she asks. It can suggest
  heuristics that have been used by other students solving similar
  problems. (It knows about these things through a combination of
  textual analysis and ``tagging'' of text by Madeleine and other users,
  e.g. Natalie, who sometimes gives comments on problems that Madeleine
  solves.)
\item
  As the system grows and improves (through efforts of students and
  mentors), learning mathematics becomes increasingly easy. The material
  has been gone over by 100s of students and learning pathways are
  optimized. Madeleine sometimes can get a quick tutoring gig helping
  out another younger student, and make some money, but mostly she's
  thinking about what other subjects she will need to add to her
  portfolio in order to become an architect\ldots{} by the time she's
  23!
\end{enumerate}
\subsubsection{Main actor}

Javier, who works for the European Commission.

\subsubsection{Main success scenario}

\begin{enumerate}
\item
  Javier is interested in research topics like ``data analytics'' and
  ``emerging topics in ICT'' -- things that will influence learning
  technology in the next 5 years. He is also concerned about how best to
  fund work on new learning and teaching environments.
\item
  He wonders what the barriers and incentives are in this niche. For
  example, why does research work frequently not have the broad-scale
  societal impact that the EC hopes it might?
\item
  Javier is invited to a peeragogy event, in which some unexpected
  experts on ``broad scale impact'' help him understand that intensive
  funding for research is often not going to have the desired effect,
  since, for various reasons, even well-funded research projects are
  frequently not well connected to actual practice.
\item
  He starts to build peeragogy into funding calls: smaller pots of money
  going to projects that connect with what people actually do, working
  with partners like the Wikimedia Foundation and the Free Software
  Foundation to multiply effort by involving volunteers. It's time for
  him to take a well-earned vacation.
\end{enumerate}
\subsubsection{Main actor}

Jorge Luis is a journalist for a London business paper.

\subsubsection{Main success scenario}

\begin{enumerate}
\item
  Jorge Luis writes on a daily and even hourly basis about the eurozone
  crisis. He uses social dashboards and curating tools and produces lots
  of curated stories about the causes of the problems, the stupidity of
  the continental europeans and how it will all end soon in complete and
  utter disaster. His sources are other journalists, well-known
  economists and famous bloggers.
\item
  On his way to the newsroom he usually passes St Pauls cathedral, where
  Occupy London people protest. He thinks they rather look like losers,
  except for one very interesting young lady. She tells him where he can
  find the center of the universe: at the Whispering Gallery of the
  cathedral. He thinks she is nuts, but also very beautiful and
  interesting, so he walks the 259 steps from ground level to the
  Gallery. Once he gets there, he realizes that the girl was right. It
  IS the center of the universe. There are murmurs to be heard there -
  it seems they come from everywhere. He hears about guilds and the
  craftsmen who built the cathedral. He learns about how proud they were
  and how they formed communities of practice, educating the
  uninitiated, teaching each other to create.
\item
  He returns to ground level. The girl is gone, but yet he feels happy.
  He realizes he can do more then repackage the social media streams,
  that there is more than Twitter-the-new broadcast medium. He starts a
  new journey: finding a guild, a community of practice, but restyled in
  a 21st century fashion. It will be more open, more connected to others
  then the old guilds. He will still use a social dashboard and curaring
  tools, but also he uses wikis, and synchronous communication. And most
  importantly, he starts building, together with others. For instance,
  together with the people formerly known as his readers. They will
  co-create the analysis, the search for solutions and sense-making,
  rather than helplessly listening to ``experts'', passively consuming
  the knowledge and information. Instead, they'll start building their
  own destiny as a community, and the newsroom will be part of the
  platform.
\end{enumerate}
\subsubsection{Main Actor}

Charlie, who does tutoring and educational consulting, and who has been
doing research on paragogy.

\subsubsection{Main success scenario}

\begin{enumerate}
\item
  Charlie usually tutors one-on-one but has been putting work into
  understanding and exploring peer learning and peer production, putting
  it into practice on P2PU and in courses and projects with Howard
  Rheingold.
\item
  X-Y-Z peer learning theory (paragogy?) helps him design learning
  activities that work well for groups of students
\item
  He deploys the new model on paragogy.net as an educational startup,
  and realizes the ``OER dream''!
\end{enumerate}
\subsubsection{Main Actor}

Howard runs \href{http://www.rheingold.com/university/}{Rheingold
University} and teaches courses at UCB and Stanford.

\subsubsection{Main success scenario}

\begin{enumerate}
\item
  Howard created the peeragogy project, as a place to experiment and
  learn: ``I want to experiment as much as possible with peeragogy, with
  the group of contributors here, with the co-learners in Rheingold U,
  and with other groups in the future. I want to personally use the
  tools we're building. I know something about how to do it, and can
  make substantial contributions. But I also am learning a lot about how
  to do it from others, and expect that to continue.''
\item
  Although ``bringing a volunteer project to completion {[}\ldots{}{]}
  isn't a guaranteed slam-dunk'', Howard learns by doing: ``If I had it
  to do over again, I would have thought out the work flow and
  delineated it before we started talking about how to do the project.''
\item
  With both frequent, and other less frequent, but thoughtful,
  contributors, the project continues to develop, and will indeed
  complete somehow (even if no one knew quite what to expect in
  advance). Howard and other contributors have learned a lot in the
  process - and this will be useful both for the duration of the
  peeragogy project, and in future projects. As hoped!
\end{enumerate}
Main Actor: Joe, who is working on a concept map about peer learning.

1. Joe is working on a concept map about peer learning, and notices that
a lot of the patterns that apply to learning would apply to other social
activities. In the end, what's so special about ``learning'' or
``education''? Why should it be separated out from the rest of what
humans do.

2. Certainly education itself has an economic facet to it: for some
people, it's a job, and for many, it means future employability.

3. Can we really discuss methods for ``doing peeragogy'' without also
rethinking the economic and productive aspects of education? Joe decides
that ``paragogy'' should at least be introduced into the ``peeragogy''
concept map.

\textbf{Footnote} This quote from Askins and Pain, ``Contact zones:
participation, materiality, and the messiness of interaction'' (2011) in
the conclusion of our essay
"\href{http://peeragogy.org/to-peeragogy/}{From Peer Learning to
`Peeragogy'}" suggests a ``paragogical'' approach to research within a
``contact zone''. That is, paragogy is research that happens
``alongside''.

\subsubsection{Main Actor}

Jess, a hacker and engineer who develops new libraries and programs
quickly and on the bleeding edge of new technologies.

\subsubsection{Main success scenario}

\begin{enumerate}
\item
  Jess develops something new and totally cool and drops the source code
  in GitHub. These tools are developed rapidly and are a much lighter
  ``learning lift'' than learning say an entirely new programming
  language.
\item
  She creates documentation for her new library and puts it up on a web
  site for other developers to read.
\item
  She is trying to find a better way for other developers to learn how
  to use the new tools and libraries she creates and starts thinking
  about peer learning.
\item
  How can she use what tools and processes or methods that are already
  out there to engage other developers to learn from and with each other
  digitally? (Jess has no background in learning theory and is not in
  the educational field.) She finds the Peeragogy Handbook and things
  start to click.
\end{enumerate}
\subsubsection{Main Actor}

A student, Madeleine, who is trying to learn multivariable calculus.

\subsubsection{Main Success Scenario}

\begin{enumerate}
\item
  Madeleine is enrolled in an advanced calculus course at university.
  She learns about PlanetMath from her instructor who recommends it as a
  place for extra practice with homework problems. Madeleine creates an
  account, fills in basic profile information, and starts solving
  problems that the system supplies based on the information she
  supplied in her profile.
\item
  The problems that the system supplies are automatically linked to
  reference resources in PlanetMath's encyclopaedia. This expository
  material gives Madeleine easy access to the relevant mathematical
  concepts, examples, and hints needed for solving the increasingly
  difficult practice problems. However, she eventually runs into a
  problem where neither the automatically supplied information, nor her
  current knowledge of the subject, is sufficient. She's completely
  stuck on a problem having to do with water flow in a pipe! Madeleine
  attaches a help request to the problem: ``I understand that I have to
  use the two variables x and y to solve for water flow, but I don't
  understand what the boundary limits of the equations would be: do I
  have to convert it to polar coordinates?''
\item
  This request is noticed by Natalie, a mathematics graduate student who
  regularly looks at the feed showing ``recent requests for help with
  advanced calculus.'' She sees that the reference resources linked to
  Madeleine's problem are probably not sufficient, and that Madeleine's
  idea about using polar coordinates would work. Natalie makes some
  changes to the encyclopaedia indicating that converting to converting
  to polar coordinates can be necessary in pipe flow problems, and
  sketches an example. Natalie then checks that this information links
  to Madeleine's problem correctly, and alerts Madeleine to the changes.
  With this new information, Madeleine is not only able to solve her
  problem, but can proceed with confidence: she had the right idea after
  all!
\end{enumerate}
\subsubsection{Main Actor}

Neo, who is a hacker by night, and an office worker by day (and who
reads Baudrillard in his spare time).

\subsubsection{Main Success Scenario}

\begin{enumerate}
\item
  Neo lives in New York City, and works as a programmer in an office
  near Wall Street. His day-job involves finding patterns in market data
  (see this Kevin Slavin's
  \href{http://www.ted.com/talks/kevin\_slavin\_how\_algorithms\_shape\_our\_world.html}{TED
  talk}).
\item
  He has been walking past
  \href{http://en.wikipedia.org/wiki/Zuccotti\_Park}{Zucotti Park} on
  his way home and more or less he finds this protest stuff annoying (he
  has other stuff on his mind). But one of these evenings, one of the
  protestors catches his attention (she's dressed rather
  strikingly\ldots{}). They talk a bit, and he comes away thinking about
  what she said:
  "\href{http://www.nycga.net/files/2011/11/DeclarationFlowchart\_v2\_large.jpg}{All
  our grievances are interconnected.}" What if all the solutions are
  interconnected too?
\item
  Night time: Neo becomes increasingly obsessed with this idea. He's
  pulling down lots of web pages from OWS activists, from companies,
  from government websites -- again, looking for patterns. What would it
  take for OWS folks to solve the problems they worry so much about?
\item
  He eventually stumbles across the idea of peeragogy and it works like
  the ``red pill'': it's possible to solve the problems but only by
  working together. It would be hard to engineer a social media platform
  that will actually help with this (OWS folks mostly use Tumblr and
  aren't necessarily all that technologically minded). But he starts
  working on a
  \href{http://campus.ftacademy.org/wiki/index.php/Free\_Technology\_Guild}{tool}
  that's geared towards learning and sharing skills, while working on
  real projects. At first, it's just hackers who are using the tool, but
  over time they adapt it for popular use. Then things start to get
  interesting\ldots{}
\end{enumerate}
\subsubsection{Introduction}

I think that Peeragogy has flavors -- learning for learning sake for
personal ends in a progression toward learning about the world to take
action as a group. The latter gets heavily into Action Research
(Stringer, 2007), which I love and work heavily in. It is research in
cycles, or loops with feedback to try something, measure it, see how it
worked with the real world, then plan the next question and set of
actions. In each cycle, the group is Learning. I look with that lens at
company start-ups as a perpectual action research cycle. I heard Eric
Reis at SXSW talk about the Lean Startup in this mode, including this
direction in how he even wrote the book. Hypothesis, experiment,
feedback, learn, pivot, next hypothesis\ldots{} Is the group in this
peeragogy learning set knowledge or creating new knowledge? Or through
new knowledge making a change in the world? A great spectrum of
alternatives! Here, my scenario about a company I was on the board on
early on:

\subsubsection{Main actors}

Cycle 1: Nick, an MBA student, plus a Computer Science PhD, John, at a
major university. John had created a unique technology for identifying
video clips and had no idea what to do with it. Nick was an ex-engineer
learning about how to launch new businesses. Cycle 2: Additional
``learners'' and co-teachers as board members, each adding new learning
elements and expertise. Cycle 3+: New learners as investors and clients.

\subsubsection{Main success scenario}

\begin{enumerate}
\item
  Nick and John used a new business plan competition as the catalyst and
  structure to experiment with what ideas might be possible to grow this
  idea. They named it Findable (not the real name; the company did
  launch with some interesting success, but we'll come to that later).
  They brought three other MBAs into the initial group, and within the
  confines of a business plan structure, researched the stereotypical
  elements of a business plan -- addressable market, competition,
  expense and revenue projections, etc. They knew nothing of the area,
  and each person did independent research work to provide some primary
  (interview-based) and secondary (existing text) information about
  their hypothesis of what the technology could do for what audience in
  what environment. They worked hard up until the competition deadline,
  and won the business plan competition, gaining \$15,000 in the process
  plus the attention of some VCs on the judging panel. Each person had
  learned a lot about the technology, the creative process of writing
  the business plan, the rituals involved of asking for money, and the
  flaws in their own plan that they found on its creation. They used
  fairly traditional technology tools: email, shared Word and Excel
  files, telephone, search, and a shared file system to store everything
  that they worked on.
\item
  Nick and Fred wanted to move forward with this project. Their next
  hypothesis was that they could launch this in a specific market. They
  first came to the idea, from the learning from the business plan and
  lots of feedback from the VCs, that they could start with the
  advertising market, as they could now identify and ``tag'' any ad that
  they could find on cable or the internet. They got seed capital from
  three interested parties, who become part of their Action Research
  learning team. They realized to launch that they needed more voices on
  their learning team, so they added their first 3 employees to design
  and sell the product. They also added an advisory board, including
  yours truly, assuming they would be working in the advertising market.
  Technologies? Traditional, though they now included all sorts of tech
  development resources. New information into the mix? They had not put
  together great resources to optimize their time learning, and spent a
  lot of energy keeping up with things, information, and opportunities.
  Learning? Some initial users loved their product, but the market size
  was smaller than they thought\ldots{}plus was very entrenched. The
  companies did not see a real pain point that was being solved.
\item
  Cycle 3 -- what the heck do Nick and Fred do with this? This became
  the true learning phase. Different companies and advisors saw
  different needs for their intriguing product set. They spent 4 years
  (!!!) getting pulled this way and that, using the VC money and needing
  more. (This is VERY much the learning path I see in many small tech
  companies.) Technologies? Same stuff. Learning team? Ebbed and flowed
  with new opportunities and people's patience. My expertise was in the
  ``old'' model, so peaceably left the team (but got options!).
\item
  Cycle 4+ -- a major public company ``found'' them through their
  learning cycles, and found that they solved a pain point. They
  invested a sizeable sum into a chunk of the company, and launched
  their product into that solution. This opened a whole other set of
  learning doors.
\item
  Final cycle -- Happily, I cashed out my options. Two major media
  technology companies ended up buying two areas of key technologies in
  2011, much to my own pocketbook's happiness. Nick and Fred had moved
  on earlier, turning the company learning over to specialized managers.
  I need to see what Nick is up to next\ldots{}.
\end{enumerate}
\subsubsection{Thoughts}

\begin{enumerate}
\item
  Many great patterns were tucked into many cycles of this use case,
  often unspoken assumptions in a new business start-up, including
  environment scanning, codifying specialist knowledge, themes,
  modeling, etc. Consensus building -- an interesting element.
\item
  For me, the additional elements are (a) the scaffolding of the
  ``norms'' of cycles (e.g., business plan creation, a competition, a
  launch of a product) help provide ``norming'' frameworks that can help
  groups achieve as well as limit their looking at the structural norms
  as anything but ``required'' and (b) the lens of Action Research
  Cycles from my own POV. Are we setting a hard limit of providing a
  hypothesis in our co-creation, so we know when we are ``done'' and
  what we have to study? Then once that chunk is done (and CELEBRATED)
  that another hypothesis can be investigated, explored, proven, and
  co-created? I believe that having pre-structured points of learning
  achievement, reflection, and celebration can really help in moving
  forward.
\item
  My own brain is rethinking these issues around content creation after
  hearing Eric Reis speak on how he tested his content creation for his
  \emph{New York Times} best-selling book.
\item
  How are we testing this Handbook, other than living through it? :)
\end{enumerate}
\begin{quote}
"Obviously such a project as Steal This Book could not have been carried
out alone. Izak Haber shared the vision from the beginning. He did
months of valuable research and contributed many of the survival
techniques. Carole Ramer and Gus Reichbach of the New York Law Commune
guided the book through its many stages. Anna Kaufman Moon did almost
all the photographs. The cartoonists who have made contributions include
Ski Williamson and Gilbert Sheldon. Tom Forcade, of the UPS, patiently
did the editing. Bert Cohen of Concert Hall did the book's graphic
design. Amber and John Wilcox set the type. Anita Hoffman and Lynn
Borman helped me rewrite a number of sections. There are others who
participated in the testing of many of the techniques demonstrated in
the following pages and for obvious reasons have to remain anonymous.
There were perhaps over 50 brothers and sisters who played particularly
vital roles in the grand conspiracy. Some of the many others are listed
on the following page. We hope to keep the information up to date. If
you have comments, law suits, suggestions or death threats, please send
them to: Dear Abbie P.0. Box 213, Cooper Station, New York, NY 10003.
Many of the tips might not work in your area, some might be obsolete by
the time you get to try them out, and many addresses and phone numbers
might be changed. \emph{If the reader becomes a participating researcher
then we will have achieved our purpose.}" -- Abbie Hoffman (emphasis
added)
\end{quote}
\subsubsection{Main actors}

The non-executive (Jim, Pamela, Julian) and executive (Clare, Malcolm,
Colin \& Jenny) directors of a housing association (a not-for-profit
organisation building and letting ``social'' housing for families in
housing need)

\subsubsection{Main success scenario}

\begin{enumerate}
\item
  The board of the housing association need to set a strategy that takes
  account of significant changes in legislation, the UK {[}welfare{]}
  benefits system and the availability of long term construction loans.
\item
  Julian, eager to make use of his new-found peeragogical insights
  suggests an approach where individuals research specific factors and
  the team work together to draw out themes and strategic options. As a
  start he proposes that each board member researches an area of
  specific knowledge or interest.
\item
  Jim, the Chairman, identifies questions he wants to ask the Chairs of
  other Housing Associations. Pamela (a lawyer) agrees to do an analysis
  of the relevant legislation. Clare, the CEO, plans out a series of
  meetings with the local councils in the boroughs of interest to
  understand their reactions to the changes from central government.
  Jenny, the operations director, starts modelling the impact on
  occupancy from new benefits rules. Colin, the development director,
  re-purposes existing work on options for development sites to reflect
  different housing mixes on each site. Malcolm, the finance director,
  prepares a briefing on the new treasury landscape and the changing
  positions of major lenders.
\item
  Each member of the board documents their research in a private wiki.
  Julian facilitates some synchronous and asynchronous discussion to
  draw out themes in each area and map across the areas of interest.
  Malcolm, the FD, adapts his financial models to take differet options
  as parameters.
\item
  Clare refines the themes into a set of strategic options for the
  association, with associated financial modelling provided by Malcolm.
\item
  Individual board members explore the options asynchronously before
  convening for an all-day meeting to confirm the strategy.
\end{enumerate}
\subsubsection{Thoughts}

\begin{enumerate}
\item
  This may be a little close to the ``peer production'' end of
  peeragogy, but on the other hand, where (if anywhere) do we draw the
  line?
\item
  This probably needs to be made a little more abstract to be a useful
  use case, and in doing so I suspect will start to overlap with
  \href{http://metameso.org/peeragogy/patterns-usecases/use-cases/peeragogy-helps-solve-complex-problems/}{Peeragogy
  helps solve complex problems}
\item
  It looks to me as if there may be some candidate patterns buried in
  this use case, e.g. Environment Scanning, Codifying Specialist
  Knowledge, Extracting Themes, Modelling Outcomes, Consensus Building
\end{enumerate}
Main Actor: Trinity, the daughter of a Texas oil magnate.

Soundtack:
\href{http://www.youtube.com/watch?v=NWsX9ggfL2Q}{http://www.youtube.com/watch?v=NWsX9ggfL2Q}
1. Trinity has spent the last year traveling around the world to join in
various \#Occupy protests. Her aim is to get people in the movement
thinking about how they can empower themselves. 2. It's tricky though,
because as much as she knows she has an impact on individuals, she still
sees a lot of problems in the world, which, given her manic-depressive
tendencies, she tends to find very disturbing. 3. She reaches out to
other folks who are privileged in one way or another -- and a bunch of
``normal folks'' -- trying not only to bring about political change, but
trying to establish a degree of personal friendship and camaraderie, and
a feeling of ``belonging in the world''. For her, this is a constant
struggle. She finds that working with other people on concrete tasks
keeps her from spiraling into a state of gloom. In the mean time, she's
also building a tremendous amount of knowledge about the way social
movements and political processes work. \textbf{Footnote}: ``The Knife
is now recording a new album to be released in 2012. Lately we have read
a lot about the ongoing discrimination of Romani people in Europe which
is totally unacceptable. The forced evictions must stop and adequate
alternative housing must be arranged. Now!'' --
\href{http://theknife.net/take-action-for-the-housing-rights-of-roma-in-rome}{http://theknife.net/take-action-for-the-housing-rights-of-roma-in-rome}

\subsubsection{Main actor}

Simone is a young media department graduate, who followed the adventures
of the journalist Jorge Luis (see previous post). Jorge Luis was
transforming the newspaper operation into a kind of collective learning
project, turning the newsroom into a platform for discussion and
learning, and inciting the developers to provide an API for external
coders. Simone wrote a paper about all this in her last year at the
media department.\textbf{}She also runs a blog about tools which empower
people to participate in politics (local, nation-wide and
international).

\subsubsection{Main success scenario}

\begin{enumerate}
\item
  Simone loves her blog. She believes verticals and specialization are
  the future in blogging. However, she needs money to live, and to pay
  back the debts she made to finance her studies. Her media department
  was moderately interesting, but nobody ever thought of organizing a
  course ``entrepreneurial blogging/journalism''.
\item
  Posting every day about collaborative online tools such as wikis,
  forums, blogs, mindmaps, synchronous sessions, social bookmarks,
  visualization tools, Simone decides to reach out and look online for
  others who are experiencing the same challenges.
\item
  As she encounters various other people, they start curating stuff
  about blogging business models and best practices. They find lots of
  useful stuff for free at Robin Good's website, and they manage to get
  access to online resources at a strange group which seems to
  specialize in ``mind amplifying tools'' and ``literacies of
  cooperation''. They also discover that ``entrepreneurial journalism''
  is taught at various colleges, and invariably the professors and most
  of the students there indulge in blogging and publishing about their
  insights and experiments. All that material is being discussed on the
  collaborative platform Simone built.
\item
  Simone uses the discussions to blog about her experience. After all,
  issues about financing media who empower people in order to broaden
  and deepen the democracy is something which is rather on topic for her
  own blogging practice. Also, because of her reaching out, her contacts
  increased considerably. She works together with someone to share a
  virtual co-working space, and people start noticing her. Some ask her
  for customized expert advice about collaborative tools and
  collaboration methodologies. The city council expresses some vague
  interest and considers hiring her as a consultant.
\item
  Even though she gets several gigs, Simone realizes it's not easy to
  earn a living as a blogger. But it seems to open other doors\ldots{}
  however, she continues her investigation about business models for
  collaborative media. As yet we don't know whether Simone's blog will
  be profitable in itself, but we do see a network around her projects,
  exchanging insights but also valuable business information and opening
  more doors.
\end{enumerate}
\subsubsection{Thoughts}

I had the opportunity to give some seminars at media departments here in
Belgium. In my experience, the students were not familiar with curation
practices or infotention strategies. They also lack courses in
entrepreneurial journalism. In other words, they're still educated for
the big media companies, but they're not prepared to start the next
TechCrunch or Huffington Post. Often the students asked me, after the
seminar, ``how can we learn all this? they won't teach us these things
here''. I think there is a need for P2P learning about not only
curation, infotention, social dashboards, communities and governance of
common pool recourses, but also about publishing strategies, social
media workflows and business models.
