The peergogical enterprise clearly involves a balance of art and
science, but let's assume we share the assumption that a) individuals
learn by doing; b) that learning is a continuous process; and 3) that
learning is most effective when it contains some form of enjoyment,
satisfaction, or accomplishment. So for each peer-learning participant,
there's a simple question: ''What makes learning fun for me?''

\subsection{How can we learn to learn better?}

Consider the following cases, one with a group learning situation, and
one more of a solo pursuit:

\begin{itemize}
\item
  A study group for a tough class in neuropsychology convenes at at the
  library late one night, resolving to do well on the next day's exam.
  The students manage to deflect their purpose for a while by gossiping
  about college hook-ups and parties, studying for other classes, and
  sharing photos. Then, first one member, then another, takes the
  initiative and as a group, the students eventually pull their
  attention back to the task at hand. They endure the monotony of
  studying for several hours, and the next day, the exam is theirs.
\item
  A young skateboarder spends hours tweaking the mechanics of how to
  make a skateboard float in the air for a split second, enduring
  physical pain of repeated wipeouts. With repetition and success comes
  a deep understanding of the physics of the trick. That same student
  cannot string together more than five minutes of continuous attention
  during chemistry class and spends even less time on homework for the
  class before giving up.
\end{itemize}
\subsection{Why is learning to skateboard fun and chemistry isn't?}

Peer-learning participants succeed when they are motivated to learn.
Skateboarding is primarily intrinsically motivated, with some extrinsic
motivation coming from the respect that kids receive from peers if they
mastered the trick. In most cases, the primary motivation for learning
chemistry is extrinsic, coming from parents and society's expectations
that the student excel and assure their future by getting into a top
college.

The student very well could be intrinsically motivated to have a glowing
report card for her own vanity, but even then, chemistry is not done for
the sake of learning chemistry, but because the student needs that high
grade as part of her overall portfolio.

Taken a different way, what is it about chemistry that's fun for a
student who does love the science? Perhaps she wants the respect, power
and prestige that comes from announcing a new breakthrough; or they may
love the way atoms bond to form new compounds. Or, she may feel her work
is important for the greater good, or prosperity, of humanity.

One way to think about fun learning is that it's fun to learn new
patterns, as Jürgen Schmidhuber wrote: ``A separate reinforcement
learner maximizes expected fun by finding or creating data that is
better compressible in some yet unknown but learnable way, such as
jokes, songs, paintings, or scientific observations obeying novel,
unpublished laws.'' So the skateboarder enjoyed coming across new
patterns (novel tricks) that he was able to learn; tricks that
challenged his current skill level.

\subsection{Learner, know thyself}

An important part of peeragogy is for participants to be as aware as
possible of their unique learning style and what they think they can
contribute to the peer learning experience. Certainly, learning is a
highly personal endeavor and what works for one individual, may miss the
mark for another. When joining the Peeragogy project, I did a brief
self-evaluation about what makes me turn on to learning:

\begin{itemize}
\item
  Context. I resist being groomed for some unforseeable future rather
  than for a purpose. Learning new Japanese words came easily while I
  was living in Japan because it had value for me in my then-current
  milieu. Since I left, the idea of learning fun, new Japanese words has
  not motivated me much. It's far more fun to turn out characters on a
  napkin in a bar with colleagues, than sitting by myself trying to
  master hiragana.
\item
  Timing and sequence. I find learning fun when I'm studying something
  as a way to procrastinate on another pressing assignment. I have never
  been known to spontaneously study accounting concepts in my free time,
  but you can be sure I will have fun learning some tangential fruit of
  the accounting tree, so long as its unrelated to the accounting exam I
  am cramming for tomorrow
\item
  Social reinforcement. Getting tips from peers on how to navigate a
  snowboard around moguls was more fun for me than my Dad showing me the
  proper way to buff the car's leather seats on chore day.
\item
  Visible reward. There's also the learning that's fun, but only later.
  In high school, it was not fun in the moment to sit and compose a
  30-page reading journal for Frankenstein. But owing in part to those
  types of prior experiences, writing is now fun and it's a pleasure to
  learn how to write better. I know that if a pot of gold at the end of
  the rainbow is in sight or I can relate to a parallel experience, I
  can endure the pain of a steep initial learning curve.
\end{itemize}
Yin to that yang, one obvious way learning becomes boring is if you are
forced to do it. Whether by parents or society, being forced to do
something, as opposed to choosing to, ends up making the individual less
likely for success.

Perhaps trying to figure out what makes learning fun is too difficult.
Maybe there is no cut and dry, clear-cut answer. Either way, identifying
what factors can make learning boring will be helpful. Could be that
learning certain things is boring, no matter what, and that that is OK!

References

\begin{enumerate}
\item
  \href{http://www.idsia.ch/\%7Ejuergen/creativity.html}{Formal Theory
  of Creativity \& Fun \& Intrinsic Motivation}, by
  \href{http://www.idsia.ch/\%7Ejuergen}{Jürgen Schmidhuber}
\item
  \href{http://en.wikipedia.org/w/index.php?title=Judo\&oldid=493563588}{The
  Contribution of Judo to Education by Kano Jigoro}
\item
  \href{http://en.wikipedia.org/wiki/The\_Pale\_King}{Pale King},
  unfinished novel by David Foster Wallace
\end{enumerate}
Sources used

\begin{itemize}
\item
  \href{http://paragogy.net/index.php?title=Meta-learning\_as\_a\_font\_of\_knowledge\&oldid=439}{Meta-learning
  as a font of knowledge}
\item
  \href{http://paragogy.net/index.php?title=What\_makes\_learning\_fun\%3F\&oldid=443}{What
  makes learning fun?}
\item
  \href{http://paragogy.net/index.php?title=Paragogy\_Book\_D2\&oldid=735}{Paragogy
  Book D2}
\item
  \href{http://paragogy.net/index.php?title=What\_makes\_learning\_boring\&oldid=442}{What
  makes learning boring \& old}
\end{itemize}
