In the introduction to
"\href{http://socialmediaclassroom.com/host/peeragogy/wiki/organizing-a-learning-context}{Organizing
a Learning Context}``, we remarked that a''learning space" is~\emph{only
potentially}~less structured than a ``course''. ~For example, a library
tends to be highly structured, with quiet rooms for reading, protocols
for checking out books, a cataloging and shelving system that allows
people to find what they are looking for, as well as rules that deter
vandalism and theft. (Digital libraries don't need to play by all the
same rules, but are still structured.)

But more structure does not always lead to better learning. In a 2010
Forbes article titled, ``The Classroom in 2020,'' George Kembel
describes a future in which ``Tidy lectures will be supplanted by messy
real-world challenges.'' The Stanford School of Design, (or ``d.school''
-- which Kemble co-founded and currently directs) is already well-known
for its open collaborative spaces, abundant supply of post-it notes and
markers, and improvisational brainstorm activities -- almost the
opposite of traditional lecture-based learning.

One ``unexpected benefit'' of dealing with real-world challenges is that
we can change our approach as we go. ~This is how it works in peer
learning: peers can decide on different structures not just once (say,
at the beginning of a course), but throughout the duration of their time
together. This way, they are never ``stuck'' with existing structures,
whether they be messy or clean. At least\ldots{} that's the ideal.

In practice, ``bottlenecks'' frequently arise. ~For example, in a
digital library context, there may be bottlenecks having to do with
software development, organizational resources, community good will, or
access to funding -- and probably all of the above. ~In a didactic
context, it may be as simple as one person knowing something that others
do not.

While we can't eliminate scarcity in one stroke, we can design
activities for peer learning that are ``scarcity aware'' and that help
us move in the direction of adaptive learning structures.

\subsection{Planning Peer Learning Activities}

We begin with two simple questions:

\begin{itemize}
\item
  How do we select an appropriate learning activity?
\item
  How do we go about creating a learning activity if we don't find an
  existing one?
\end{itemize}

``Planning a learning activity'' should mean planning
an~\emph{effective}~learning activity, and in particular that means
something that people can and will engage with. ~In short, an
appropriate learning activity may be one that you already do! ~At the
very least, current activities can provide a ``seed'' for even more
effective ones.

\begin{quote}
\emph{Here's a little trick to help you keep focused on things you're
trying to do. Get a bunch of index cards and do this every day:}
\emph{1. Sit down and write down all the things you think you need to do
right then. {[}\ldots{}{]} Write them as short little notes like a ``to
do list''. 2. Then, take the first thing that you can do right now and
do it. Get it done then cross it off the card. 3. Keep doing this, and
if you think of something else you need to do, put it on a card. Just
keep filling them up. 4. At the end of the day, go back through your
card and find any unfinished things and remove any that you'll honestly
never do. 5. The next day, take all the things you didn't do from the
day before and copy them onto a new card, then start with \#1 again.} --
Zed Shaw, in the
\href{http://learnpythonthehardway.org/book/intro.html\#comment-409972596}{Learn
Python the Hard Way forums}
\end{quote}

But when entering unfamiliar territory, it can be difficult to know
where to begin.~ And remember the bottlenecks mentioned above?~ When you
run into difficulty, ask yourself:
\href{http://peeragogy.org/patterns-and-heuristics/}{why is this hard?}~
You might try adapting Zed Shaw's exercise, and make a list of limiting
factors, obstacles, etc., then cross off those which you can find a
strategy to deal with (add an annotation as to why). ~For example, you
might decide to overcome your lack of knowledge in some area by hiring a
tutor or expert consultant, or by putting in the hours learning things
the hard way (Zed would particularly approve of the latter choice).~ If
you can't find a strategy to deal with some issue, presumably you can
table it, at least for a while.

Strategic thinking like this works well for one person. What about when
you're planning activities for someone else? ~Here you have to be
careful: remember, this is peer learning, not traditional ``teaching''
or ``curriculum design''. ~The first rule of thumb for \emph{peer
learning} is: don't plan activities for others unless you plan to to
take part as a fully engaged participant.~ Otherwise, it might be a peer
learning activity, but it's not yours.~ (Perhaps your engagement is just
as ``designer'' -- that's OK.~ But if you don't plan to ``get'' as well
as ``give'', you're not really a peer -- which is perfectly OK, but you
might find other reading material that will serve you better than this
handbook in that case!)

In short, it would be useful to walk through the ``what do you need to
do'' and ``why is it hard'' exercises from the point of view of all of
the participants, keeping in mind that they will, in general, assume
different roles. ~To the extent that you can do so, spell out what these
roles are and what activities comprise them.

For example, in a mathematics learning context, you would be likely to
find people\ldots{}

\begin{itemize}
\item
  solving textbook-style problems
\item
  finding and sharing new problems
\item
  asking questions when something seems too difficult
\item
  fixing expository material to respond to critique
\item
  offering critique and review of proposed solutions
\item
  offering constructive feedback to questions (e.g. hints)
\item
  organizing material into structured collections
\item
  working on applications to real-world problems
\item
  doing ``meta'' research activities that analyse ``what works'' for any
  and all of the above
\end{itemize}

Each one of those activities may be ``hard'' for one reason or another.
~In particular, as a system the different activities tend to depend on
one another. ~If you have people working in a ``student role'' but no
one who can take on a ``TA role'', things will be more difficult for the
students. ~As a~ (co-)organizer, part of \emph{your} job is to try to
make sure all of the relevant roles are covered by someone (who may in
the end wear many hats).

You can further decompose each role into specific concrete activities.
~They might come in the form of instructions to follow: "\emph{How to
write a good critique}" or "\emph{How to write a proof}``.~ They might
come in the form of accessible exercises (where''accessible" depends on
the person``):''\emph{Your first geometry problem}" or
"\href{http://www.ic.unicamp.br/\%7Emeidanis/courses/mc336/2006s2/funcional/L-99\_Ninety-Nine\_Lisp\_Problems.html}{Ninety-Nine
LISP problems}", etc. ~Depending on the features of the learning
context, you may be able to support the written instructions or
exercises with live/in-person feedback (e.g. meta-critique to coach and
guide novice critics, a demonstration, etc.).

\subsection{Our immediate scenario: building activities for the
Peeragogy Handbook}

Adding a bunch of activities to the handbook won't solve all of our
usability issues, but we've agreed that they will help a lot. ~So at
this point, we are revisiting the
\href{http://peeragogy.org/table-of-contents/}{table of contents}~and
thinking about each article or section from this perspective:

\begin{enumerate}
\item
  When looking at this piece of text, what type of knowledge are we (and
  the reader) trying to gain? ~ Technical skills (like learning how to
  edit Final Cut Pro), or abstract skills (like learning how to make
  sense of data)? ~What's the takeaway?~ I.e., what's the point?
\item
  What's difficult here? ~What might be difficult for someone else?
\item
  What learning activity recipes might be appropriate? (See below.)
\item
  What customizations do we need for this particular application?
\end{enumerate}

\textbf{\emph{~}As a quick example: designing a learning activity for
the current page}

\begin{enumerate}
\item
  We want to be able to come up with effective learning activities to
  accompany a ``how to'' article for peer learners. ~These activities
  will extend the ``how to'' aspect from the written word to the world
  of action.
\item
  It might be difficult for some of us to ``unplug'' from all the
  reading and writing that we're now habituated to doing. ~But peer
  learning isn't just about the exchange of text: there are lots and
  lots of ways to learn.
\item
  Like
  \href{http://peeragogy.org/use-cases/paeragogy-helps-solve-complex-problems/}{Neo}~(in
  one of our use cases), it could be useful to ``become more aware of
  the peer learning we do every day''.~ And to think about ``How do you
  learn best?''
\item
  So, the proposed handbook activity is to step away from the handbook
  for a while. ~In fact, why not take a
  \href{http://zenhabits.net/edit-your-life-part-6-a-media-fast/}{media
  fast}~for a given period of time and look at peer learning as a basic
  human activity.~ (Hey, it just sounds to me like you might need to
  unplug, man!)
\end{enumerate}

\subsection{Resources for identifying a dozen or so ``Learning Activity
Recipes'':}

\begin{itemize}
\item
  \href{http://www.kstoolkit.org/KS+Methods}{KS ToolKit}
\item
  ~\href{http://serc.carleton.edu/NAGTWorkshops/coursedesign/tutorial/strategies.html}{Designing
  Effective and Innovative Sources}~(See the section on ``Teaching
  Strategies for Actively Engaging Students in the Classroom'')
\item
  Each of our
  \href{http://peeragogy.org/patterns-and-heuristics/}{patterns and
  heuristics} suggest various activities, like ``practicing the
  heuristics'', ``finding examples of the patterns'', etc.
\item
  Our \href{http://peeragogy.org/use-cases}{Use Cases}~provide many
  hypothetical examples of ``peeragogy in action''.
\end{itemize}

\subsection{Recommended Reading}

\href{http://dschool.stanford.edu/wp-content/uploads/2011/03/BootcampBootleg2010v2SLIM.pdf}{The
d.school Bootcamp Bootleg} (CC-By-NC-SA) includes lots of fun activities
to try. ~Can you crack the code and define new ones that are equally
cool?
