\subsubsection{Main Actor}

Kim, a Ph. D. student in Geography.

\subsubsection{Main success scenario}

\begin{enumerate}
\item
  Kim has 5 different people on her supervision team: some in her field,
  others from geology. ~They all have somewhat different ideas about
  what she should be doing with her thesis work. ~None of them are
  co-located. ~This situation can be quite frustrating.
\item
  Kim decides to go spend a few weeks working in close proximity to the
  one member of the team who she has the most rapport with. ~This will
  also give her a chance to be in touch with other students in her
  field.
\item
  In the mean time, she establishes contact with yet another researcher
  whose work is quite closely related to hers. ~Although he does not
  have any formal responsibilities or ties to her project, they are
  already colleagues in an academic sense, and they have more congruent
  views on what her project is about. ~After she visits her favorite
  supervisor, she may plan to spend a month or so visiting this other
  researcher in his home country.
\end{enumerate}

\subsubsection{Note}

I think this sort of networking to create an informal supervision team
happens fairly frequently for postgrad students in the UK system.
~Certainly there are other examples of distributed project management -
e.g. W3C working groups come to mind.
