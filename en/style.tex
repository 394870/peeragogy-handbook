{\centering\emph{This is a How-To Handbook.}\par}

\subsection{Keep it short}

The easiest sections to read are those that are shorter and include some
kind of visual (video or image) and have some personal connection (i.e.
they tell a story). For anything longer, break it up into sub-pages, add
visuals, make sure each sub-page is accessible to someone (who is it?).
Think clearly of this reader, talk to them.

\subsection{Make it clear}

We'll illustrate this point by example. The original full title of the
book was ``The Peeragogy Handbook: A resource for self-organizing
self-learners''. But
``\href{http://en.wikipedia.org/wiki/Self-organization}{self-organizing}''
is a technical term, and ``self-learner'' is a confusing neologism. We
shouldn't use technical terms unless we explain them. So we really
shouldn't use it in the first sentence or paragraph, or title, of the
book because we'll scare people off or confuse them. If we want to
explain what ``self-organization'' means and why it is relevant for
peeragogy, then we can take a chapter to do that much later on in the
book. At the same time, we shouldn't try to ``say the same thing in a
simpler way.'' We should try to get rid of the technical concept
completely and see what's left. The easiest thing to do in such cases is
to delete the sentence completely and start over: when in doubt, speak
plainly.

\subsection{Don't overdo it with bullet points}

Text can be very hard to read when there are more than a few bullet
points included. Numbered lists should also be used sparingly. It also
seems that when many disjointed bullet points appear, sometimes the
author is really just indexing the main points that are presented better
in someone else's narrative. Therefor, consider replacing an entire
bulleted list with a reference to someone else's book/webpage/chapter.
In today's hyperlinked world, it's easy enough for the reader to go
elsewhere to get good content (and indeed, we should make it easy for
them to find the best treatments around!). It is not very pleasant to
have to \emph{read} a taxonomy.

\subsection{Include activities}

When reading, editing or otherwise working your way through the book,
please make note of any activities or exercises that come to mind, and
share them.~ We're always striving to be more practical and applicable.

\subsection{Don't be overly chatty}

In our efforts to escape from academia-speak and simplify the text in
the handbook, it's important to make sure we are not heading towards the
other extreme -- being too conversational. When we're having a
conversation with someone, we tend to pepper our ideas with transitional
or pivotal phrases (``In any event,'' ``With that said,'' ``As I
mentioned elsewhere,'' etc.) that help to keep the talk flowing. We also
go off on brief tangents before making our way back to the main topic,
and sometimes express ourselves in run-on sentences. While this is
perfectly natural in speech, it can be confusing and complex in written
text. Let's strive for the perfect balance of simple yet professional
writing.

\subsection{Additional style bonus points}

\begin{itemize}
\itemsep1pt\parskip0pt\parsep0pt
\item
  Avoid double lines after paragraphs; this is a leftover from the age
  of typewriters and can create ``rivers'' of white space.
\item
  Capitalize the first word of each item in a bulleted list, especially
  if items include a verb form (this list is an example!).~ Punctuate
  uniformly.
\item
  Capitalize the first word of headings and subheadings; lower case all
  others.
\end{itemize}

\subsection{Format your HTML nicely}

We need to be able to process the content from this Wordpress site and
turn it into various formats like LaTeX and EPUB. Our automated tools
work much better if pages are formatted with simple and uniform HTML
markup. Some key points:

\begin{itemize}
\itemsep1pt\parskip0pt\parsep0pt
\item
  Mark up your links: use~\href{http://peeragogy.org}{The Peeragogy
  Handbook}~instead
  of~\href{http://peeragogy.org}{http://peeragogy.org}.~ It's best if
  the link text is somewhat descriptive.
\item
  Use a numbered list to format your references
  (see~\href{http://peeragogy.org/convening-a-group/}{Convening a
  Group}~for one example of an article that gets this right!)
\item
  Use Heading 2 and Heading 3 tags to mark up sections,
  not~\textbf{bold}~text.~ If you use bold or italics in your
  paragraphs, you should~\textbf{check}~that the markup~\emph{is
  actually correct}. It should exactly surround the words that you're
  marking up
  --~\texttt{\textless{}em\textgreater{}like this\textless{}/em\textgreater{}}~--
  and it should not include extra spaces around marked up words
  --~\texttt{\textless{}em\textgreater{} NOT like this \textless{}/em\textgreater{}.}
\item
  Be aware that Wordpress does not always add paragraph tags to your
  paragraphs.
\item
  Wordpress also tries to replace straight quote marks with ``smart
  quotes'', but it sometimes doesn't achieve the aim. If you notice
  weird quotemarks (especially in the PDF version), you can add smart
  quote marks by hand.
\end{itemize}

~
