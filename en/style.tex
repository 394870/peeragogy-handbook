\subsection{Format your HTML nicely}

We need to be able to process the content from this Wordpress site and
turn it into various formats like LaTeX and EPUB. Our automated tools
work much better if pages are formatted with simple and uniform HTML
markup. Some key points:

\begin{itemize}
\item
  Mark up your links: use \href{http://peeragogy.org}{The Peeragogy
  Handbook} instead of
  \href{http://peeragogy.org}{http://peeragogy.org}. It's best if the
  link text is somewhat descriptive.
\item
  Use a numbered list to format your references (see
  \href{http://peeragogy.org/convening-a-group/}{Convening a Group} for
  one example of an article that gets this right!)
\item
  Wordpress does not automatically add paragraph tags to your
  paragraphs.If you want your text to appear justified and if you want
  the paragraphs to transfer to downstream formats, switch to HTML
  editing mode and wrap individual paragraphs
  with\texttt{ \textless{}p style="text-align: justify;"\textgreater{}...\textless{}/p\textgreater{}}
\item
  Use Heading 2 and Heading 3 tags to mark up sections, not
  \textbf{bold} text. If you use bold or italics in your paragraphs, you
  should \textbf{check} that the markup \emph{is actually correct}. It
  should exactly surround the words that you're marking up --
  \texttt{\textless{}em\textgreater{}like this\textless{}/em\textgreater{}}
  -- and it should not include extra spaces around marked up words --
  \texttt{\textless{}em\textgreater{} NOT like this \textless{}/em\textgreater{}.}
\end{itemize}
\subsection{Keep it short}

The easiest sections to read are those that are shorter and include some
kind of visual (video or image) and have some personal connection (i.e.
they tell a story). For anything longer, break it up into sub-pages, add
visuals, make sure each sub-page is accessible to someone (who is it?).
Think clearly of this reader, talk to them.

\subsection{Make it clear}

We'll illustrate this point by example. The original full title of the
book was ``The Peeragogy Handbook: A resource for self-organizing
self-learners''. But
"\href{http://en.wikipedia.org/wiki/Self-organization}{self-organizing}"
is a technical term, and ``self-learner'' is a confusing neologism. We
shouldn't use technical terms unless we explain them. So we really
shouldn't use it in the first sentence or paragraph, or title, of the
book because we'll scare people off or confuse them. If we want to
explain what ``self-organization'' means and why it is relevant for
peeragogy, then we can take a chapter to do that much later on in the
book. At the same time, we shouldn't try to ``say the same thing in a
simpler way.'' We should try to get rid of the technical concept
completely and see what's left. The easiest thing to do in such cases is
to delete the sentence completely and start over: when in doubt, speak
plainly.

\subsection{Don't overdo it with bullet points}

Maybe this is just a ``pet peeve'', but I find text very hard to read
when there are more than a few bullet points included. For me, it works
better when the bullet points are replaced with numbered lists (which
should still be used sparingly). It also seems that when many disjointed
bullet points appear, sometimes the author is really just indexing the
main points that are presented better in someone else's narrative.
Therefor, consider replacing an entire bulleted list with a reference to
someone else's book/webpage/chapter. In today's hyperlinked world, it's
easy enough for the reader to go elsewhere to get good content (and
indeed, we should make it easy for them to find the best treatments
around!). In particular, it is not entirely pleasant to \emph{read} a
taxonomy. Maybe that sort of thing can be moved into an appendix if we
need to have it.

\subsection{Include activities}

In today's live meeting, we agreed that activities would not magically
solve all possible usability/readability problems, but they are good to
have anyway. And, according to our page layout, each chapter should have
at least one activity (linked to from the sidebar). So, when reading the
book, please make note of any activity that can be included. (Also make
note of problems that \emph{won't} be solved by adding activities!)

\subsection{Don't be overly chatty}

In our efforts to escape from academia-speak and simplify the text in
the handbook, it's important to make sure we are not heading towards the
other extreme -- being too conversational. When we're having a
conversation with someone, we tend to pepper our ideas with transitional
or pivotal phrases (``In any event,'' ``With that said,'' ``As I
mentioned elsewhere,'' etc.) that help to keep the talk flowing. We also
go off on brief tangents before making our way back to the main topic,
and sometimes express ourselves in run-on sentences. While this is
perfectly natural in speech, it can be confusing and complex when being
read (in our handbook or elsewhere). Let's stay conscious of our
audience and try to meet that perfect balance of simple, yet
professional in our writing.

\subsection{Additional style bonus points}

\begin{itemize}
\item
  Avoid double spaces after paragraphs; this is a leftover from the age
  of typewriters and can create ``rivers'' of white space.
\item
  Capitalize the first word of each item in a bulleted list, especially
  if items include a verb form (this list and the one above are
  examples!).
\item
  Capitalize the first word of headings and subheadings; lower case all
  others.
\end{itemize}
