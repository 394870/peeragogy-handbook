\textbf{Summary}: Peeragogy takes a different shape in the corporation.
It's driven by the need to do something, not just to learn about it.
\href{http://vimeo.com/45989903}{Workplace Learning 2} from
\href{http://vimeo.com/user7021511}{Jay Cross} on
\href{http://vimeo.com}{Vimeo}.

\section{Peeragogy in the corporate setting}

\subsection{From training to learning}

A dozen years ago, the words \emph{training} and \emph{learning} were
interchangeable, but today \emph{learning} is revered and
\emph{training} is in the dog house. What's the difference? Training is
something that's pushed on you; someone else is in charge. Learning is
something you choose to do, whether you're being trained or not. You're
in charge. Many knowledge workers will tell you, ``I love to learn, but
I hate to be trained.'' Learning is in keeping with the democratization
of the workplace spawned by the network revolution. Decision making is
passing from the manager to the worker, and part of the deal is that
learning is crowding out training. Corporations are shifting from
top-down training to self-directed learning, from ``push'' to ``pull.''
However, most of that learning is extremely pragmatic: the purpose is
learning to \emph{do} something as opposed to learning \emph{about}
something. However, many corporations would bypass learning altogether
if they could. Executives typically don't want learning; they want
\emph{execution.} They want the job done. They want
\emph{performance.}To a business manager, learning is a means to an end.
If someone were to invent a smart pill that enabled workers to excel at
their jobs without training, that person would make a fortune marketing
smart pharmaceuticals, and most trainers would be out of work.
Conversely, workers often learn more in the coffee room than in the
classroom. They discover how to do their jobs through experience:
talking, observing others, trial and error, and simply working with
people in the know. Formal learning---classes and workshops---conveys
only 5 to 20 percent of what people learn at work. Informal learning is
generally more effective and less expensive than its formal counterpart.
Corporations traditionally over-invest in formal training programs while
neglecting natural, simpler informal processes. People in corporations
already do most of their learning informally, but they do it
unconsciously and without much support. It's haphazard. The notion of
Peeragogy can make this informal, experiential learning more visible and
help improve the informal learning process. Learning is not schooling.
Forget about classes and courses and grades and tests and all the other
school-related paraphernalia we push on children. Most of what we learn,
we learn from other people---parents, grandparents, aunts, uncles,
brothers, sisters, playmates, cousins, Little Leaguers, Scouts, school
chums, roommates, teammates, classmates, study groups, coaches, bosses,
mentors, colleagues, gossips, co-workers, neighbors, and our kids.
Sometimes we even learn from teachers. Informal learning is effective
because it is personal. The individual calls the shots. The learner is
responsible. It's real. How different from formal learning, which is
imposed by someone else. How many learners believe the subject matter of
classes and workshops is ``the right stuff''? How many feel the
corporation really has their best interests at heart?

\subsection{Inspire, don't command}

To extract optimal performance from workers, managers must inspire them
rather than command them. Antoine de Saint-Exupéry put it nicely: ``If
you want to build a boat, do not instruct the men to saw wood, stitch
the sails, prepare the tools and organize the work, but make them long
for setting sail and travel to distant lands.'' Today's free-range
learners are knowledge workers. They expect the freedom to connect the
dots for themselves. Imagine the difference between a free-range
(informal) learner and a (formal) high school student. The high school
student is not allowed to take notes, books, or a cell phone into the
room for the final exam on which their future rides. Happily for us all,
life is unlike high school. It is no longer useful to define learning as
what someone is able to do all on his or her lonesome. Life is not the
same as \emph{Survivor.} Knowledge workers of the future will have
instant, ubiquitous access to the Net. The measure of their learning is
an open-book exam. ``What do you know?'' is replaced with ``What can you
do?'', which is in the process of being supplanted by, ``What can you
and your network connections do?'' Knowledge itself is moving from the
individual to the Net. And peeragogy is supplanting pedagogy.

\subsection{Self-directed, informal learning, in practice}

Corporate culture resists sharing control of learning with learners.
Staid executives recoil at terms like \emph{self-directed} or
\emph{informal} because they imply things are ``out of control.'' This
is an issue of trust. If an executive does not trust workers, informal
can mean \emph{lackadaisical.} However, when executives trust workers,
informal means \emph{natural} and \emph{unconstrained}.

\begin{itemize}
\item
  No matter what we call it, people learn best when they:
\item
  Know what's in it for them and deem it relevant
\item
  Understand what's expected of them
\item
  Connect with other people
\item
  Are challenged to make choices
\item
  Feel safe about showing what they do and do not know
\item
  Receive information in small packets
\item
  Get frequent progress reports
\item
  Learn things close to the time they need them
\item
  Are encouraged by coaches or mentors
\item
  Learn from a variety of modalities (for example, discussion followed
  by a simulation)
\item
  Confront maybes instead of certainties
\item
  Teach others
\item
  Get positive reinforcement for small victories
\item
  Make and correct mistakes
\item
  Try, try, and try again
\item
  Reflect on their learning and apply its lessons
\end{itemize}
\subsection{Learning on demand instead of learning just in case}

Learning things in advance, ``just in case,'' is a losing game. Until
the case arrives, the worker suspects the subject matter won't be
relevant. And when the case does come along, the knowledge acquired in
advance is probably long gone. Knowledge, like muscle tissue,
deteriorates when it's not used. But learning something at the moment of
need couples learning closely with application, and has lasting effects.
When you cannot predict the future and emergence is unpredictable, you
can't build training programs in advance because you don't know what
you'll need. Those who are charged with developing an organization's
talent must rise above the level of training programs. Static programs
do not fare well in a dynamic world. Instead, we should focus on setting
the right conditions for learning. Sometimes there will be a course
thrown in; at other times, a loose collective exercise will prompt
learning, and often managers and trainers can just get out of the way
and let learning happen on its own.

\subsection{Overcoming Stockholm Syndrome}

Bank employees were held hostage for six days by robbers of Kreditbanken
in Stockholm, Sweden, in September of 1973. The hostages became
emotionally attached to their captors and defended them after being
released. About 27\% of victims succumb to what's now called
\href{http://http://en.wikipedia.org/wiki/Stockholm\_syndrome}{Stockholm
Syndrome}. When it comes to learning, many workers suffer from Stockholm
syndrome. They're accustomed to putting their minds on hold,
relinquishing control to their trainers. They leave their curiosity at
the door. They prefer spoon-feeding to foraging on their own. Some
consider workshops the ideal environment for catching up on email;
others treat an off-site meeting as a mini-vacation. Of course, there
are also some great training experiences, some masterful instructors,
and diligent students -- but not enough. Put in the sway of trainers,
most workers passively wait for instructions. Escaping Stockholm
Syndrome in this case requires new skills and attitudes. The attitudes
come from a collaborative culture that values optimism, self-confidence,
curiosity, resilience, purpose, and autonomy. The Internet Time Group is
currently researching the skills required by self-directed learners.
Their list thus far:

\begin{itemize}
\item
  Learning how to learn
\item
  Critical thinking and conceptualization
\item
  Pattern recognition and making meaning
\item
  Design thinking
\item
  Working with one another, co-creation
\item
  Navigating complex environments
\item
  Software literacy
\end{itemize}
\subsection{Summary and Conclusion}

How might this work in a concrete case? If I were an instructional
designer in a moribund training department, I'd polish up my resume and
head over to marketing. Co-learning can differentiate services, increase
product usage, strengthen customer relationships, and reduce the cost of
hand-holding. It's cheaper and more useful than advertising. Were I that
instructional designer, I'd consider tweaking what Google recently did
with their Power Searching course. Almost all of the interaction in that
course took the form of top-down delivery from Dan Russell. The Hangouts
on Google+ had a lot in common with most of today's educational
practice: I answer the question, you answer the question, hundreds of
other people answer the same question. There's no interaction and no
camaraderie. Perhaps a future courses would do more to encourage
collaboration in, and competition among, ad hoc teams\ldots{} or at
least a leader board that awards the most sophisticated searchers.
Again, the emphasis should always be on learning in order to do
something!

\subsection{Recommended Reading}

My colleague Jane Hart has posted a
\href{http://www.c4lpt.co.uk/blog/2012/04/20/is-it-time-for-a-byol-bring-your-own-learning-strategy-in-your-organization-byol/}{parallel
list} of what's required from BYO Learners (who support
\href{http://www.c4lpt.co.uk/blog/2012/04/20/is-it-time-for-a-byol-bring-your-own-learning-strategy-in-your-organization-byol/}{Bringing
Your Own Learning}).

\subsection{References}

Cross, Jay.
\href{http://www.jaycross.com/wp/?portfolio=informal-learning}{Informal
Learning Center}. Hart, Jane.
\href{http://www.c4lpt.co.uk/blog/2012/04/20/is-it-time-for-a-byol-bring-your-own-learning-strategy-in-your-organization-byol/}{Is
it time for a BYOL (Bring Your Own Learning) strategy for your
organization}?
