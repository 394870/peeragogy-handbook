Cultivating a results-oriented peer-learning program in a corporate
learning ecosystem involves a few tweaks of the approach and tools we
discussed in relation to more open, diverse networks.

\subsection{The Workscape, a platform for learning}

Formal learning takes place in classrooms; informal learning happens in
\emph{workscapes.} A workscape is a learning ecology. As the environment
of learning, a workscape includes the workplace. In fact, a workscape
has no boundaries. No two workscapes are alike. Your workscape may
include being coached on giving effective presentations, calling the
help desk for an explanation, and researching an industry on the Net. My
workscape could include participating in a community of field
technicians, looking things up on a search engine, and living in France
for three months. Developing a platform to support informal learning is
analogous to landscaping a garden. A major component of informal
learning is natural learning, the notion of treating people as organisms
in nature. The people are free-range learners. Our role is to protect
their environment, provide nutrients for growth, and let nature take its
course. A landscape designer's goal is to conceptualize a harmonious,
unified, pleasing garden that makes the most of the site at hand. A
workscape designer's goal is to create a learning environment that
increases the organization's longevity and health and the individual's
happiness and well-being. Gardeners don't control plants; managers don't
control people. Gardeners and managers have influence but not absolute
authority. They can't make a plant fit into the landscape or a person fit
into a team. In an ideal Workscape, workers can easily find the people
and information they need, learning is fluid and new ideas flow freely,
corporate citizens live and work by the organization's values, people
know the best way to get things done, workers spend more time creating
value than handling exceptions, and everyone finds their work
challenging and fulfilling.

\subsection{The technical infrastructure of the Workscape}

When an organization is improving its Workscape, looking at consumer
applications is a good way to think about what's required. Ask net-savvy
younger workers how they would like to learn new skills, and they bring
up the features they enjoy in other services:

\begin{itemize}
\item
  Personalize my experience and make recommendations, like Amazon.
\item
  Make it easy for me to connect with friends, like Facebook.
\item
  Keep me in touch with colleagues and associates in other companies, as
  on LinkedIn.
\item
  Persistent reputations, as at eBay, so you can trust who you're
  collaborating with.
\item
  Multiple access options, like a bank that offers access by ATM, the
  Web, phone, or human tellers.
\item
  Don't overload me. Let me learn from YouTube, an FAQ, or linking to an
  expert.
\item
  Show me what's hot, like Reddit, Digg, MetaFilter, or Fark do.
\item
  Give me single sign-on, like using my Facebook profile to access
  multiple applications.
\item
  Let me choose and subscribe to streams of information I'm interested
  in, like BoingBoing, LifeHacker or Huffpost.
\item
  Provide a single, simple, all-in-one interface, like that provided by
  Google for search.
\item
  Help me learn from a community of kindred spirits, like SlashDot,
  Reddit, and MetaFilter.
\item
  Give me a way to voice my opinions and show my personality, as on my
  blog.
\item
  Show me what others are interested in, as with social bookmarks like
  Diigo and Delicious.
\item
  Make it easy to share photos and video, as on Flickr and YouTube.
\item
  Leverage ``the wisdom of crowds,'' as when I pose a question to my
  followers on Twitter or Facebook.
\item
  Enable users to rate content, like ``Favoriting'' an item on Facebook
  or +!ing is on Google or YouTube.
\end{itemize}
Some of those consumer applications are simple to replicate in-house.
Others are not. You can't afford to replicate Facebook or Google behind
your firewall. That said, there are lots of applications you can
implement at reasonable cost. Be skeptical if your collaborative
infrastructure that doesn't include these minimal functions:

\textbf{Profiles} - for locating and contacting people with the right
skills and background. Profile should contain photo, position, location,
email address, expertise (tagged so it's searchable). IBM's Blue Pages
profiles include how to reach you (noting whether you're online now),
reporting chain (boss, boss's boss, etc.), link to your blog and
bookmarks, people in your network, links to documents you frequently
share, members of your network.

\textbf{Activity stream} - for monitoring the organization pulse in real
time, sharing what you're doing, being referred to useful information,
asking for help, accelerating the flow of news and information, and
keeping up with change

\textbf{Wikis} - for writing collaboratively, eliminating multiple
versions of documents, keeping information out in the open, eliminating
unnecessary email, and sharing responsibility for updates and error
correction

\textbf{Virtual meetings} - to make it easy to meet online. Minimum
feature set: shared screen, shared white board, text chat, video of
participants. Bonus features: persistent meeting room (your office
online), avatars.

\textbf{Blogs} - for narrating your work, maintaining your digital
reputation, recording accomplishments, documenting expert knowledge,
showing people what you're up to so they can help out

\textbf{Bookmarks} - to facilitate searching for links to information,
discover what sources other people are following, locate experts

\textbf{Mobile access} - Half of America's workforce sometimes works
away from the office. Smart phones are surpassing PCs for connecting to
networks for access and participation. Phones post most Tweets than
computers. Google designs its apps for mobile before porting them to
PCs.

\textbf{Social network} - for online conversation, connecting with
people, and all of the above functions.

\subsection{Conclusion}

Learning used to focus on what was in an individual's head. The
individual took the test, got the degree, or earned the certificate. The
new learning focuses on what it takes to do the job right. The workplace
is an open-book exam. What worker doesn't have a cell phone and an
Internet connection? Using personal information pipelines to get help
from colleagues and the Internet to access the world's information is
encouraged. Besides, it's probably the team that must perform, not a
single individual. Thirty years ago, three-quarters of what a worker
need to do the job was stored in her head; now it's less than 10\%.
