\attrib{Jan Herder}

\begin{quote}
From this conversational piece you can engage in a journey to affect
your learning space through many points of entry interacting with the
physical one.~ We hope to inspire emerging structure and reciprocal
mentoring to create a ripple effect for those willing to open the door
to a new possible world.
\end{quote}

\subsection{The Guiding Strategy:}

In his \href{http://peeragogy.org/case-study-5ph1nx/}{Peeragogical Case
Study} David Preston states:

\begin{quote}
\emph{Peeragogical interaction requires refining relational and topical
critique, as well as skills in other ``meta'' literacies, including but
not limited to critical thinking, collaboration, conflict resolution,
decision-making, mindfulness, patience and compassion.} 
\sourceatright{From ``Case Study: 5PH1NX'' (this volume).}
\end{quote}

A~\href{http://en.wikipedia.org/wiki/Self_Organised_Learning_Environment}{Self-Organizing
Learning Environment}, or SOLE, with a living structure accomplishes all
of these outcomes, or David's ``meta-literacies,'' simultaneously. An
authentic problem and/or project based activity in a connected learning
environment includes diverse learners in diverse ways by~empowering all
learners as peers. This provides the authentic learning environment with
which to design a SOLE. SOLEs are everywhere. How have we evolved as a
species, if not through self-organizing? A conversation between
strangers is self organizing, each learning about something or each
other. The spaces around people conversing is also~an environment,
though~not explicitly a learning one. While we are always
self-organizing to learn or accomplish things, one place that SOLEs do
not always exist are in learning institutions. In many
educational~institutions, our learning environments are predominately
organized by the teacher, curriculum, or society. How can we nurture
peer to peer learning environments to organize? How does the role of the
teacher differ in a SOLE?~ In what ways~can we~unite that fundamental,
passionate human characteristic of curiosity and self-organizing back
into our Learning Environments? The model that
\href{http://sugatam.wikispaces.com/}{Sugata Mitra} {[}2{]} is
experimenting with gives us some scaffolding to create one ourselves.
This is the goal of his
\href{http://www.ted.com/pages/sole_toolkit}{SOLE Tool Kit} (3).
Sugata's kit is directed towards children between 8 and 12 years old.
I~was wondering if there is a way to make it more universal in its
application. How can I apply it to my situation? How is a SOLE different
in the context of peer to peer learning? This chapter of the Handbook
uses Sugata's model as a doorway into our understanding a SOLE approach
to peer to peer learning.~Its three key components are: learners,
context and project. I find the discussion needs to integrate what we
are learning about diverse learners into a
\href{http://www.udlcenter.org/aboutudl/udlguidelines}{Universal Design
for Learning} {[}4{]} context. After all, we cannot take for granted who
the peers are in the SOLE. Equally, the context, the learning
environment (LE) must be as deeply considered as the learners
participating. As a learning designer, I am also seeking~more clues
about the living structure of a well crafted SOLE.

\subsection{Centers within the Center}

SOLEs exist in a particular context. Take Sugata's
\href{http://www.ted.com/talks/sugata_mitra_shows_how_kids_teach_themselves.html}{hole
in the wall} {[}5{]} experiment. The parameters of the environment of a
computer embedded in a wall in India are very specific. Sugata's act was
to design a project in order to facilitate a process within that
environment. The elements he introduced were a touch screen computer
embedded in a wall with specific software. Sugata has abstracted this
design into a Tool Kit. He speaks of `Child Driven Learning',
intrinsically motivated learning with the curiosity to learn something
in particular. As a learner-centric peeragogy, SOLEs are emergent,
bottom up, seeking to answer: How do we design a project (or phrase a
problem) that ignites a learner's passion? A SOLE is a facilitated
learning environment (LE) that can nurture learner driven activity.~For
instance, in~the Hole in the Wall example, the design is the context of
the wall, the street, the neighborhood --and the facilitation is the
touch screen monitor in the wall. They are brilliantly united. In this
sense it is an intentional,~self-aware learning environment. It is a
strange foreign object that anyone would have to figure out how it works
to take advantage of. But this is not in the classroom, or in the
`school.' It is an informal LE. Just like
\href{http://www.academia.edu/1137269/Game-based_Learning_and_Intrinsic_Motivation}{learning
a game} {[}6{]}, there is an entire ecology that surrounds you. This is
very much a systemic approach. The context is facilitated explicitly
(your design of the SOLE), but also implicitly in the
\href{http://en.wikipedia.org/wiki/Hidden_curriculum}{hidden curriculum}
{[}7{]} that defines your LE. Above is the layout of the
\href{http://www.scribd.com/doc/181089012/Transformed-Learning-Environment-Analysis}{transformed
learning environment} {[}8{]} I explored to work around the hidden
curriculum of the traditional classroom. The LE has a tremendous, if not
\href{http://scholar.lib.vt.edu/theses/available/etd-09232007-220306/unrestricted/SElmasryETDbodytext.pdf}{overwhelming
influence, on learning} {[}9{]}. The first step in connected learning is
to reconnect to the environment around us. For me, the primary context
of my LE is a performing arts center at a small rural liberal arts
college. The Performing Arts Center is a Center within the context of
the college and community. A diversity of spaces within the facility are
inhabited: small and cozy, large and public, technology embedded
everywhere, all focused on the project based learning that emerges
producing a performance. I stay away from a formal classroom as much as
possible. These spaces are Centers within the Center,
\href{http://nourdiab.wordpress.com/2011/02/23/the-theories-of-christopher-alexander/}{`loosely
connected adaptive complex systems}' {[}10{]} within themselves, just
like people. I believe that the possibility of a SOLE emerging as a
living structure seems to depend on the correct types of complex systems
engaged in the LE.

What is the role of the internet in your design? Mitigating inequalities
and accommodating diverse learners~are somewhat assisted by access to
the internet. But it is the immediate,
\href{http://www.wordstream.com/blog/ws/2013/10/02/just-in-time-information-hacks}{just-in-time
learning} {[}11{]} that makes free and open access to the world wide web
so important in a SOLE. Wireless is available throughout this LE. Nooks
and lounges, interconnected, but separate rooms, provide lots of places
for collaboration or solitary work, for staying connected or hiding out.
In a UDL vision of a facilitated peer to peer SOLE, technology is
integral to the design. In the case of my LE, with the use of digital
audio, multi-media, database management, robotic lighting and
\href{http://en.wikipedia.org/wiki/Dichroic_filter}{dichroic} {[}12{]}
colors, learners are accustomed to accessing and augmenting reality with
technology: allowing learners to access their social media is~ part of
their content creation.

Do we start our SOLE as peers? Peer to peer assumes your participants
are peers--especially you, the facilitator. There needs to be enough
diversity and complexity to include all learners, engendering a
\href{http://www.cast.org/library/UDLguidelines/}{Universally Designed
Context} {[}13{]}. What is the role of diversity in peer to peer SOLE
building? How are diverse learners peers? In my LE, I discovered 70\% of
my learners have learning challenges. I know my LE is not unique in this
regard. I have to facilitate a SOLE design that is inclusive. This is in
contradistinction to commonality, yet this diversity is what we crave,
for creativity and innovation, for deep learning to occur. Crafting your
SOLE using multiple means of representation, expression and engagement
empowers learners to be peers. A diverse learning environment,
supporting diverse learning styles and diverse learners, supports a
complex project based SOLE. But there are many SOLEs within the SOLE
since learning is occurring on many levels with each student and within
each group. We do not all get the same thing at the same time. Learning
outcomes are diverse, emergent, serendipitous.

What type of project, problem or event will focus your efforts? Either a
\href{http://www.theatreprof.com/2011/active-learning-student-generated-syllabus/}{learner
generated syllabus {[}}14{]} may emerge from the SOLE, or a
\href{http://usergeneratededucation.wordpress.com/}{user generated
education} {[}15{]} within a specific context may answer this question.
Ownership and leadership emerge when learners can apply their creativity
and/or authentically assist each other in a common goal. Opportunities
to design and modify even small things will draw learners into a
project. The more they must rely on each other, collaborate and share
their creativity, their designs and actualization--the more they work
together as peers. The spaces in your LE are most likely already
designed and built to accommodate the purpose of the facility in the
context of the college or school. We cannot really redesign the actual
space, but we can redesign many aspects. We can look for designs within
it. Being able to design your own space, or project, is critical to
taking ownership of your learning and experiencing the consequences. As
learners mature and look for ways to be more involved, I suggest they
redesign the shop, the repertory lighting plot, or the procedures of
their department and/or SOLE overall. Exchanging roles as designer also
stimulates peer interaction. Why not integrate design and design
thinking? In my context,~lighting, scene, costume and sound design are
interconnected opportunities. Along with accompanying technology, every
opportunity is used to nurture empathy, creativity, rationality and
systems thinking. Integral to the learner generated syllabus or project
design should be continuous artifact creation. A great place to start
the design process and to begin to generate content is by using a
virtual world.

Constant content creation can integrate assessment into your SOLE. It is
the quality of the artifacts created along the way that reveals the
success of your SOLE. Media that chronicles a journey through time,
created by each learner, reveals the depth of participation. It is
nearly impossible to cheat. The learner expresses their comprehension in
the types and~extent of artifact creation.

As the facilitator, I look for opportunities to introduce the
unexpected, bigger questions, deeper considerations, along the way. For
example, in the context of my LE, one of the events feature Tibetan
Monks. They bring a counterpoint to the inflated egos and cult of
personality which is prevalent in our context. The SOLE Plan is
extended. It can happen over a much longer amount of time than one class
or one day. The actors rehearse for weeks, as the design team designs,
giving time for: research, absorption, misleads, mistakes, correction
and reflection. A SOLE needs time and persistence to generate artifacts,
documentation and experiences of the project and virtual worlds are an
excellent way to extend time and space synchronously and asynchronously.

Sugata emphasizes the Big questions. We do not always know what they
are. A focus? A goal? A product? And the event? That should be decided
with the group. The learners intuit the direction that leads to deep
engagement and the bigger questions. I try and leave it ambiguous,
suggesting some of the things they might encounter. Facilitating the
SOLE in this context, we face endless questions connected to the
specific LE, to all the imaginary scenarios, Herculean tasks and
questions-- like building castles, programming a digital sound console,
troubleshooting robotic lighting instruments, how to make the illusion
of fire or, even, who killed Charlemagne? The Box Office is an example
of an informal SOLE that has emerged recurrently over time. I have
noticed that its vitality depends on the characters and the ebb and flow
of learners entering the group or graduating. The physical space is a
small, windowless and often damp room with a couple of couches and a
desk with a computer squeezed in. My very own `Hole in the Wall'
experiment. The bottom of the door can remain closed, while the top is
open, like a stable. Primarily the students are paid to be there,
answering the phone, reserving tickets, greeting patrons and managing
the Box Office and the Front of the House. In the SOLE, this subtle
inversion of the institutional value proposition turns `work study' into
studying work. This is an informal LE nested within the context of the
formal institution and the wider LE: a center within a center. Some
semesters there are business majors working their way up the job ladder:
Usher to Assistant Front of House Manager, to Assistant Box Office
Manager, to Box Office Manager. Sometimes this takes 4 years, sometimes
it happens in a semester or two. It is a recursive SOLE that differs as
the interests and skills of the students who inhabit the space change.
As the current manager puts it, the Box Office is a `constantly evolving
puzzle.'

This example of a SOLE in an informal LE is similar to the other types
of SOLE's that occur within a facilitated LE. The learner's interact as
reciprocal apprentices, leaning on one another to solve challenges and
problems. Groups are self-selective, this type of work suits their
temperament and interests, or time. This cohort is almost a clique,
attracting their boyfriends and girlfriends. They begin initiatives,
re-design the lobby for crowd control, redecorate and rearrange the
space constantly, decide their schedules and split up responsibility.
Everyone is always training everyone, because the environment turns over
each semester. It is explicitly an informal LE. The workers are
students. This inverts the usual state of affairs, where essentially
they are being paid to learn, though they may not even be aware of it.
Occasionally, the learning experience resonates deeply with them. A
number of them have used the experience to leverage jobs that parallel
their interests, or get them started on their careers. Job titles, roles
of responsibility, are often problematic in a SOLE. The bottom line is
that as peers we are all equal and at certain times everyone is expected
to do everything regardless of their roles. Titles go to people's heads.
But this is part of the experience. Keep the titles moving, change it up
when things get bottlenecked over personalities. Sometimes I create
duplicate positions, Assistants of Assistants. and Department Heads. The
Apprenticeship model is at play but in a new way in a SOLE. There are
peers and there are peers. As power struggles emerge, some like-to-like
grouping occurs. The role of the facilitator becomes mediator. The
emergent epistemology of abundance and connected learning asks for a
multitude of `experts.' In the same way, leadership can be distributed,
flowing as varying needs arise.

The experience of practicing leadership skills and encountering all the
variables of working with diverse folks quickly gives feedback to us if
this is a helpful role for this person. It is messy sometimes, and there
are conflicts. After a few events, they learn how to manage a Box
Office,~dealing with patrons, emergencies, complaints and bag check.
They confront the larger peer group, the student body, with authority
and empathy. They are very proud of their jobs and make their own name
tags with titles. A hierarchy gives them rewards that they have been
trained to expect from years in school. It is another way of developing
intrinsic motivation and challenges them to interact with their peers
authentically. As facilitator, I try to leave them alone as much as
possible. The context has been created, the computer in the wall is on a
desk. Extending the design of your SOLE contributes to its living
structure. I have used
\href{http://community.telecentre.org/profiles/blogs/facebook-as-a-supplemental-lms}{Facebook
as a Supplemental LMS} {[}16{]} since 2007 because this~is where my
students are and it allows them to control the structures of groups
emergently. The learners create the groups as they are relevant. The
facilitator does not. Usually they invite me in! For now, Facebook
aggregates the learning community that the SOLE inspires as learners
become leaders, establish connections with each other and mentor
newbies. This activity is integrated into artifact creation, `comments'
and documentation of their personal learning journey. Facebook becomes a
precursor for their portfolios, and in some cases, it is their
portfolio.
\href{http://starwars.wikia.com/wiki/Reciprocal_apprenticeship}{Reciprocal
Apprenticeships} {[}17{]} occur in the dynamic of collaboration among
peers. Continuity in time beyond the event horizon is accomplished by
these relationships. Peers nurture one another along the shared learning
journey that the SOLE provides. As facilitator and designer, you are,
most of all, in a reciprocal relationship with the other learners. This
is the essence of being a peer, an interaction that respects what each
of us brings to the experience.

\subsection{A Review by the progenitor of SOLEs}

\begin{quote}
\textbf{Sugata Mitra}: It is great to see the thinking that has gone
into taking the idea of a SOLE forward. To my mind, SOLEs are quite
experimental at this time and efforts such as these will provide
invaluable data. I look forward to this. I notice that most of the
important design features of a SOLE are incorporated into the article. I
repeat them anyway, just to emphasise: 1. Large, publicly visible
displays are very important, this is what resulted in the surprising
results in the hole in the wall experiments and subsequent SOLEs for
children in England and elsewhere. 2. The absence of unnecessary people
in the learning space, no matter who they are; parents, teachers,
principals, curious adults etc. 3. Free, undirected activity,
conversation and movement. 4. A certain lack of order: I must emphasise
that `Self Organised', the way I use it does not mean `organising of the
self'. Instead it has a special meaning from the subject, Self
Organising Systems, a part of Chaos Theory. The SOLE should be a space
at the `edge of chaos', thereby increasing the probability of the
appearance of `emergent order'.
\end{quote}

\subsection{References}

\begin{enumerate}
\itemsep1pt\parskip0pt\parsep0pt
\item
  \href{http://peeragogy.org/case-study-5ph1nx/}{Case Study: 5PH1NX}
  (Part of the Peeragogy Handbook.)
\item
  \href{http://sugatam.wikispaces.com/}{About Sugata Mitra}
\item
  \href{http://www.ted.com/pages/sole_toolkit}{The SOLE Toolkit}
\item
  \href{http://www.udlcenter.org/aboutudl/udlguidelines}{National Center
  for Universal Design for Learning \textbar{} Universal Design for
  Learning Guidelines}
\item
  \href{http://www.ted.com/talks/sugata_mitra_the_child_driven_education.html}{TED
  Talk \textbar{}~Sugata Mitra: The child-driven education}
\item
  \href{http://www.academia.edu/1137269/Game-based_Learning_and_Intrinsic_Motivation}{Game-based
  Learning and Intrinsic Motivation by Kristi Mead}
\item
  \href{http://en.wikipedia.org/wiki/Hidden_curriculum}{Wikipedia
  \textbar{} Hidden Curriculum}
\item
  \href{http://www.scribd.com/doc/181089012/Transformed-Learning-Environment-Analysis}{Transformed
  Learning Environment Analysis}
\item
  Elmasry, Sarah Khalil. 2007.
  \href{http://scholar.lib.vt.edu/theses/available/etd-09232007-220306/unrestricted/SElmasryETDbodytext.pdf}{Integration
  Patterns of Learning Technologies}. IRB\# 05-295-06.
\item
  Curious :
  \href{http://nourdiab.wordpress.com/2011/02/23/the-theories-of-christopher-alexander/}{In-Forming
  singular/plural design, The Theories of Christopher Alexander}.
\item
  \href{http://www.wordstream.com/blog/ws/2013/10/02/just-in-time-information-hacks}{Overwhelmed
  with Blog Tips? Hack Learning with Just In Time Information}
\item
  \href{http://en.wikipedia.org/wiki/Dichroic_filter}{Wikipedia
  \textbar{} Dichroic Filter}
\item
  \href{http://www.cast.org/library/UDLguidelines/}{CAST \textbar{} UDL
  Guidelines}
\item
  \href{http://www.theatreprof.com/2011/active-learning-student-generated-syllabus/}{Active
  Learning Student Generated Syllabus}
\item
  \href{http://usergeneratededucation.wordpress.com/}{User Generated
  Education Blog}
\item
  \href{http://community.telecentre.org/profiles/blogs/facebook-as-a-supplemental-lms}{Facebook
  as a Supplemental LMS}
\item
  \href{http://starwars.wikia.com/wiki/Reciprocal_apprenticeship}{Reciprocal
  Apprenticeship}
\end{enumerate}
