\paragraph{Definition:} It is very useful to have an up-to-date public
roadmap for the project, a place where it can be discussed and
maintained. The Roadmap exists as an artifact with which to share
current, but never complete, understanding of the space.

\paragraph{Problem:} Without a roadmap, there will not be a shared sense of
the project's goals or working methods. It will be much harder for
people to volunteer to help out, or to assess the project's progress.

\paragraph{Solution:} Keeping a list of current and upcoming activities, as
well as goals and working methods can
help \href{http://peeragogy.org/practice/heuristics/newcomer/}{newcomers}
and old-timers alike see where they can jump in. As we cross things off
the list, this gives a sense of the accomplishments to date, and any
major challenges that lie ahead.

\paragraph{Examples:}

\begin{itemize}
\item
  In the Peeragogy project, once the handbook's outline became fairly
  mature, we could use it as a roadmap, by marking the sections that are
  ``finished'', marking the sections where editing is currently taking
  place, and marking the stubs (possible starting points for future
  contributors). After this outline matured into a
  real \href{http://peeragogy.org/table-of-contents/}{table of contents},
  we started to look in other directions for things to work on, and
  created a \href{http://peeragogy.org/peeragogy-org-roadmap/}{roadmap
  for further development of the website and peeragogy project as a
  whole}.
\item
  There can be a certain roadmappiness to ``presentation of self'', and
  you can learn to use this well. For instance, when introducing
  yourself and your work to other people, you can focus on highlights
  like these: ``\emph{What is the message behind what you're doing?}''
    ``\emph{How do you provide a model others can follow or improve upon?}''
    ``\emph{How can others get directly involved with your project?}''
\end{itemize}

\paragraph{Challenges:} Unless the roadmap is easy for people to see and to
update, they are not likely to use it. In the Peeragogy Accelerator
phase of the project, we've included a roadmap in the ``behind the
scenes'' version of our landing page, we're using it as a way to link to
other documents we're working on. Accordingly, people participating in
the accelerator frequently encounter the roadmap as a ``first level''
object. All of this said, sometimes it's impossible to know in advance
what will happen! A roadmap that's not quite right will feel burdening.
Sometimes it's better to become more open to the unknown.

\paragraph{What's Next:} Our roadmap document, which currently includes
many sub-sections, needs refining and re-outlining. We're hoping that
our work in the Accelerator will inform the 3rd edition of the Peeragogy
Handbook, so it's useful to think about the roadmap as a table of
contents for the book. However, since we are not just interested in
writing activities, the current roadmap will develop in different ways
than the first one did. A shared roadmap is very similar to
a \href{http://peeragogy.org/to-peeragogy/personal-learning-plan/}{Personal
Learning Plan}, or ``paragogical profile''. We made
some \href{http://campus.ftacademy.org/wiki/index.php/Free_Technology_Guild\#Learning_design}{examples}
of these as we worked on the Free Technology Guild, but more work would
have to be done before we have a rich ecosystem of peer learning
profiles that people can use to develop a peer learning plan.
