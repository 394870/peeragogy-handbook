\subsubsection{Main actor}

Jorge Luis is a journalist for a London business paper.

\subsubsection{Main success scenario}

\begin{enumerate}
\item
  Jorge Luis writes on a daily and even hourly basis about the eurozone
  crisis. He uses social dashboards and curating tools and produces lots
  of curated stories about the causes of the problems, the stupidity of
  the continental europeans and how it will all end soon in complete and
  utter disaster. His sources are other journalists, well-known
  economists and famous bloggers.
\item
  On his way to the newsroom he usually passes St Pauls cathedral, where
  Occupy London people protest. He thinks they rather look like losers,
  except for one very interesting young lady. She tells him where he can
  find the center of the universe: at the Whispering Gallery of the
  cathedral. He thinks she is nuts, but also very beautiful and
  interesting, so he walks the 259 steps from ground level to the
  Gallery. Once he gets there, he realizes that the girl was right. It
  IS the center of the universe. There are murmurs to be heard there -
  it seems they come from everywhere. He hears about guilds and the
  craftsmen who built the cathedral. He learns about how proud they were
  and how they formed communities of practice, educating the
  uninitiated, teaching each other to create.
\item
  He returns to ground level. The girl is gone, but yet he feels happy.
  He realizes he can do more then repackage the social media streams,
  that there is more than Twitter-the-new broadcast medium. He starts a
  new journey: finding a guild, a community of practice, but restyled in
  a 21st century fashion. It will be more open, more connected to others
  then the old guilds. He will still use a social dashboard and curaring
  tools, but also he uses wikis, and synchronous communication. And most
  importantly, he starts building, together with others. For instance,
  together with the people formerly known as his readers. They will
  co-create the analysis, the search for solutions and sense-making,
  rather than helplessly listening to ``experts'', passively consuming
  the knowledge and information. Instead, they'll start building their
  own destiny as a community, and the newsroom will be part of the
  platform.
\end{enumerate}
