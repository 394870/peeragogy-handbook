Massive Open Online Courses (MOOCs) are online learning events that can
take place synchronously and asynchronously for months. Participants
assemble to hear, see, and participate in backchannel communication
during live lectures. They read the same texts at the same time,
according to a calendar. Learning takes place through self-organized
networks of participants, and is almost completely decentralized:
individuals and groups create blogs or wikis around their own
interpretations of the texts and lectures, and comment on each other's
work; each individual and group publicises their RSS feed, which are
automatically aggregated by a special (freely available) tool,
gRSShopper. Every day, an email goes out to all participants,
aggregating activity streams from all the blogs and wikis that engage
that week's material. MOOCs are a practical application of a learning
theory known as ``connectivism'' that situates learning in the networks
of connections made between individuals and between texts.

Not all MOOCs are Connectivist MOOCs (or \emph{cMOOCs}). ~Platforms such
as \href{https://www.coursera.org/}{Coursera},
\href{https://www.edx.org/}{edX} and
\href{http://www.udacity.com/}{Udacity} offering MOOCs which follow a
more traditional, centralized approach (these are sometimes called
\emph{xMOOCs}). In this type of MOOC, a professor is taking the lead and
the learning-experience is organized top-down. However, some xMOOCs seem
to adopt a more blended approach. For instance, the course
\emph{\href{https://www.coursera.org/course/edc}{E-learning and Digital
Cultures}} makes use of online spaces beyond the Coursera environment,
and the course organizers want some aspects of participation in this
course to involve the wider social web.

In this chapter we'll focus on cMOOCs.~ One might wonder why a course
would want to be `massive' and what `\emph{massive}' means.
cMOOC-pioneer Stephen Downes explains that his~focus is on the
development of a \emph{network structure}, as opposed to a \emph{group
structure}, to manage the course. In a network structure there isn't any
central focus, for example, a central discussion. That's also the reason
why he considers~the figure of 150 active participants -- \emph{Dunbar's
Number} -- to be the lower cut-off in order to talk about `massive':

\begin{quote}
\textbf{Stephen Downes}: Why Dunbar's number? The reason is that it
represents the maximum (theoretical) number of people a person can
reasonably interact with. How many blogs can a person read, follow and
respond to? Maybe around 150, if Dunbar is correct. Which means that if
we have 170 blogs, then the blogs don't constitute a `core' - people
begin to be selective about which blogs they're reading, and different
(and interacting) subcommunities can form.
\end{quote}

\subsection{A learning theory for the digital
age}\label{a-learning-theory-for-the-digital-age}

Traditionally, scholars distinguish between three
main~\href{http://ryan2point0.wordpress.com/2010/01/12/taxonomy-of-learning-theories/}{categories
of learning theories}: \emph{behaviorism}, \emph{cognitivism} and
\emph{constructivism}. Stephen Downes and others would add a fourth
one:~\href{http://en.wikipedia.org/wiki/Connectivism}{connectivism}, but
this
is~\href{http://en.wikipedia.org/wiki/Talk:Connectivism}{disputed}.~ The
central application of connectivism to date is as a theory of what
happens in Massive Open Online Courses.

The connectivist theory describes learning as a~process of creating
connections and developing networks.~It is based on the premise that
knowledge exists out in the world, rather than inside an individual's
mind. Connectivism sees the network as a central metaphor for learning,
with a node in the network being a concept (data, feelings, images,
etc.) that can be meaningfully related to other nodes. Not all
connections are of equal strength in this metaphor; in fact, many
connections may be quite weak.

On a~practical level, this approach recommends that learning should
focus on where to find information (streams), and how to evaluate and
mash up those streams, rather than trying to enter lots of (perishable)
information into one's skull. Knowing the pipes is more important than
knowing what exactly each pipe contains at a given moment.~ This is the
theory.~ The practice takes place in Connectivist MOOCs (cMOOCs), like
\href{http://change.mooc.ca/about.htm}{Change11}.~ Here, people are free
to participate at will.~ Each week a subject is discussed during
synchronous sessions, which are recorded and uploaded for reference on
the Change11 website. The site also includes an archive of daily
newsletters and RSS-feeds of blog posts and tweets from participants.

cMOOCs tend to be learner-centered. People are encouraged to pursue
their own interests and link up with others who might help them. But the
distributed and free nature of the projects also leads to complaints;
participants often find it confusing when they attempt to follow up on
all the discussions (the facilitators say one should not try to follow
up on \emph{all} the content).

\begin{quote}
\textbf{Stephen Downes}: This implies a pedagogy that (a) seeks to
describe `successful' networks (as identified by their properties, which
I have characterized as diversity, autonomy, openness, and
connectivity); and (b) seeks to describe the practices that lead to such
networks, both in the individual and in society (which I have
characterized as modeling and demonstration (on the part of a teacher)
and practice and reflection (on the part of a learner).
\end{quote}

\subsection{Anatomy of a cMOOC}\label{anatomy-of-a-cmooc}

One example of a MOOC that claims to embody the connectivist theory is
\href{http://change.mooc.ca/index.html}{change.mooc.ca}. The
``\href{http://change.mooc.ca/how.htm}{how it works}'' section of the
site explains what connectivism means in practice.~ The MOOC organizers
developed a number of ways to combine the distributed nature of the
discussions with the need for a constantly updated overview and for a
federated structure. So, if your team wants to organize an open online
course, these are five points to take into consideration:

There is no body of content the participants have to memorize, but the
learning results from activities they undertake.~The activities are
different for each person. A course schedule with suggested reading,
assignments for synchronous or asynchronous sessions~is~provided
(e.g.~using Google Docs spreadsheets internally, Google Calendar
externally; one could also use a wiki), but participants are free to
pick and choose what they work on. Normally there is a topic,
activities, reading resources and often a guest speaker for each week.
One should even reflect upon the question whether a start and end date
are actually needed. It is crucial~to explain the particular philosophy
of this kind of MOOC, and this right from the outset, because chances
are learners will come with expectations informed by their more
traditional learning experiences.

\begin{enumerate}
\def\labelenumi{\arabic{enumi}.}
\item
  It is important to discuss the ``internal'' aspects, such as
  self-motivation: what do the participants want to achieve, what is
  their~larger goal? And what are~their intentions~when they select
  certain activities (rather than other possibilities)? Everyone has her
  own intended outcome. Suggest that participants meditate on all this
  and jot down their objectives. And how can they avoid becoming
  stressed out and getting depressed because they feel they cannot
  ``keep up with all this?'' The facilitators should have a good look at
  these motivations, even if it's impossible to assist every participant
  individually (for large-scale MOOCs).
\item
  Ideally, participants should prepare for this course by acquiring the
  necessary digital skills. ~Which skills are ``necessary'' can be
  decided by the group itself in advance. It's all about selecting,
  choosing, remixing - also called ``curating''. There are lots of tools
  which you can use for this: blogs, social bookmarks, wikis, mindmaps,
  forums, social dashboards, networks such as Twitter with their
  possibilities such as hashtags and lists. Maybe these tools are
  self-evident for some, but not necessarily for all the participants.
\item
  The course is not located in one place but is distributed across the
  web: on various blogs and blogging platforms, on various groups and
  online networks, on photo- and video-sharing platforms, on mindmaps
  and other visualization platforms, on various tools for synchronous
  sessions. This wide variety is in itself an important learning
  element.
\item
  There are weekly~synchronous sessions~(using Blackboard collaborate,
  or similar group chatting tool). During these sessions, experts and
  participants give presentations and enter into discussions. Groups of
  participants also have synchronous meetings at other venues (such as
  Second Life). Try to plan this well in advance!
\item
  Many participants highly appreciate efforts to give~an overview~of the
  proceedings.~Specifically,
  the~\href{http://change.mooc.ca/newsletter.htm}{Daily Newsletter}~is a
  kind of hub, a community newspaper. In that Daily there is also a list
  of the blog posts mentioning the course-specific tag (e.g.
  ``Change11''), also the tweets with hashtag \#change11 are listed in
  the Daily. Of course, the MOOC has~a
  \href{http://change.mooc.ca/index.html}{site}~where sessions,
  newsletters and other resources are archived and discussion threads
  can be read.
\end{enumerate}

From the very beginning of the course, it's necessary to explain
the~importance of tagging~the various contributions, to suggest
a~hashtag.

For harvesting all this distributed content, Stephen Downes advocates
the use of~\href{http://grsshopper.downes.ca/index.html}{gRSShopper},
which is a personal web environment that combines resource aggregation,
a personal dataspace, and personal publishing (Downes developed it and
would like to build a hosted version - eventually financed via
Kickstarter). The gRSShopper can be found on the~registration page,
which is useful primarily for sending the newsletter. It allows you to
organize your online content any way you want, to import content - your
own or others' - from remote sites, to remix and repurpose it, and to
distribute it as RSS, web pages, JSON data, or RSS feeds.

\begin{quote}
\textbf{Stephen Downes}: For example, the gRSShopper harvester will
harvest a link from a given feed. A person, if he or she has admin
privileges, can transform this link into a post, adding his or her own
comments. The post will contain information about the original link's
author and journal. Content in gRSShopper is created and manipulated
through the use of system code that allows administrators to harvest,
map, and display data, as well as to link to and create their own
content. gRSShopper is also intended to act as a fully-fledged
publishing tool.
\end{quote}

Alternatives for registrations: Google Groups for instance. But specific
rules about privacy should be dealt with: what will be the status of the
contributions? In this MOOC the status is public and open by default,
for Downes this is an important element of the course.

\subsection{Technologies}\label{technologies}

Some MOOCs use Moodle, but Downes dislikes the centralization aspect and
it's not as open as it could be, saying ``people feel better writing in
their own space.'' Other possibilities: Google Groups, Wordpress, Diigo,
Twitter, Facebook page, Second Life; but each course uses different
mixtures of the many tools out there. People choose their environment -
whether it is WoW or Minecraft. Students use Blogger, WordPress, Tumblr,
Posterous as blogging tools.

\subsection{RSS harvesting is a key
element}\label{rss-harvesting-is-a-key-element}

Give participants a means to contribute their blogfeed. In
``\href{http://change.mooc.ca/new_feed.htm}{Add a New Feed},'' Downes
explains how to get this structure and additional explanations (via
videos) in order to contribute their blog feed. The administrator in
this case uses gRSShopper to process the content and put it in a
database, process it and send it to other people. Alternatively one can
use Google Reader (the list of feeds is available as an OPML file -
which can be imported to other platforms). There is also a plug-in for
Wordpress that lets you use a Google Doc spreadsheet for the feeds, then
~Wordpress for the aggregation). Many other content management systems
have RSS harvesting features.

Each individual could run her own aggregator, but Downes offers it as a
service.~But aggregators are needed, whether individual, centralized or
both.

\subsubsection{Specialized harvesting}\label{specialized-harvesting}

Using Twitter, Diigo, Delicious, Google Groups, If This Then That
(\href{http://ifttt.com}{IFTTT}) and \href{http://feed43.com}{Feed43}
(take ordinary web page and turn it into an RSS feed).

\subsubsection{Synchronous environments}\label{synchronous-environments}

Synchronous platforms include Blackboard Collaborate (used now for
Change11); Adobe Connect; Big Blue Button; WizIQ; Fuze; WebX;
webcasting; web radio; videoconferencing with Skype or Google Hangout in
conjunction with Livestream or ustream.tv. Or take the Skype/Hangout
audiostream and broadcast is as webradio.~Set up and test ahead of time,
but don't hesitate to experiment.~ Note also, there is a more extensive
discussion of \href{http://peeragogy.org/real-time-meetings/}{real-time
tools} in another section of the handbook.

\subsubsection{Newsletter or Feeds}\label{newsletter-or-feeds}

Feeds are very important (see earlier remarks about the Daily
newsletter). You can use Twitter or a Facebook page, Downes uses email,
he also creates an RSS version through gRSShopper and sends it through
Ifttt.com back to Facebook and Twitter. For the rest of us there is
Wordpress, which you can use to
\href{http://www.wpbeginner.com/wp-tutorials/create-a-free-email-newsletter-service-using-wordpress/\%20}{create
an email news letter}.~ Downs also suggests a handy guide on
\href{http://www.smashingmagazine.com/2010/01/19/design-and-build-an-email-newsletter-without-losing-your-mind/}{how
to design and build an email newsletter without loosing your mind}!

Consider using a content management system and databases to put out
specialized pages and the newsletter in an elegant way, but this
requires a steep learning curve. Otherwise, use blogs / wikis.

\subsubsection{the Use of Comments}\label{the-use-of-comments}

Participants are strongly encouraged to comment on each others' blogs
and to launch discussion threads. By doing so they practice a
fundamental social media skill - developing networks by commenting on
various places and engaging in conversations. It is important to have
activities and get people involved rather than to just sit back and
watch. For an in-depth presentation, have a look
at~\href{http://www.downes.ca/presentation/290}{Facilitating a Massive
Open Online Course}~by Stephen Downes, in which he focuses on research
and survey issues, preparing events, and other essentials.

\subsection{Resources}\label{resources}

\begin{itemize}
\itemsep1pt\parskip0pt\parsep0pt
\item
  Change MOOC: ``How this Course Works'' (\url{http://change.mooc.ca/how.htm})
\item
  \href{http://www.youtube.com/watch?v=eW3gMGqcZQc}{``What is a MOOC''} (video)
\item
  \href{http://www.youtube.com/watch?v=r8avYQ5ZqM0}{``Success in a MOOC''} (video)
\item
  \href{http://www.youtube.com/watch?v=bWKdhzSAAG0}{``Knowledge in a MOOC''} (video)
\item
  \href{http://www.youtube.com/watch?v=mqnyhLfNH3I}{``Change MOOC 11: An introduction and an invitation''} (video)
\end{itemize}

\noindent (We've collected these videos -- all originally uploaded by David Cormier -- in the YouTube playlist accessible here: \url{http://is.gd/peeragogy_mooc_playlist}.)
