\begin{refsection}

This section reprises the ``What's Next'' steps from all of the previous
patterns, offering another view on the Peeragogy project's
\patternname{Roadmap} in a concrete emergent form.

\paragraph{\hyperref[sec:Peeragogy]{Peeragogy}} 
 We intend to revise and extend the patterns and methods of peeragogy to make it a workable model for learning, inside or outside of institutions.

\paragraph{\hyperref[sec:Roadmap]{Roadmap}} 
If we sense that something needs to change about the project, that is a clue that we might need to record a new pattern, or revise our existing patterns.

\paragraph{\hyperref[sec:Reduce, reuse, recycle]{Reduce, reuse, recycle}}
We've converted our old pattern catalog from the \emph{Peeragogy Handbook} into this paper, sharing it with a new community and gaining new perspectives.  Can we repeat that for other things we've made?

\paragraph{\hyperref[sec:Carrying capacity]{Carrying capacity}} 
Making it easy and fruitful for others to get involved is one of the best ways to redistribute the load.  This often requires skill development among those involved; compare the \patternname{Newcomer} pattern.

\paragraph{\hyperref[sec:A specific project]{A specific project}}
We need to build specific, tangible ``what's next'' steps and connect them with concrete action. Use the \patternname{Scrapbook} to organize that process. 

\paragraph{\hyperref[sec:Wrapper]{Wrapper}}
We have prototyped and deployed a visual ``dashboard'' that people can use to get involved with the ongoing work in the project.  Let's improve it, and match it with an improved interaction design for peeragogy.org.

\paragraph{\hyperref[sec:Heartbeat]{Heartbeat}} Identifying and fostering new \patternnameplural{Heartbeat} and new working groups is a task that can help make the community more robust.  This is the time dimension of spin off projects described in \patternname{Reduce, reuse, recycle}.

\paragraph{\hyperref[sec:Newcomer]{Newcomer}} A more detailed (but non-limiting) ``How to Get Involved'' walk-through or ``DIY Toolkit'' would be good to develop. We can start by listing some of the things we're currently learning about.

\paragraph{\hyperref[sec:Scrapbook]{Scrapbook}} 
After pruning back our pattern catalog, we want it to grow again: new patterns are needed.
One strategy would be to ``patternize'' the rest of the \emph{Peeragogy Handbook.}

\subsection*{Summary}

We introduced nine patterns of peeragogy and connected them to
concrete next steps for the Peeragogy project.  In order to
demonstrate the generality of these patterns, we included examples
showing how they manifest in current Wikimedia projects, and 
how the patterns could inform the design of a future university rooted in the
values and methods of peer production.
%
% organization, motivation, and quality
We will close by reviewing our contribution using
three dimensions of analysis borrowed from \cite{benkler2015peer}.

\paragraph{Organization} 
Managing work on our project with design patterns that are augmented with
a ``What's next'' follow through step \emph{decentralizes both goal setting
  and execution} \cite{benkler2015peer}, reintegrating structure in
the form of an emergent \patternname{Roadmap}.  We have aimed to make
our discussion general and our methods extensible enough to work at
varied levels of scale and degrees of formality, inside or outside of
institutional frameworks.

\paragraph{Motivation}  The future university may be
the Chartes of programming, but it will have plenty in common with the
bazaar \cite{raymond2001cathedral}.  As P2PU cofounder Philipp Schmidt
indicates, \emph{learning is at the core of peer production
  communities} \cite{schmidt+commons-based+2009}.  Our patterns help
to explicate the way these communities work, but more importantly,
we hope they will foment a culture of learning.

\paragraph{Quality} 
``By intervening in real communities, these efforts achieve a level of
external validity that lab-based experiments cannot''
\cite{benkler2015peer}.  The ``What's next'' annotation piloted here
will be helpful to other design pattern authors who aim to use
patterns as part of a research intervention.  Peer production is not guaranteed to
  out-compete proprietary solutions
\cite{benkler2015peer,free-software-better}; its potential for
success will depend on the way problems are framed,
and our ability to follow through.

\printbibliography[heading=subbibliography]

\end{refsection}
