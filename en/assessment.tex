\begin{quote}
This article~is about both assessment in peer learning and an exercise
in assessment, as we put our strategy for assessment into practice by
evaluating the~\href{http://peeragogy.org}{Peeragogy Handbook}~itself.
\end{quote}

\subsection{Adapting strategies for learning assessment to the
peer-learning context}

In
``\href{http://books.google.com/books?id=EJxy06yX_NoC\&printsec=frontcover\&source=gbs_atb\#v=onepage\&q\&f=false}{Effective
Grading: A Tool for Learning and Assessment},'' Barbara E. Walvoord and
Virginia Johnson Anderson have outlined an approach to grading. They
address three questions:

\begin{enumerate}
\itemsep1pt\parskip0pt\parsep0pt
\item
  Who needs to know, and why?
\item
  Which data are collected?
\item
  How does the assessment body analyze data and present findings?
\end{enumerate}

The authors suggest that institutions, departments, and assessment
committees should begin with these simple questions and work from them
towards anything more complex. These simple questions provide a way to
understand - and assess - any strategy for assessment! For example,
consider ``formative assessment'' (in other words, keeping track of how
things are going).~ In this context, the answers to the questions above
would be:

\begin{enumerate}
\itemsep1pt\parskip0pt\parsep0pt
\item
  Teachers need to know about the way students are thinking about their
  work, so they can deliver better teaching.
\item
  Teachers gather a lot of these details on learning activities by
  ``listening over the shoulders'' of students.
\item
  Teachers apply analysis techniques that come from their training or
  experience -- and they do not necessarily present their assessments to
  students directly, but rather, feed it back in the form of improved
  teaching.
\end{enumerate}

This is very much a ``teacher knows best'' model! In order to do
something like formative assessment among peers, we would have to make
quite a few adjustments.

\begin{enumerate}
\itemsep1pt\parskip0pt\parsep0pt
\item
  At least some of the project participants would have to know how other
  participants are thinking about their work as well as analyzing their
  own progress. We~are~then able to ``deliver better teaching'' and~work
  together to problem-solve when difficulties arise.
\item
  It may be most convenient for each participant to take on a share of
  the work (e.g.~by maintaining a ``learning journal'' which~might be
  shared with other participants). This imposes a certain overhead, but
  as we remarked elsewhere, ``meta-learning is a font of knowledge!''
  Outside of persistent self-reflection, details about others' learning
  can sometimes be abstracted from their contributions to the project
  (``learning analytics'' is a whole topic unto itself).
\item
  If a participant in a ``learning project'' is bored, frustrated,
  feeling closed-minded, or for whatever other reason ``not learning,''
  then there is definitely a question. But for whom? For the person who
  isn't learning? For the collective as a whole? We may not have to
  ponder this conundrum for long: if we go back to the idea that
  ``learning is adaptation,'' someone who is not learning in a given
  context will likely leave and find another context where they can
  learn more.
\end{enumerate}

This is but one example of an assessment strategy: in addition to
``formative assessment'', ``diagnostic'' and ``summative'' strategies
are also quite popular in mainstream education. The main purpose of this
section has been to show that when the familiar roles from formal
education devolve ``to the people,'' the way assessment looks can change
a lot. In the following section, we offer and begin to implement an
assessment strategy for evaluating the peeragogy project as a whole.

\subsection{Case study in peeragogical evaluation: the Peeragogy project
itself}

We can evaluate this project partly in terms of its main
``deliverable,'' the Peeragogy Handbook (which you are now reading). In
particular, we can ask: Is this handbook useful for its intended
audience? If so, in what ways?~ If not, how can we adapt? The ``intended
audience'' could potentially include anyone who is participating in a
peer learning project, or who is thinking about starting one. We can
also evaluate the learning experience that the co-creators of this
handbook have had. Has working on this book been a useful experience for
those involved? These are two very different questions, with two
different targets for analysis -- though the book's co-creators are also
part of the ``intended audience''. Indeed, we might start by asking
``how has working on this book been useful for us?''

\subsubsection{A methodological interlude: ``Follow the money''}

The metrics for learning in corporations are business metrics based on
financial data. Managers want to know: Has the learning experience
enhanced the workers' productivity?~ When people ask about the ROI of
informal learning, ask them how they measure the ROI of formal learning.
Test scores, grades, self-evaluations, attendance, and certifications
prove nothing. The ROI of any form of learning is the value of changes
in behavior divided by the cost of inducing the change. Like the tree
falling over in the forest with no one to hear it, if there's no change
in behavior over the long haul, no learning took place. ROI is in the
mind of the beholder, in this case, the sponsor of the learning who is
going to decide whether or not to continue investing. Because the figure
involves judgment, it's never going to be accurate to the first decimal
place. Fortunately, it doesn't have to be. Ballpark numbers are solid
enough for making decisions. ~

The process begins before the investment is made. What degree of change
will the sponsor accept as worthy of reinvestment? How are we going to
measure that? What's an adequate level of change? What's so low we'll
have to adopt a different approach? How much of the change can we
attribute to learning? You need to gain agreement on these things
beforehand. Monday morning quarterbacking is not credible. It's
counterproductive to assess learning immediately after it occurs. You
can see if people are~engaged or if they're complaining about getting
lost, but you cannot assess what sticks until the forgetting curve has
ravaged the learners' memories for a few months. Interest also doesn't
guarantee results in learning, though it helps.~ Without reinforcement,
people forget most of what they learn in short order. It's beguiling to
try to correlate the impact of learning with existing financial metrics
like increased revenues or better customer service scores. Done on its
own, this approach rarely works because learning is but one of many
factors that influence results, even in the business world. Was today's
success due to learning or the ad campaign or weak competition or the
sales contest or something else? The best~way to assess how people learn
is to ask them. How did you figure out how to do this? Who did you learn
this from? How did that change your behavior? How can we make it better?
How will you?~ Self-evaluation through reflective practice can build
both metacognition and self-efficacy in individuals and groups. Too time
consuming? Not if you interview a representative sample. For example,
interviewing less than 100 people out of 2000 yields an answer within
10\% nineteen times out of twenty, a higher confidence level than most
estimates in business. Interviewing 150 people will give you the right
estimate 99\% of the time.

\subsubsection{Roadmaps in Peer Learning}

We have identified several basic and more elaborated patterns that
describe ``the Peeragogy effect''.~ These have shaped the way we think
about things since.~ We think the central pattern is the Roadmap,
which can apply at the individual level, as a personal learning plan,
or at a project level.~ As we've indicated, sometimes people simply
plan to see what happens: alternative versions of the Roadmap might be
a compass, or even the ocean chart from the \emph{Hunting of the
  Snark}.~ The roadmap may just be a North Star, or it may include
detailed reasons ``why,'' further exposition about the goal,
indicators of progress, a section for future work, and so forth.~ Our
initial roadmap for the project was the preliminaly outline of the
handbook; as the handbook approached completion at the ``2.0'' level,
we spun off additional goals into a new roadmap for a Peeragogy
Accelerator. Additional patterns flesh out the project's properties in
an open ``agora'' of possibilities.~ Unlike the ocean, our map retains
traces of where we've been, and what we've learned. In an effort to
document these ``paths in the grass,'' we prepared a short survey for
Peeragogy project participants.

We asked people how they had participated (e.g., by signing up for
access to the Social Media Classroom and mailing list, joining the
Google+ Community, authoring articles, etc.) and what goals or
interests motivated their participation.~ We asked them to describe
the Peeragogy project itself in terms of its aims and to evaluate its
progress over the first year of its existence. As another measure of
``investment'' in the project, we asked, with no strings attached,
whether the respondent would consider donating to the Peeragogy
project. This survey was circulated to 223 members of the Peeragogy
Google+ community, as well as to the currently active members of the
Peeragogy mailing list.~ The responses outlining the project's purpose
ranged from the general: ``How to make sense of learning in our
complex times?'' -- to much more specific:

\begin{quote}
\textbf{Anonymous Survey Respondent 1}: Push education further,
providing a toolbox and techniques to self-learners. In the
peeragogy.org introduction page we assume that self-learners are
self-motivated, that may be right but the Handbook can also help them
to stay motivated, to motivate others and to face obstacles that may
erode motivation.
\end{quote}

Considering motivation as a key factor, it is interesting to observe how
various understandings of the project's aims and its flaws intersected
with personal motivations. For example, one respondent (who had only
participated by joining the Google+ community) was: ``{[}Seeking{]}
{[}i{]}nformation on how to create and engage communities of interest
with a shared aim of learning.'' More active participants justified
their participation in terms of what they get out of taking an active
role, for instance:

\begin{quote}
\textbf{Anonymous Survey Respondent 2}: ``Contributing to the project
allows me to co-learn, share and co-write ideas with a colourful mix of
great minds. Those ideas can be related to many fields, from
communication, to technology, to psychology, to sociology, and more.''
\end{quote}

The most active participants justified their participation in terms of
beliefs or a sense of mission:

\begin{quote}
\textbf{Anonymous Survey Respondent 3}: ``Currently we are witnessing
many efforts to incorporate technology as an important tool for the
learning process. However, most of the initiatives are reduced to the
technical aspect (apps, tools, social networks) without any theoretical
or epistemological framework. Peeragogy is rooted in many theories of
cooperation and leads to a deeper level of understanding about the role
of technology in the learning process. I am convinced of the social
nature of learning, so I participate in the project to learn and find
new strategies to learn better with my students.''
\end{quote}

Or again:

\begin{quote}
\textbf{Anonymous Survey Respondent 4}: ``I wanted to understand how
peer production really works. Could we create a well-articulated
system that helps people interested in peer production get their own
goals accomplished, and that itself grows and learns? Peer production
seems linked to learning and sharing - so I wanted to understand how
that works.''
\end{quote}

They also expressed criticism of the project, implying that they may
feel rather powerless to make the changes that would correct course:

\begin{quote}
\textbf{Anonymous Survey Respondent 5}: ``Sometimes I wonder whether the
project is not too much `by education specialists for education
specialists.' I have the feeling peer learning is happening anyway, and
that teens are often amazingly good at it. Do they need `learning
experts' or `books by learning experts' at all? Maybe they are the
experts. Or at least, quite a few of them are.''
\end{quote}

Another respondent was more blunt:

\begin{quote}
\textbf{Anonymous Survey Respondent 6}: ``What problems do you feel we
are aiming to solve in the Peeragogy project? We seem to not be sure.
How much progress did we make in the first year? Some\ldots{} got stuck
in theory.''
\end{quote}

But, again, it is not entirely clear how the project provides clear
pathways for contributors to turn their frustrations into changed
behavior or results. Additionally we need to be entirely clear that we
are indeed paving new ground with our work. If there are proven peer
learning methods out there we have not examined and included in our
efforts, we need to find and address them. Peeragogy is not about
reinventing the wheel. It is also not entirely clear whether excited new
peers will find pathways to turn their excitement into shared products
or process. For example, one respondent (who had only joined the Google+
community) had not yet introduced~current, fascinating projects
publicly:

\begin{quote}
\textbf{Anonymous Survey Respondent 7}: ``I joined the Google+ community
because I am interested in developing peer to peer environments for my
students to learn in. We are moving towards a community-based,
place-based program where we partner with community orgs like our
history museum for microhistory work, our local watershed community and
farmer's markets for local environmental and food issues, etc. I would
love for those local efforts working with adult mentors to combine with
a peer network of other HS students in some kind of cMOOC or social
media network.''
\end{quote}

Responses such as this highlight our need to make ourselves available to
hear about exciting new projects from interested peers, simultaneously
giving them easier avenues to share. Our work on developing a peeragogy
accelerator in the next section is an attempt to address this situation.

\subsubsection{Summary}

\noindent We can reflect back on how this feedback bears on the main sections of
this book with a few more selected quotes.~ These motivate further
refinement to our strategies for working on this project, and help build
a constructively-critical jumping off point for future projects that put
peeragogy into action. \\

\noindent {\large\textbf{Cooperate}}~ \emph{How can we build strong
collaboration?} 

\begin{quote}
``A team is not a group of people who work together.~ A
team is not a group of people who work together.~ A team is a group of
people who trust each other.''
\end{quote}

\noindent {\large \textbf{Convene}}~ \emph{How can we build
a more practical focus?}
\begin{quote}
``The insight that the project will thrive if people are working hard
  on their individual problems and sharing feedback on the process
  seems like the key thing going forward.~ This feels valuable and
  important.''
\end{quote}

\noindent {\large \textbf{Organize}}~ \emph{How to connect with
  newcomers and oldcomers?}
\begin{quote}
``I just came on board a month ago.~ I am designing a self-organizing
  learning environment (SOLE) or PLE/PLN that I hope will help enable
  communities of life long learners to practice digital literacies.''
\end{quote}

\noindent {\large \textbf{Assess}}~ \emph{How can we be effective and
relevant?}
\begin{quote}
``I am game to also explore ways to attach peeragogy to spaces where
  funding can flow based on real need in communities.''
\end{quote}

\subsection{Conclusion}

We can estimate individual learning by examining the real problems
solved by the individual.~ It makes sense to assess the way groups solve
problems in a similar way.~ Solving real problems often happens very
slowly, with lots of practice along the way.~ We've learned a lot about
peer learning in this project, and the assessment above gives a serious
look at what we've accomplished, and at how much is left.
