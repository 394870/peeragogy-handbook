\emph{Authors}: Joe Corneli and David Preston

\subsubsection{Summary}

This article will be about both (a) assessment in peer learning and (b)
an exercise in assessment, as we will try to put our strategy for
assessment into practice by evaluating the
\href{http://peeragogy.org}{Peeragogy Handbook} itself.

\subsection{Thinking about ``contribution''}

It is intuitive to say: ``learning is adaptation.'' What else would it
be?

Further, since adaptation happens not just on the individual level, but
also on the socio-cultural level -- anthropologists use the phrase
``adaptive strategy'' as a synonym for ``culture'' -- we can say that
contributions to social adaptation are ``paragogical.''

\subsection{Adapting strategies for learning assessment to the
peer-learning context}

In
"\href{http://books.google.com/books?id=EJxy06yX\_NoC\&printsec=frontcover\&source=gbs\_atb\#v=onepage\&q\&f=false}{Effective
Grading: A Tool for Learning and Assessment}," Barbara E. Walvoord and
Virginia Johnson Anderson have outlined an approach to grading. They
address three questions:

\begin{enumerate}
\item
  Who needs to know, and why?
\item
  Which data are collected?
\item
  How does the assessment body analyze data and present findings?
\end{enumerate}

The authors suggest that institutions, departments, and assessment
committees should begin with these simple questions and work from them
towards anything more complex. These simple questions provide a way to
understand - and assess - any strategy for assessment! For example,
consider "formative assessment:

\begin{quote}
``\ldots{}which involves constantly monitoring student understanding
through a combination of formal and informal measures. Teachers ask
searching questions, listen over the shoulders of students working
together on a problem, help students assess their own work, and
carefully uncover students' thinking {[}and{]} react to what they learn
by adjusting their teaching, thereby leading students to greater
understanding.'' (Quote from the website for the book ``New Frontiers in
Formative Assessment''.)
\end{quote}

In this context, our answers to the questions above would be:

\begin{enumerate}
\item
  Teachers need to know about the way students are thinking about their
  work, so they can deliver better teaching.
\item
  Teachers gather lots of details on learning activities by ``listening
  over the shoulders'' of students.
\item
  Teachers apply (hopefully well-informed) analysis techniques that come
  from their training or experience -- and they do not necessarily
  present their assessments to students directly, but rather, feed it
  back in the form of improved teaching.
\end{enumerate}

This is very much a ``teacher knows best'' model! In order to do
something like formative assessment among peers, we would have to make
quite a few adjustments.

\begin{enumerate}
\item
  At least some of the project participants would have to know how
  participants are thinking about their work. We might not be able to
  ``deliver better teaching,'' but perhaps we could work together to
  problem-solve when difficulties arise.
\item
  It may be most convenient for each participant to take on a share of
  the work, e.g. by maintaining a ``learning journal'' (which could be
  shared with other participants). This imposes a certain overhead, but
  as we remarked elsewhere, ``meta-learning is a font of knowledge''!
  Outside of self-reflection, details about others' learning can
  sometimes be abstracted from their contributions to the project
  (``learning analytics'' is a whole topic unto itself).
\item
  If a participant in a ``learning project'' is bored, frustrated,
  feeling closed-minded, or for whatever other reason ``not learning'',
  then there is definitely a question. But for whom? For the person who
  isn't learning? For the collective as a whole? We may not have to
  ponder this conundrum for long: if we go back to the idea that
  ``learning is adaptation'', someone who is not learning in a given
  context will likely leave, and find another context where they can
  learn more.
\end{enumerate}

This is but one example of an assessment strategy: in addition to
``formative assessment'', ``diagnostic'' and ``summative'' strategies
are also quite popular in mainstream education. The main purpose of this
section has been to show that when the familiar roles from formal
education devolve ``to the people'', the way assessment looks can change
a lot. In the following section, we offer and begin to implement an
assessment strategy for evaluating the peeragogy project as a whole.

\subsection{Case study in peeragogical evaluation: the Peeragogy project
itself}

We can evaluate this project partly in terms of its main
``deliverable,'' the Peeragogy Handbook (which you are now reading). In
particular, we can ask: Is this handbook useful for its intended
audience? The ``intended audience'' could potentially include anyone who
is participating in a peer learning project, or who is thinking about
starting one. We can also evaluate the learning experience that the
co-creators of this handbook have had. Has working on this book been a
useful experience for those involved? These are two very different
questions, with two different targets for analysis -- though the book's
co-creators are also part of the ``intended audience''. Indeed, we might
start by asking ``has working on this book been useful for us?''

For me (Joe) personally, it has been useful:

\emph{to see some more abstract, conceptual, and theoretical ideas
(paragogy.net) extended into practical advice (which I'm sure I can
personally use), with references to literature I would not have come up
with in library or internet searches, and with a bunch of ideas and
insights that I wouldn't have come up with on my own. I definitely
intend to use this handbook further in my work.}

It's true; I do see myself as one of the more involved participants to
date, which stands to reason since I'm actually paid to research peer
learning, and this project is (in my opinion) one of the most
cutting-edge places to talk about that topic! If ``you get out of it
what you put into it'' is true, then, again, as a major contributor, I
think I ``deserve'' a lot. And I'm certainly not the only one: quite a
large number of person-hours have been poured into this project by quite
a number of volunteers. This should say something!

Nevertheless, one does not need to be a ``handbook contributor'' at all
to get value from the project: if it were otherwise, we might as well
just get rid of the book after writing it. Actually, our thought is that
this work will indeed have ``value'' for downstream users, and our
choice of legal terms around the book reflects that idea. Anyone
downstream is free to use the contents of this book for any purpose
whatsoever. For all we know, there will be future users who will add
much more to the study and practice of paragogy/peeragogy than any of us
have so far. This could happen by putting the ideas to the test, feeding
back information on the results to the project
(\href{http://peeragogy.org/contact/}{please do}! - the ultimate
assessment of the Peeragogy Handbook will be based on what people
actually~\emph{do}~with it): perhaps further developing the book,
developing additional case studies or recipes, and so forth.

In fact, questions about ``usefulness'' are what we aim to study in our
``alpha testing'' phase (which is beginning now!).

\subsection{Conclusion}

We can estimate individual learning by examining the real problems
solved by the individual. Sometimes those are solved in collaboration
with others. If someone only consumes information, they may well be
``learning'', but there is no way for us to measure that. On the other
hand, if they only solve ``textbook problems'', again, they may be
learning and gaining intuition (which is good), but it is still not
100\% clear that they are actually learning anything ``useful'' until
they start solving problems that they really care about! So, to assess
learning, we do not just measure ``contribution'' (in terms of quantity
of posts or what have you) but instead we measure ``contribution to
solving real problems''. Sometimes that happens very slowly, with lots
of practice along the way. Furthermore, at any given point in time, some
of the ``problems'' are actually quite fun and are ``solved'' by
playing! Indeed (as people like Piaget and Vygotsky recognized), if
we're interested to know real experts on learning, we should talk with
kids, since they learn tons and tons of things.

\subsubsection{Recommended reading}

\begin{itemize}
\item
  Chris Morgan, Meg O'Reilly,
  \href{http://books.google.com/books/about/Assessing\_Open\_and\_Distance\_Learners.html?id=wZcihyWRdIIC}{Assessing
  Open and distance learners (1999), Open University}
\item
  Jan Philipp Schmidt, Christine Geith, Stian Håklev, and Joel
  Thierstein,
  \href{http://www.irrodl.org/index.php/irrodl/article/view/641/1389}{Peer-To-Peer
  Recognition of Learning in Open Education}
\item
  L.S. Vygotsky:
  \href{http://books.google.com/books?id=RxjjUefze\_oC\&printsec=frontcover\&source=gbs\_atb\#v=onepage\&q\&f=false}{Mind
  in Society: Development of Higher Psychological Processes}
\item
  \href{http://org.sagepub.com/search?author1=Reijo+Miettinen\&sortspec=date\&submit=Submit}{Reijo
  Miettinen}~and~\href{http://org.sagepub.com/search?author1=Jaakko+Virkkunen\&sortspec=date\&submit=Submit}{Jaakko
  Virkkunen},~\href{http://org.sagepub.com/content/12/3/437.abstract}{Epistemic
  Objects, Artifacts and Organizational
  Change},~\emph{Organization,}~May 2005 ,12:~437-456.
\end{itemize}

\subsection{Supplement: An overview of assessment topics}

\begin{itemize}
\item
  Diagnostic, formative and summative evaluation
\item
  Competency-based learning
\item
  Experiential learning
\end{itemize}

\subsubsection{UNIT OF ANALYSIS}

\begin{itemize}
\item
  individual
\item
  group/team
\item
  class
\item
  course
\item
  program
\item
  organization
\end{itemize}

\subsubsection{Purpose}

\begin{itemize}
\item
  diagnostic
\item
  formative
\item
  summative
\end{itemize}

\subsubsection{Feedback source}

\begin{itemize}
\item
  peer
\item
  pedadogical authority
\item
  content expert
\item
  group
\item
  public
\end{itemize}

\subsubsection{Models}

\begin{itemize}
\item
  Peer assessment
\item
  Self-assessment
\item
  Norm-referenced testing
\item
  Criterion-referenced testing
\item
  Information-referenced testing
\item
  Writing
\item
  Transmedia/e-portfolios
\end{itemize}

\subsubsection{Other considerations}

\begin{itemize}
\item
  Suitability to task
\item
  Suitability to learner's desired/expected outcomes (e.g., ``If I want
  to master a skill, I need more expert/critical/constructive feedback
  than someone clicking a `like' button.'')
\item
  Capital: time, money, energy, ROI
\item
  Future documentary usage
\item
  professional guidelines
\end{itemize}

\subsubsection{Further reading}

\begin{itemize}
\item
  \href{http://www.nclrc.org/essentials/assessing/peereval.htm}{Peer and
  self-assessment}~(from National Capital Language Resource Center)
\item
  Steven Jay
  Gould's~\emph{The
  Mismeasure of Man}
\item
  \href{http://en.wikipedia.org/wiki/Self-\_and\_Peer-Assessment}{Wikipedia
  entry on peer and self-assessment}
\item
  \href{http://www.elearning-reviews.org/topics/pedagogy/assessment/1999-boud-et-al-peer-learning-assessment/}{Assessment
  as it relates to peer learning in university courses}
\item
  \href{http://www.eric.ed.gov/ERICWebPortal/detail?accno=ED501727}{Self,
  peer, and group assessment in e-learning}
\item
  \href{http://www.hepg.org/hep/book/151}{New Frontiers in Formative
  Assessment}
\end{itemize}
