\paragraph{Definition:} While we could imagine an ideal information
processing system that would (magically) come with all solutions
pre-built, a more realistic approach recognizes that real problem
solving always takes time and energy.

\paragraph{Problem:} Given a difficult problem, we usually want to take a
shortcut.

\paragraph{Solution:} Magical thinking robs a context of its ``process'' or
``motion''. The more completely we fall back on ``traditional'' modes of
doing things (including magical ones) the less we stand to learn. It's
also true that traditions and habits can serve a useful function: they
can massively simplify and streamline, and adopting some healthy habits
can free up time and energy, making learning possible. But if we try
something new and imagine that things work the way they always have
(e.g.~sign up for a course and get told what to do, then do it and
pass), we can run into trouble when the situation doesn't match our
preconceptions.

\paragraph{Example:} Joe Corneli's 2011 DIY Math course at P2PU went quite
badly. Students signed up hoping to learn mathematics, but none of them
had very concrete goals about what to learn, or very developed knowledge
about how to study this subject. This was what the class was supposed to
help teach. However, it seemed as if the students felt that signing up
for the course would ``magically'' give them the structure they needed.
Still, it's not as if the blame can be placed entirely on the students
in this case. Building a learning space with no particular structure and
saying, ``go forth and self-organize!''~is not likely to work, either.
The one saving grace of DIY Math is that the course post-mortem informed
the development of the paragogy principles (see page \pageref{paragogy-principles}): it was not a mistake we
would repeat again.

\paragraph{Challenges:} If we already ``knew'', 100\%, how to do peeragogy,
then we would not stand to learn very much by writing this handbook.
Difficulties and tensions would be resolved ``in advance''. We know
this, but readers may still expect ``easy answers''.

\paragraph{What's Next:} Fast-forwarding a few years from the DIY Math
experiment: as part of the PlanetMath project, we are hoping to build a
well-thought-through example of a peer learning space for mathematics.
One of the ideas we're exploring is to use patterns and antipatterns
(exactly like the ones in this catalog) as a way not only of designing a
learning space, but also of talking about the difficulties that people
frequently run into when studying mathematics. Building an initial
collection of Calculus Patterns may help give people the guide-posts
they need to start effectively self-organizing.


