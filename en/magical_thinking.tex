\subsubsection{Introduction}

While we could imagine an ideal information processing system that would
(magically) come with all solutions pre-built, a more realistic approach
recognizes that real problem solving always takes time and energy. For
instance, if we ``knew'', 100\%, how to do peeragogy, then we would not
stand to learn very much by writing this handbook. Difficulties and
tensions would be resolved ``in advance''. The relevant problem solving
approach and associated ``learning orientation'' will depend on the task
and resources at hand.

\subsubsection{Magical Thinking is the thief of process}

Magical thinking of the kind described above robs a context of its
``process'' or ``motion''. The more structure we have in advance, the
more completely we fall back on ``traditional'' modes of doing things,
and the less we stand to learn. It's also true that traditions and
habits can serve a useful function: they can massively simplify and
streamline, and adopting some healthy habits can free up time and
energy, making learning possible {[}1{]}. But it's still going to take
work. Time for a few deep breaths?

\subsubsection{Reference}

\begin{enumerate}
\item
  Dias-Ferreira, Eduardo, \emph{et al}. ``Chronic stress causes
  frontostriatal reorganization and affects decision-making.''
  \emph{Science} 325.5940 (2009): 621-625.
\end{enumerate}
