\subsubsection{Introduction}

While the ideal platform would (magically) come with solutions
pre-built, a more realistic approach recognizes that problem solving
always takes time and energy. ~The problem solving approach and
associated~``learning orientation'' will also depend on the task and
resources at hand. {[}\ldots{}{]} Arguably, if we ``knew'', 100\%, how
to do peeragogy, then we would not stand to learn very much by writing
this handbook.~ Difficulties and tensions would be resolved ``in
advance'' (see earlier comments about ``magical'' technologies for peer
production).

\subsubsection{Magical Thinking is the thief of process}

Magical thinking of the kind described above robs a context of its
``process'' (Nishida might say, its ``motion'').~ It seems possible that
the more structure we have ``in advance'', and the more we can fall back
on ``traditional'' modes of doing things, the less we stand to learn.~ I
quote at length:

\begin{quote}
"\emph{Optimization of decision-making processes confers an important
advantage in response to a constantly changing environment. The ability
to select the appropriate actions on the basis of their consequences and
on our needs at the time of the decision allows us to respond in an
efficient way to changing situations. However, the continuous control
and attention that this process demands can result in an unnecessary
expenditure of resources and can be inefficient in many situations. For
instance, when behavior is repeated regularly for extensive periods
without major changes in outcome value or contingency, or under
uncertain situations where we cannot manipulate the probability of
obtaining an outcome, general rules and habits can be advantageous.
Thus, the more rapid shift to habits after chronic stress could be a
coping mechanism to improve performance of well-trained behaviors, while
increasing the bioavailability to acquire and process new information,
which seems essential for adaptation to complex environments. However,
when objectives need to be re-updated in order to make the most
appropriate choice, the inability of stressed subjects to shift from
habitual strategies to goal-directed behavior might be highly
detrimental. Such impairment might be of relevance to understand the
high comorbidity between stress-related disorders and addictive behavior
or compulsivity, but certainly has a broader impact spanning activities
from everyday life decisions to economics.}" --
\href{http://www.sciencemag.org/content/325/5940/621.full}{Science
Magazine}
\end{quote}

This also has interesting implications when it comes to ``detecting
learning'' (see
"\href{http://peeragogy.org/to-peeragogy/researching-peeragogy/}{researching
peeragogy}").~ How do emotions, stress, learning, habit, and adaptation
relate?
