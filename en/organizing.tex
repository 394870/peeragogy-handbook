This section about organizing Co-Learning rests on the assumption that
learning always happens in a context, whether this context is a
structured ``course'' or a (potentially) less structured ``learning
space''. For the moment we consider the following division:

\begin{itemize}
\item
  \emph{Organizing Co-learning Contexts}
  \begin{itemize}
  \item
    Courses (= ``learning linked to a timeline or syllabus'')
  \item
    Spaces (= ``learning not linked to a timeline or syllabus'')
  \end{itemize}
\end{itemize}
This section focuses on existing learning contexts and examines in
detail how they have been ``organized'' by their (co-)creators. At a
``meta-level'' of media, we can talk about this parallel structure:

\begin{itemize}
\item
  \emph{Building Co-learning Platforms}
  \begin{itemize}
  \item
    Development trajectories (e.g. ``design, implement, test, repeat'')
  \item
    Platform features (e.g. forums, wikis, ownership models, etc.)
  \end{itemize}
\end{itemize}
A given learning environment with have both time-like and space-like
features as well as both designed-for and un-planned features. A given
learning platform will encourage certain types of engagement and impose
certain constraints. The question for both ``teachers'' and ``system
designers'' -- as well as for learners -- should be: \emph{what features
best support learning?}

The answer will depend on the learning task and available resources.

For example, nearly everyone agrees that the best way to learn a foreign
language is through immersion. But not everyone who wants to learn, say,
French, can afford to drop everything to go live in a French-speaking
country. Thus, the space-like full immersion ``treatment'' is frequently
sacrificed for course-like treatments (either via books, CDs, videos, or
ongoing participation in semi-immersive discussion groups).

System designers are also faced with scarce resources: programmer time,
software licensing concerns, availability of peer support, and so forth.
While the ideal platform would (magically) come with solutions
pre-built, a more realistic approach recognizes that problem solving
always takes time and energy. The problem solving approach and
associated ``learning orientation'' will also depend on the task and
resources at hand. The following sections will develop this issue
further through some specific case studies.

\subsection{Case study 1 (pilot, completed): ``Paragogy'' and the After
Action Review.}

In our analysis of our experiences as course organizers at P2PU, we (Joe
Corneli and Charlie Danoff) used the US Army's technique of After Action
Review (AAR). To quote from
\href{http://paragogy.net/ParagogyPaper2}{our paper} {[}2{]}:

\begin{quote}
As the name indicates, the AAR is used to review training exercises. It
is important to note that while one person typically plays the role of
evaluator in such a review {[}\ldots{}{]} the review itself happens
among peers, and examines the operations of the unit as a whole.

The four steps in an AAR are:

\begin{enumerate}
\item
  Review what was supposed to happen (training plans).
\item
  Establish what happened.
\item
  Determine what was right or wrong with what happened.
\item
  Determine how the task should be done differently the next time.
\end{enumerate}
The stated purpose of the AAR is to ``identify strengths and
shortcomings in unit planning, preparation, and execution, and guide
leaders to accept responsibility for shortcomings and produce a fix.''

\end{quote}
We combined the AAR with several principles (see Discussion section
below), which we felt described effective peer learning, and went
through steps 1-4 for each principle to look at how well it was
implemented at P2PU. This process helped generate a range of advice that
could be applied at P2PU or similar institutions. By presenteding our
paper at the \href{http://okfn.org/okcon/}{Open Knowledge Conference
(OKCon)}, we were able to meet P2PU's executive director, Philipp
Schmidt, as well as other highly-involved P2PU participants; our
feedback may have contributed to shaping the development trajectory for
P2PU.

In addition, we developed a strong prototype for constructive engagement
with peer learning that we and others could deploy again. In other
words, variants on the AAR and the paragogical principles could be
incorporated into future learning contexts as platform features {[}3{]}
or re-used in a design/administration/moderation approach {[}4{]}. For
example, we also used the AAR to help structure our writing and
subsequent work on \href{http://paragogy.net}{paragogy.net}.

\subsection{Case Study 2 (in progress): ``Peeragogy''.}

Our particular focus in the interviews was on drawing out and
emphasizing the relational dimension of students, learning experiences
within their environment and, consequently, on inferring from their
accounts a sense of how they perceived and indeed constituted their
environment. We asked them who they learned with and from and how. A
further question specifically focused on whom they regarded as their
peers and how they understood their peers as a source and a site for
learning." {[}1{]}

In this section, we will interview and/or survey members of the
Peeragogy community with questions similar to those used by Boud and Lee
{[}1{]} and then identify strengths and shortcomings as we did with the
AAR above. These questions are derived from the AAR.

\subsection{Questions}

These were discussed, refined, and answered on an etherpad: revisions to
the original set of questions are marked in italics.

\begin{enumerate}
\item
  Who have you learned with \emph{or} from in the Peeragogy project?
  \emph{What are you doing to contribute to your peers' learning?}
\item
  How have you been learning during the project?
\item
  Who are your peers in this community, and why?
\item
  What were your expectations of participation in this project?
  \emph{And, specifically, what did you (or do you) hope to learn
  through participation in this project?}
\item
  What actually happened during your participation in this project (so
  far)? \emph{Have you been making progress on your learning goals (if
  any; see previous question) -- or learned anything unexpected, but
  interesting?}
\item
  What is right or wrong with what happened (Alternatively: how would
  you assess the project to date?)
\item
  How might the task be done differently next time? (What's ``missing''
  here that would create a ``next time''\emph{, ``sequel'', or
  ``continuation''?})
\item
  \emph{How would you like to use the Peeragogy handbook?}
\item
  \emph{Finally, how might we change the questions, above, if we wanted
  to apply them in your peeragogical context?}
\end{enumerate}
\subsubsection{\textbf{Reflections on participants' answers}}

The questions were intended to help participants reflect on, and change,
their practice (i.e. their style of participation). There is a tension,
however, between changing midstream and learning what we might do
differently next time. There is a related tension between initial
structure and figuring things out as we go. Arguably, if we knew, 100\%,
how to do peeragogy, then we would not learn very much in writing this
handbook. Difficulties and tensions would be resolved ``in advance''
(see earlier comments about ``magical'' technologies for peer
production).

And yet, despite our considerable collected expertise on collaboration,
learning, and teaching, there have been a variety of tensions here!
Perhaps we should judge our ``success'' partly on how well we deal with
those. Some of the tensions highlighted in the answers are as follows:

\begin{enumerate}
\item
  \emph{Slow formation of ``peer'' relationships.} There is a certain
  irony here: we are studying ``peeragogy'' and yet many respondents did
  not feel they were really getting to know one another ``as peers'', at
  least not yet. Those who did have a ``team'' or who knew one another
  from previous experiences, felt more peer-like in those relationships.
  Several remarked that they learned less from other individual
  participants and more from ``the collective'' or ``from everyone''. At
  the same time, some respondents had ambiguous feelings about naming
  individuals in the first question: ``I felt like I was going to leave
  people out and that that means they would get a bad grade - ha!'' One
  criterion for being a peer was to have built something together, so by
  this criterion, it stands to reason that we would only slowly become
  peers through this project.
\item
  \emph{``Co-learning'', ``co-teaching'', ``co-producing''?} One
  respondent wrote: ``I am learning about peeragogy, but I think I'm
  failing {[}to be{]} a good peeragog. I remember that Howard {[}once{]}
  told us that the most important thing is that you should be
  responsible not only for your own learning but for your peers'
  learning. {[}\ldots{}{]} So the question is, are we learning from
  others by ourselves or are we {[}\ldots{}{]} helping others to
  learn?'' Another wrote: ``To my surprise I realized I could contribute
  organizationally with reviews, etc. And that I could provide some
  content around PLNs and group process. Trying to be a catalyst to a
  sense of forward movement and esprit de corps.''
\item
  \emph{Weak structure at the outset, versus a more ``flexible''
  approach.} One respondent wrote: ``I definitely think I do better when
  presented with a framework or scaffold to use for participation or
  content development. {[}\ldots{}{]} (But perhaps it is just that I'm
  used to the old way of doing things).'' Yet, the same person wrote:
  ``I am interested in {[}the{]} applicability {[}of pæragogy{]} to new
  models for entrepreneurship enabling less structured aggregation of
  participants in new undertakings, freed of the requirement or need for
  an entrepreneurial visionary/source/point person/proprietor.'' There
  is a sense that some confusion, particularly at the beginning, may be
  typical for peeragogy. With hindsight, one proposed ``solution'' would
  be to ``have had a small group of people as a cadre that had met and
  brainstormed before the first live session {[}\ldots{}{]} tasked
  {[}with{]} roles {[}and{]} on the same page''.
\item
  \emph{Technological concerns.} There were quite a variety, perhaps
  mainly to do with the question: how might a (different) platform
  handle the tension between ``conversations'' and ``content
  production''? For example, will Wordpress help us ``bring in'' new
  contributors, or would it be better to use an open wiki? Another
  respondent noted the utility for many readers of a take-away PDF
  version. The site (peeragogy.org) should be ``{[}a{]} place for people
  to share, comment, mentor and co-learn together in an ongoing
  fashion.''
\item
  \emph{Sample size.} Note that answers are still trickling in. How
  should we interpret the response rate? Perhaps what matters is that we
  are getting ``enough'' responses to make an analysis. One respondent
  proposed asking questions in a more ongoing fashion, e.g., asking
  people who are leaving: ``What made you want to quit the project?''
\end{enumerate}
With regard to Points 1 and 2, we might use some ``icebreaking''
techniques or a ``buddy system'' to pair people up to work on specific
projects. The project's ``teams'' may have been intended to do this, but
commitment or buy-in at the team level was not always high (and in many
cases, a ``team'' ended up being comprised of just one person). It does
seem that as the progress has progressed, we have begun to build tools
that could address Point 3: for example, the Concept Map could be
developed into a process diagram that would used to ``triage'' a project
at its outset, help project participants decide about their roles and
goals. Point 4 seems to devolve to the traditional tension between the
``good enough'' and the ``best'': we have used an existing platform to
move forward in an ``adequate'' way. And yet, some technological
improvements may be needed for future projects in pæragogy.
(Furthermore, note that our choice to use a CC0 license means that if
other people find the content useful, they are welcome to deploy it on
their own platform, if they prefer.) Finally, Point 5 is still up in the
air (more answers more be coming in shortly - I think I have sent around
enough reminders). Hopefully the questionnaire will be useful to the
group even with a not-100\% response rate! Points 4 and 5 are related,
in that an ongoing questionnaire for people leaving (or joining) the
project could be implemented as a fairly simple technology, which would
provide feedback for site maintainers. Gathering a little information as
a condition of subscribing or unsubscribing seems like a safe,
light-weight, way to learn about the users (tho there is always the
possibility that rather than unsubscribing, non-participating users will
just filter messages from the site).

An underlying tension (or synergy?) -- between learning and producing --
was highlighted in our earlier work on paragogy. If we learn by
producing, that is good. However, I have argued in {[}4{]} that
paragogical praxis is based less on producing and more on reusing. If
downstream users of this handbook find it to, indeed, be useful, we may
have done enough. \emph{For all we know, we are the ``cadre'' (see
above) charged with determining how best to do things in ``subsequent
rounds''!} And, with this, we turn to a third case study, where our work
so far is reapplied in an offline educational context.

\subsection{Discussion}

\begin{quote}
\textbf{Lisewski and Joyce}: In recent years, the tools, knowledge base
and discourse of the learning technology profession has been bolstered
by the appearance of conceptual paradigms such as the `five stage
e-moderating model' and the new mantra of `communities of practice'.
This paper will argue that, although these frameworks are useful in
informing and guiding learning technology practice, there are inherent
dangers in them becoming too dominant a discourse.

\end{quote}
In a sense, the more rigid we are about the form of a given a pattern,
the less we learn by deploying it (see the
\href{http://peeragogy.org/practice/antipatterns/magical-thinking/}{Magical
Thinking} anti-pattern). If we were trying to ``validate'' the paragogy
principles simply by fitting feedback to them, that would be an act of
intellectual dishonesty. But, nevertheless, we can use the model in a
constructive and creative way, and be ready to go beyond it: after all,
we are still learning what makes this stuff work.

{\footnotesize
\ctable[pos = p, center, botcap, framerule=2pt]{p{.45\textwidth}p{.45\textwidth}}
{% notes
}
{% rows
\FL
\textbf{Paragogical Principles\ldots{}} & \textbf{Reflections on
practice and experience suggest\ldots{}}
\\\noalign{\medskip}
1. \emph{Changing context as a decentered center.} \par \medskip
\par
[We interact by changing the space.] & \emph{It seems we begin with \emph{w}eak ties, and
then experience a slow formation of ``peer'' relationships, as we form
and re-form our social context, and come to better understand our
goals.}
\\\noalign{\medskip}
2. \emph{Meta-learning as a font of knowledge.}\par \medskip
\par
[We interact by changing what we know about ourselves.] & \emph{We learn a lot about ourselves by interacting with
others. But participants struggle to find the right way to engage:}
\emph{``co-learning'', ``co-teaching'', or ``co-producing''? Moreover,
``People come--they stay for a while, they flourish, they build--and
they go.''}
\\\noalign{\medskip}
3. \emph{Peers provide feedback that would not be there otherwise.}\par \medskip
\par
[We interact by changing our perspective on things.] & \emph{We begin with a weak structure at the outset but this
may afford a more ``flexible'' approach as time goes on (see also this
\href{http://peeragogy.org/adding-structure-with-activities/}{handbook
section} which offers advice on designing activities that help create a
``flexible structure'').}
\\\noalign{\medskip}
4. \emph{Paragogy is distributed and nonlinear.}\par \medskip
\par [We interact by changing
the way things connect.] & \emph{There are a number of technological concerns, which
in a large part have to do with tensions between ``content production''
and ``conversation'', and to a lesser extent critique the platforms
we're using.}
\\\noalign{\medskip}
5. \emph{Realize the dream if you can, then wake up!}\par \medskip
\par [We interact by
changing our objectives.] & \emph{Even
with a small group, we can extract meaningful ideas about peer learning
and form a strong collective effort, which moves things forward for
those involved: this means work. We would not get the same results
through ``pure contemplation''.}\label{principle-table}
\LL
}}

This table helps to emphasize something we saw in the pattern catalog:
in practice, what we really have is a patchwork collection of tricks or
heuristics. The paragogy principles are themselves a non-linear
interface that we can plug into and adapt where appropriate. Instead of
a grand narrative, we see a growing collection of case studies. As we
share our experiences and make needed adaptations, our techniques for
peer learning and peer production become more robust.

\subsection{References}

\begin{enumerate}
\item
  Boud, D. and Lee, A. (2005).
  \href{http://manainkblog.typepad.com/faultlines/files/BoudLee2005.pdf}{`Peer
  learning' as pedagogic discourse for research education}.
  \emph{Studies in Higher Education}, 30(5):501--516.
\item
  Joseph Corneli and Charles Jeffrey Danoff,
  \href{http://ceur-ws.org/Vol-739/paper\_5.pdf}{Paragogy}, in Sebastian
  Hellmann, Philipp Frischmuth, Sören Auer, and Daniel Dietrich (eds.),
  \emph{Proceedings of the 6th Open Knowledge Conference, Berlin,
  Germany, June 30 \& July 1, 2011},
\item
  Joseph Corneli and Alexander Mikroyannidis (2011).
  \href{http://greav.ub.edu/der/index.php/der/article/view/188/330}{Personalised
  and Peer-Supported Learning: The Peer-to-Peer Learning Environment
  (P2PLE)}, \emph{Digital Education Review}, 20.
\item
  Joseph Corneli,
  \href{http://paragogy.net/ParagogicalPraxisPaper}{Paragogical Praxis},
  \emph{E-Learning and Digital Media} (ISSN 2042-7530), Volume 9, Number
  3, 2012
\item
  Lisewski, B., and P. Joyce (2003). Examining the Five Stage
  e-Moderating Model: Designed and Emergent Practice in the Learning
  Technology Profession, \emph{Association for Learning Technology
  Journal}, 11, 55-66.
\end{enumerate}
