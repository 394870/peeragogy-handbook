\attrib{Miguel Ángel Pérez Álvarez \& Elisa
Armendáriz \\ \emph{with}\\ Paola Ricuarte \& Joseph Corneli}

\begin{quote}
Learning to use technology with peers -- the case of Students With
Abilities in Technology (SWATs).
\end{quote}

\subsection{Part 1: Introduction}

Mind-amplifying technologies {[}1{]}, technologies of cooperation
{[}2{]}, such as conversation technologies, as well as visualization
tools, video and photo edition software, simulators or programming
technologies are emerging learning tools in schools around the world.
They are affordable and accessible enough to design learning
environments. Latin America is no exception and it is fast becoming the
norm to find convergent technology in the classroom.

We challenged students to develop a three-level game with a score or
marker using Scratch, a program developed by the MIT. This program
allows you to develop computer programs using modules or blocks of
instructions. The educational value of this tool lies not in its ease of
use but in its nature as an authentic learning environment and ideal
context for developing intellectual skills.

Once students have developed their programs and documented the process
in a learning log, we asked them if they had faced problems in handling
Scratch. In this way we were able to identify which students had
difficulties in developing programs and what their problems were in the
process of choosing instructions. ~However, we're also able to identify
those students with particular technical skills. ~We call them Students
With Abilities in Technology (SWATs). ~In case of difficulty, SWATs can
be called in, and can decide if they want to give advice to peers and
the teacher in the use of Scratch.

The idea of identifying these students and asking them to support their
peers and teachers in specific tasks has an additional educational
component. It is clear that when a student is given the task of
explaining or advising peers or teachers, he develops new competences
and masters, to an even greater extent, those competences for which
he/she was selected as SWAT.

We have observationally determined that this approach is relevant to the
widespread use of digital devices in academic tasks and its extended
application contributes to a more positive use of digital technologies
for learning. We see how, as the use of technology in all learning
environments becomes general, this approach of peer learning becomes an
alternative to underpin the work of teachers. The figure of the SWAT in
the classroom also enables a different form of relationship between
pairs that generate new forms of interaction and learning that we can
appraise and evaluate.

\subsection{Part 2. Representation as a pattern}

Here's how the short case study could be described using the pattern
template that we've presented in the book.  It's pretty easy to turn a
case study into a design pattern!

\paragraph{Title:} Students With Abilities in Technology (SWATs)

\paragraph{Context:}
Private and public schools increasingly have digital devices in
classrooms with Internet access (laptops, desktops, tablets, cell
phones, etc.) and \textbf{teachers with little or no expertise in the
  educational use of such devices} . Furthermore, the teachers are
often embedded in a \textbf{framework for teaching and learning} that
doesn't take into account the ``horizontal'' modes of thinking that
modern technology enables.  \textbf{However, some of the students do
  have considerable background with these kinds of tools.}

\paragraph{Problem:}
In general, teachers have multiple deficiencies in the adoption of
emerging technologies. Their lack of expertise prevents them from
realizing the full potential that technology has as a relevant
pedagogical mediation. ~The rate of change in the school context,
however, is not coupled to the rate of change in current teacher
training programs. This lack of pedagogical training is having a
majorly disruptive impact in the classroom given this presence of
technological devices in the classroom. ~The reaction of
administrators and teachers to the proliferation of devices is, in a
significant number of cases, rejection and stigmatization of emerging
technologies.  Teachers may be ignoring the educational potential of
technology, and ignoring the way technology has changed the cognitive
model of a whole generation. ~Having paid technical specialists to
support the work of the teacher in the classroom is unthinkable from
an economic standpoint.

\paragraph{Solution:}
The students themselves can be the solution to this problem. Some have
superior technical knowledge and this is usually wasted. Teachers can
incorporate them as assistants to help them and their peers. A student
with digital skills can be the agent of change that many teachers need
in order to learn how to use technology for the design of learning
environments. ~This empowered group of students, that we named SWAT
(Students With Abilities in Technology) support teachers and peers with
lower-level digital competences. Support from students with technical
knowledge could mean a significant change in the learning process,
because teachers can now combine that knowledge with their teaching
experience and pedagogical strategies. The result of this can be the
discovery of the many possibilities technology has for the construction
of knowledge and the development of new intellectual abilities.

In our work with middle school students (ages 12-14), the support of
SWATs inside and outside the classroom was a very positive experience.
At the beginning of each course students are required to develop
projects involving the use of technology. Students who show a greater
competence in the use of technical tools are invited to join as SWAT.
Once SWATs are identified, they are asked regarding the possibility of
supporting teachers and their peers in the use of specific computer
tools. It is impressive to see teachers becoming co-learners who take
advantage of this privileged status of their students to master tools
that promote their ability to redesign learning environments. ~When
students need support for developing their projects, SWATs show them
strategies to accomplish them. The majority report great pleasure and
pride in their new role as peer advisors.

\paragraph{Rationale:}

This peeragogical approach changes the prevailing educational paradigm
through collaboration between teachers and students, and among
students themselves. ~There are many possible points of friction
(which is not necessarily a bad thing). ~To have one or more SWATs in
each learning group transforms the way in which teachers and students
interact with each other and with available technologies.  This can
create challenges for teachers who may be used to a more ``banking''
style of teaching (again, not a bad thing in and of itself).  In
order for teachers who are new to peer learning to become more
comfortable with a shifting balance of power, they will need to have
access to the overall \patternname{Roadmap}.

\paragraph{Resolution:} SWATs can help the teachers and other students
with technical issues.  This goes along with another social result:
injecting some peer learning into an otherwise hierarchical setting.

\begin{framed}
\noindent 
\emph{What's Next.}  Can we help ways to engage with SWATs to help
them become even more proficient with technology, offering them some
mentoring in exchange for help with our project?
\end{framed}    


\subsection{References}

\begin{enumerate}
\item
  Howard Rheingold (2012), ``Mind Amplifier: Can Our Digital Tools Make
  Us Smarter?''
\item
  Instute of the Future (2005), ``\href{http://www.rheingold.com/cooperation/Technology_of_cooperation.pdf}{Technology of cooperation}''
\end{enumerate}

