\subsubsection{Main Actor}

Madeleine, a student who is trying to learn real analysis.

\subsubsection{Main success scenario}

\begin{enumerate}
\item
  Madeleine has been using a peer-learning website for mathematics for a
  while now. When she gets stuck, she asks for help in context, and her
  request is brought to the attention of the appropriate community
  member, who improves the pedagogic quality of the material. This help
  enables her to solve math problems very effectively.
\item
  Now, however, the system's software is being updated. Instead of being
  solely a ``Web 2.0'' system for communicating about the subject, the
  system can keep track of new concepts that Madeleine is using in the
  problems she solves and the questions she asks. It can suggest
  heuristics that have been used by other students solving similar
  problems. (It knows about these things through a combination of
  textual analysis and ``tagging'' of text by Madeleine and other users,
  e.g.~Natalie, who sometimes gives comments on problems that Madeleine
  solves.)
\item
  As the system grows and improves (through efforts of students and
  mentors), learning mathematics becomes increasingly easy. The material
  has been gone over by 100s of students and learning pathways are
  optimized. Madeleine sometimes can get a quick tutoring gig helping
  out another younger student, and make some money, but mostly she's
  thinking about what other subjects she will need to add to her
  portfolio in order to become an architect\ldots{} by the time she's
  23!
\end{enumerate}
