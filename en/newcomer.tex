\begin{quote}
Unless there is a new person to talk to, a lot of the ``education
stuff'' we do could grow stale. Many of the patterns and use cases for
peeragogy assume that there will be an audience or a new generation of
learners - hence the drive to create a \emph{guide}. Note that the
\emph{newcomer} and the \emph{wrapper} may work together to make the
project accessible. Even in the absence of actual newcomers, we're often
asked to try and look at things with a ``beginner's mind.''
\end{quote}

Newcomers can provide constructive (sometimes critical) feedback on
the way a project is organized.

\begin{quote}
\textbf{R\'egis Barondeau}: I joined this handbook project late, making
me a ``newcomer''. When I started to catch up, I rapidly faced doubts:

\begin{itemize}
\item Where do I start?
\item How can I help?
\item How will I make it, having to read more than 700 posts?
\item What tools are we using ? How do I use them?
\item Etc.
\end{itemize}

Even if this project is amazingly interesting, catching the train
while it already reached high speed can be an extreme sport.  By
taking care of newcomers, we might avoid loosing valuable contribuors
because they don't know how and where to start, and keep our own
project on the track, if we are able to explain the project to
newcomers. Some ideas about things we could do better:

\begin{itemize}
\item Allow participants to see the whole, for example by having a
  project landing page, by using dynamic mindmapping of wiki pages,
  etc.

\item Have mentors that will guide newcomers for some time. They could
  for example kickstart them by showing where there expertise may fit
  into the project.

\item Have landing pages overviews for every sub-project. This
  overview could show : who is involved, what are the main objectives,
  what tools are used (assuming sub-project teams may use some
  specific tools), etc.

\item Have tools introduction and if possible support. In this project
  we all seem to be used to online tools but this is often not the
  case.

\item Create a newcomer wiki page listing the basics.
\end{itemize}
\end{quote}

\begin{quote}
\textbf{Charlotte Pierce}: Joe Corneli's
\href{http://peeragogy.org/practice/heuristics/heartbeat/}{example}
evoked my own experience energing the Peeragogy community. Joe was
working a lot on the book, and I thought ``this is interesting hard
work, and he shouldn't have to do this alone.'' As a Peeragogy newcomer,
I was kindly welcomed and mentored by Joe, Howard, Fabrizio, and others.
I asked naive questions and was met with patient answers, guiding
questions, and resource links. Concurrently, I bootstrapped myself into
a position to contribute to the workflow by editing the live manuscript
for consistency, style, and continuity. The concrete act of editing and
fact-checking this relatively (to me) unfamiliar topic in physical
isolation rapidly raised my understanding of the field. I also returned
to the \href{http://socialmediaclassroom.com/host/peeragogy}{Social
Media Classroom} forums to follow up on early offers of editing help
from recently uninvolved particpants, resulting in the rekindled
interest of several new editors (if not an overwhelming army).
\end{quote}

\subsubsection{Additional Reading}

\begin{enumerate}
\item
  \href{http://lapessc.ime.usp.br/public/papers/13872/CHASE13\_present.pdf}{Why
  do newcomers abandon open source software projects?} (sildes by Igor
  Steinmacher and coauthors)
\end{enumerate}
