Author: Verena Roberts

Teachers have a reputation of working in
isolation, of keeping their learning to themselves and on their own
islands. They are also known for generously sharing resources with one
another. It is this latter trait that is becoming increasingly important
as the role of the educator continues to expand. As educational
technology research specialist Stephen Downes
\href{http://www.huffingtonpost.com/stephen-downes/the-role-of-the-educator\_b\_790937.html}{observes},
the expectations on teachers have grown from ``being expert in the
discipline of teaching and pedagogy\ldots{}{[}to needing to have{]}
up-to-date and relevant knowledge and experience in it. Even a teacher
of basic disciplines such as science, history or mathematics must remain
grounded, as no discipline has remained stable for very long, and all
disciplines require a deeper insight in order to be taught
effectively.'' It is no longer possible for an educator to work alone to
fulfil each of these roles: the solution is to work and learn in
collaboration with others. This is where peer-based sharing and learning
online, connected/networked learning, or peeragogy, can play an
important role in helping educators.

\subsection{Becoming a connected/networked learner}

The following steps are set out in `phases' in order to suggest possible
experiences one may encounter when becoming connected. It is
acknowledged that every learner is different and these `phases' only
serve as a guide.

\subsubsection{Phase 1: Deciding to take the plunge}

To help educators begin to connect, the
\href{http://www.google.com/url?q=https\%3A\%2F\%2Fdl.dropbox.com\%2Fu\%2F38904447\%2Fstarter-kit-final.pdf\&sa=D\&sntz=1\&usg=AFQjCNE9sNo1Lz9-zJ0KH48djXeYVoAF4A}{Connected
Educator's Starter Kit} was created during Connected Educator's Month in
August 2012. This article previews the main steps. The first step to
becoming a `connected educator-learner' involves making the commitment
to spending the time you'll need to learn how to learn and share in an
open, connected environment.

\subsubsection{Phase 2: Lurking}

We start off as lurkers. A learner can be considered a true `lurker'
after reviewing the starter kit, establishing a digital presence
(through a blog or a wiki) or signing up for Twitter and creating a
basic profile containing a photo. In this phase, lurkers will begin to
\href{http://www.google.com/url?q=http\%3A\%2F\%2Fwww.fractuslearning.com\%2F2012\%2F05\%2F25\%2Ftwitter-follow-education-technology\%2F\&sa=D\&sntz=1\&usg=AFQjCNF8grPMuRwU\_ImW9Jk3ZYrg0m9KgQ}{`follow'
other users on Twitter} and observe
\href{http://www.google.com/url?q=http\%3A\%2F\%2Fcybraryman.com\%2Fchats.html\&sa=D\&sntz=1\&usg=AFQjCNFJASZiwfvPbfOzFbHvAunpXfNC1g}{educational
Twitter `chats'}. Lurkers will also begin to seek out other resources
through
\href{http://theinnovativeeducator.blogspot.ca/2012/04/ten-best-education-blogs.html}{blogs},
\href{http://www.google.com/url?q=http\%3A\%2F\%2Fwww.edsocialmedia.com\%2F2011\%2F02\%2Fthe-advantage-of-facebook-groups-in-education\%2F\&sa=D\&sntz=1\&usg=AFQjCNEvc43Q7GqJqS-2S8GhEJ53Ye-j4Q}{Facebook},
\href{http://www.slideshare.net/cmsdsquires/edmodo-for-teachers-guide}{Edmodo}
and
\href{http://www.emergingedtech.com/2012/02/8-great-linkedin-groups-for-educators/}{LinkedIn}
groups.

\subsubsection{Phase 3: Entering the fray}

The lurker begins to develop into a connected educator-learner once he
or she makes the decision to enter into a dialogue with another user.
This could take the form of a personal blog post, participation on an
education-related
\href{http://edudemic.com/2012/08/education-blogs/?utm\_medium=twitter\&utm\_source=twitterfeed}{blog}
or
\href{http://educationalwikis.wikispaces.com/Examples+of+educational+wikis}{wiki}
or a an exchange with another Twitter user. Once this exchange takes
place, relationships may begin to form and the work towards building a
Personal Learning Network (PLN) begins.

One such site where such relationships can be built is
\href{http://www.classroom20.com/}{Classroom 2.0}, which was founded by
\href{http://www.stevehargadon.com/}{Steve Hargadon.} Through Classroom
2.0, Steve facilitates a number of free online learning opportunities
including weekly
\href{http://www.google.com/url?q=http\%3A\%2F\%2Fwww.futureofeducation.com\%2Fnotes\%2FPast\_Interviews\&sa=D\&sntz=1\&usg=AFQjCNHVYOvP-w7NTgKp2Fu2AX4YycnPQQ}{Blackboard
Collaborate} sessions, conferences, book projects and grassroots
cross-country educational-transformation tours. Classroom 2.0 also
offers a supportive Social Ning---a free, social learning space that
provides online conferences and synchronous and recorded interviews with
inspirational educators---for connected educator-learners around the
world.

\subsubsection{Phase 4: Building and shaping your PLN}

Just as not every person one meets becomes a friend, it is important to
remember that not every exchange will lead to a co-learning peeragogy
arrangement. It may be sufficient to follow another who provides useful
content without expecting any reciprocation. It is dependent on each
educator-learner to determine who to pay attention to and what learning
purpose that individual or group will serve. It is also up to the
learner-educator to demonstrate to others that he or she will actively
participate.

There are a number of
\href{http://storify.com/digiphile/how-to-build-a-personal-learning-network-on-twitte}{strategies}
one can use when shaping the PLN to learn. However, one of the best ways
educators can attract a core of \emph{peeragogues} is by sharing
actively and demonstrating active and open learning for others.

There are a number of sites where a new educator-learner can actively
and openly learn. In addition to personal blogging and wikis, other
professional development opportunities include open, online courses and
weekly synchronous online meetings through video, podcasts or other
forms of media. Examples of these opportunities are:
\href{http://connectedlearning.tv/howard-rheingold-social-media-and-peer-learning-mediated-pedagogy-peeragogy}{Connected
Learning TV},
\href{http://techtalktuesdays.global2.vic.edu.au/}{TechTalkTuesdays},
\href{http://learning2gether.pbworks.com/w/page/32206114/volunteersneeded}{VolunteersNeeded},
\href{http://simplek12.com/webinars}{SimpleK12},
\href{http://k12onlineconference.org/}{K12 Online,}
\href{http://www.learnnowbc.ca/educators/moodlemeets/default.aspx}{CEET},
and \href{http://edtechtalk.com/taxonomy/term/130}{EdTechTalk}.
Alternatively, courses are offered with
\href{https://p2pu.org/en/schools/school-of-ed-pilot/}{P2PU's} School of
Education or a wide variety of other opportunities collected by
\href{http://www.teachthought.com/}{TeachThought} and Educator's CPD
online. Peggy George, the co-faciliator of the weekly Classroom 2.0 LIVE
Sessions, created a livebinder package of free
`\href{http://www.google.com/url?q=http\%3A\%2F\%2Fwww.livebinders.com\%2Fplay\%2Fplay\_or\_edit\%3Fid\%3D429095\&sa=D\&sntz=1\&usg=AFQjCNHCIdRn64rPwske2vP7xrpWolb-jA}{PD
On Demand}' connected professional development online options for
peeragogy enthusiasts.

\subsubsection{Stage 5: Extending the digital PLN and connecting
face-to-face}

Over time, once the connected educator-learner has established a refined
PLN, these peeragogues may choose to shift their learning into physical
learning spaces. Some options available for these educator-learners
would include the new `grassRoots unconferences', which include examples
such as: \href{http://educonphilly.org/}{EduCon},
\href{http://davidwees.com/content/what-edcamp}{EdCamps},
\href{http://thatcamp.org/}{THATcamp} and
\href{http://connectedcanada.org/}{ConnectedCA}. These conferences are
free or extremely low-cost and focus on learning from and with others.
These `unconferences' are typically publicized through Twitter, Google
Apps, and Facebook. Connecting face-to-face with other peeragogues can
strengthen bonds to learning networks and help to promote their
sustainability.

\subsection{Postscript}

Sylvia Tolisano, Rodd Lucier and Zoe Branigan-Pipen co-created an
\href{http://farm9.staticflickr.com/8160/7161689001\_9b6725a4ca\_h.jpg}{infographic}
that which explores the experiences individuals may encounter in the
journey to become connected learners through another related sequence of
steps: \emph{Lurker}, \emph{Novice}, \emph{Insider}, \emph{Colleague},
\emph{Collaborator}, \emph{Friend}, and \emph{Confidant}. Check it out,
and also have a look at our
\href{http://peeragogy.org/recommended-reading/}{Recommended Readings}
for some additional resources.
