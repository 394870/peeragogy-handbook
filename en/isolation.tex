(This is in some ways related
to~\href{http://socialmediaclassroom.com/host/peeragogy/forum/anti-patterns-concerns-complaints-and-critiques\#comment-1808}{Participatory
Design vs Navel Gazing}.) ~An effort that isolates itself - e.g. through
lack of humility - will not have the occasion to draw on other
resources. ~In popular discourse, idiosyncratic or asocial behavior is
often casually referred to as
"\href{http://en.wikipedia.org/wiki/Autism}{autistic}", which may indeed
be a servicable metaphor (tho not without some caveats). ~As with the
symptoms of autism, social isolation (of various forms) may have its
roots in\emph{uncomfortably-intense}~experiences of sensation.

At the other end of the spectrum: it can of course be fun (and useful)
to run into the same people or ideas in different contexts, to make
connections in a creative way, to help others succeed. ~With a
too-narrow focus (cf. the notion of
"\href{http://paragogy.net/ParagogyConcepts\#transversality}{transversality}"),
people will bump into each other uncomfortably, or remain isolated; with
a too-wide focus, everything is chaotic in other ways
(see~\href{http://socialmediaclassroom.com/host/peeragogy/forum/anti-patterns-concerns-complaints-and-critiques\#comment-2146}{Co-Learning:
Messy with Lurkers}), motivating a narrowing of focus. From a design
point of view: we should be consious of interfaces that are ``too
loud'', and think about how that is compensated for by isolation (of
various forms).
