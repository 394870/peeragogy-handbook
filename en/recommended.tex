\subsection{Which is more fun, skateboarding or physics?}

\subsubsection{On the subject of fun and boredom}

\begin{enumerate}
\item
  Kano, J. (1995/2013).   \href{http://judoinfo.com/kano.htm}{The Contribution of Judo to Education} by Jigoro Kano.
\item
  Pale King, unfinished novel, by David Foster Wallace
\item
  On the Poverty of Student Life, by Mustapha Khayati
\end{enumerate}
\subsection{How do we know if we've won?}

\begin{enumerate}
\item
  Forming, Norming, Storming, Performing (from Bruce Tuckman)
\item
  The ``five-stage e-moderating model'' (from Gilly Salmon)
\item
  I, We, Its, It (from Ken Wilber --- for an application in modeling
  educational systems, see {[}1{]})
\item
  Assimilative, Information Processing, Communicative, Productive,
  Experiential, Adaptive (from Martin Oliver and Gráinne Conole)
\item
  Guidance \& Support, Communication \& Collaboration, Reflection \&
  Demonstration, Content \& Activities (from Gráinne Conole)
\item
  Considered in terms of ``Learning Power'' (Ruth Deakin-Crick \emph{et
  al}.)
\item
  Multiple intelligences (after Howard Gardner)
\item
  The associated ``mental state'' (after Mihaly Csíkszentmihályi; see
  picture)
\end{enumerate}
\subsection{Motivation}

\begin{enumerate}
\item
  Simon Sinek, Start With Why: How Great Leaders Inspire Everyone To
  Take Action, Penguin Books, 2011
\end{enumerate}
\subsection{Case Study: 5PH1NX}

\begin{enumerate}
\item
  Senge, Peter. ``The fifth discipline: The art and science of the
  learning organization.'' \emph{New York: Currency Doubleday} (1990).
\end{enumerate}
\subsection{Patterns}

\subsubsection{Further readings on patterns}

\begin{enumerate}
\item
  \href{http://en.wikipedia.org/wiki/The\_Timeless\_Way\_of\_Building}{The
  Timeless Way of Building}, by Christopher Alexander.
\item
  Article, ``Manifesto 1991'' by Christopher Alexander, Progressive
  Architecture, July 1991, pp. 108--112, provides a brief summary of
  Alexander's ideas in the form of a critique of mainstream
  architecture. Many of the same sorts of critical points would carry
  over to mainstream education. Some highlights are excerpted
  \href{https://plus.google.com/u/0/108598104736826154120/posts/agWYcqPhqSN}{here}.
\item
  \href{http://www.wikipatterns.com/display/wikipatterns/About}{Wikipatterns}
\item
  \href{http://www.patternlanguage.com/archive/ieee/ieeetext.htm}{The
  Origins of Pattern Theory, the Future of the Theory, And The
  Generation of a Living World}, Christopher Alexander's talk at the
  1996 ACM Conference on Object-Oriented Programs, Systems, Languages
  and Applications (OOPSLA)
\end{enumerate}
\subsubsection{Other related work}

\begin{enumerate}
\item
  \href{http://www.cluetrain.com}{Cluetrain Manifesto} (the
  \href{http://www.cluetrain.com/book/index.html}{First edition} is
  available for free)
\item
  \href{http://www.kk.org/newrules/contents.php}{New Rules for the New
  Economy}\href{http://www.kk.org/newrules/contents.php}{(you can
  also}\href{http://www.kk.org/newrules/contents.php}{read the book
  online})
\item
  OpenHatch.org, ``an open source community aiming to help newcomers
  find their way into free software projects.''
\end{enumerate}
\subsubsection{On Newcomers}

\begin{enumerate}
\item
  \href{http://lapessc.ime.usp.br/public/papers/13872/CHASE13\_present.pdf}{Why
  do newcomers abandon open source software projects?} (sildes by Igor
  Steinmacher and coauthors)
\end{enumerate}
\subsection{Antipatterns}

\begin{enumerate}
\item
  \href{http://en.wikipedia.org/wiki/Linguistic\_relativity}{Sapir-Whorf
  Hypothesis}
\item
  Bourdieu's notion of
  ``\href{http://en.wikipedia.org/wiki/Symbolic\_violence}{symbolic
  violence}''.
\item
  \href{http://www.livingneighborhoods.org/ht-0/fifteen.htm}{These
  fifteen properties are the glue which binds wholeness together} -
  Christopher Alexander's more recent work, going beyond the idea of a
  pattern language.
\end{enumerate}
\subsection{Convening a Group}

\begin{enumerate}
\item
  Engeström, Y. (1999). Innovative learning in work teams: Analyzing
  cycles of knowledge creation in practice. In Y. Engeström, R.
  Miettinen \& R.-L-. Punamäki (Eds.), \emph{Perspectives on activity
  theory}, (pp. 377-404). Cambridge, UK: Cambridge University Press
\item
  Gersick, C. (1988). Time and transition in work teams: Toward a new
  model of group development. \emph{Academy of Management Journal} 31
  (Oct.): 9-41.
\item
  Mimi Ito's observations about
  \href{http://mitpress.mit.edu/books/full\_pdfs/hanging\_out.pdf}{manga
  fan groups co-learning Japanese}
\item
  Rheingold U,
  \href{http://socialmediaclassroom.com/host/mindamp5/lockedwiki/main-page}{MindAmp
  groups}
\item
  Shneiderman, B. (2007).
  \href{http://doi.acm.org/10.1145/1323688.1323689}{Creativity support
  tools: accelerating discovery and innovation}. \emph{Commun. ACM} 50,
  12 (December 2007), 20-32. 
\item
  David de Ugarte, Phyles.
  (\href{http://david.lasindias.com/phyles/}{Summary})
  (\href{http://deugarte.com/gomi/phyles.pdf}{Book})
\item
  Scheidel, T. M., \& Crowell, L. (1964). Idea development in small
  discussion groups. \emph{Quarterly Journal of Speech}, 50, 140-145.
\item
  Scheidel, T. M., \& Crowell, L. (1979), \emph{Discussing and Deciding
  - A Desk Book for Group Leaders and Members}, Macmillan Publishing
\item
  Ozturk and Simsek, ``Of Conflict in Virtual Learning Communiities in
  the Context of a Democratic Pedagogy: A paradox or sophism?,'' in
  \emph{Proceedings of the Networked Learning Conference, 2012,
  Maastricht.}
   \href{http://www.google.com/search?client=chrome-mobile\&sourceid=chrome-mobile\&ie=UTF-8\&q=Of+Conflict+in+Virtual+Learning+Communiities+in+the+Context+of+a+Democratic+Pedagogy}{Video}
  or \href{http://networkedlearningconference.org.uk/abstracts/pdf/ozturk.pdf}{text}.
\item
  Paragogy Handbook,
  \href{http://peeragogy.org/organizing-a-learning-context/connectivism-in-practice-how-to-organize-a-mooc/}{How
  to Organize a MOOC}
\item
  Cathy Davidson et al.,
  \href{http://news.rapgenius.com/Cathy-davidson-how-a-class-becomes-a-community-theory-method-examples-chapter-one-lyrics}{How
  a Class Becomes a Community}
\end{enumerate}
\subsection{K-12 Peeragogy}

\subsubsection{amazing technology tools for your classroom:}

\begin{itemize}
\item
  \href{http://www.freetech4teachers.com/}{Richard Byrne}
\item
  \href{http://langwitches.org/blog/}{Sylvia Tolisano}
\item
  \href{http://catlintucker.com/2011/11/12-tech-tools-that-will-transform-your-classroom/}{Caitlin
  Tucker}
\item
  \href{http://coolcatteacher.blogspot.ca/}{Vicki Davis}
\end{itemize}
\subsubsection{How to develop your PLN:}

\begin{itemize}
\item
  \href{\%20http://thecleversheep.blogspot.ca/2012/06/seven-degrees-of-connectedness\_06.html}{Degrees
  of Connected Teaching} by Rodd Lucier
\item
  \href{\%20http://thecleversheep.blogspot.ca/2012/06/seven-degrees-of-connectedness\_06.html}{TeachThought}
\end{itemize}
\subsubsection{Theory \& philosophy of connnected learning for classroom
transformation:}

\begin{itemize}
\item
  \href{http://pairadimes.davidtruss.com/}{David Truss}
\item
  \href{http://www.downes.ca/presentation/264}{Steven Downes}
\item
  \href{http://willrichardson.com/}{Will Richardson}
\end{itemize}
\subsection{Adding Structure with Activities}

\begin{enumerate}
\item
  \href{http://dschool.stanford.edu/wp-content/uploads/2011/03/BootcampBootleg2010v2SLIM.pdf}{The
  d.school Bootcamp Bootleg} (CC-By-NC-SA) includes lots of fun
  activities to try. Can you crack the code and define new ones that are
  equally cool?
\item
  Puzio, R. S. (2005). ``On free math and copyright bottlenecks.''
  \emph{Free Culture and the Digital Library Symposium Proceedings}.
\end{enumerate}
\subsection{Connectivism in Practice --- How to Organize a MOOC (Massive
Open Online Class)}

\begin{enumerate}
\item
  Downes \& Siemens \href{http://change.mooc.ca}{MOOC site}
\item
  \href{http://halfanhour.blogspot.com/2007/02/what-connectivism-is.html}{What
  Connectivism Is} by Stephen Downes
\item
  \href{http://www.downes.ca/post/33034}{An Introduction to Connective
  Knowledge} by Stephen Downes
\item
  \href{http://www.downes.ca/presentation/290}{Facilitating a Massive
  Open Online Course}, by Stephen Downes
\item
  \href{http://grsshopper.downes.ca/index.html}{gRSShopper}
\item
  \href{\%20http://www.elearnspace.org/Articles/connectivism.htm}{Connectivism:
  A Learning Theory for the Digital Age} by George Siemens
\item
  \href{http://en.wikiversity.org/wiki/Connectivism\_glossary}{A
  Connectivism Glossary}
\item
  \href{http://www.connectivism.ca/?p=329}{Rhizomes and Networks} by
  George Siemens
\item
  \href{http://innovateonline.info/pdf/vol4\_issue5/Rhizomatic\_Education-\_\_Community\_as\_Curriculum.pdf}{Rhizomatic
  Education: Community as Curriculum} by Dave Cormier
\item
  \href{http://www.amazon.ca/Knowing-Knowledge-George-Siemens/dp/1430302305}{Knowing
  Knowledge}, a book by George Siemens
\item
  \href{http://www.amazon.com/Net-Smart-ebook/dp/B007D5UP9G}{Net Smart,}
  Howard Rheingold (about internal and external literacies for coping
  with the `always on' digital era)
\item
  \href{http://www.learningsolutionsmag.com/articles/886/}{Massive Open
  Online Courses}: Setting Up (StartToMOOC, Part 1)
\item
  \href{https://sites.google.com/site/themoocguide/}{The MOOC guide}
\end{enumerate}
\subsection{Co-Facilitation}

\begin{enumerate}
\item
  \href{http://www.scribd.com/doc/54544925/51/TRAINING-TOPIC-Co-facilitation-skills}{Peer
  Education: Training of Trainers Manual}; UN Interagency Group on Young
  Peoples Health
\item
  \href{http://www.breakoutofthebox.com/Co-FacilitatingPfeifferJones.pdf}{Co
  Facilitating}: Advantages \& Potential Disadvantages. J. Willam
  Pfeifer and John E Johnes
\item
  \href{http://reviewing.co.uk/archives/art/13\_1\_what\_do\_facilitators\_do.htm\#8\_WAYS\_OF\_FACILITATING\_ACTIVE\_LEARNING}{Summary}
  of John Heron's model of the role of facilitators
\item
  \href{http://www.infed.org/thinkers/et-rogers.htm}{Carl
  Rogers, Core Conditions and Education}, Encyclopedia of Informal
  Education
\item
  \href{http://www.studygs.net/peermed.htm}{Peer Mediation}, Study
  Guides and Strategies
\item
  \href{http://sk.cupe.ca/updir/cofacilitation-handouts.doc}{Co-Facilitation:
  The Advantages and Challenges}, Canadian Union of Public Employees
\item
  \href{http://community.bistudio.com/wiki/Bohemia\_Interactive\_Community:Guidelines}{Bohemia
  Interactive Community Wiki Guidelines}
\item
  Barrett-Lennard, G. T. (1998)
  \emph{\href{http://openlibrary.org/works/OL2014352W/Carl\_Rogers'\_Helping\_System}{Carl
  Roger's Helping System. Journey and Substance}}, London: Sage
\item
  \href{http://en.wikipedia.org/w/index.php?title=Wikipedia:Five\_pillars\&oldid=501472166}{5
  Pillars of Wikipedia}, from Wikipedia
\item
  \emph{\href{http://www.africom.mil/WO-NCO/DownloadCenter/\%5C40Publications/Training\%20the\%20Force\%20Manual.pdf}{Training
  the Force}} (2002) US Army Field Manual \#FM 7-0 (FM 25-100)
\item
  \href{http://dmlcentral.net/blog/howard-rheingold/learning-reimagined-participatory-peer-global-online}{Learning
  Reimagined: Participatory, Peer, Global, Online}, by Howard Rheingold
\item
  \href{http://www.researchgate.net/}{Research Gate} is a network
  dedicated to science and research, in which members connect,
  collaborate and discover scientific publications, jobs and
  conferences.
\item
  \href{http://ctb.ku.edu/en/tablecontents/section\_1180.aspx}{Creating
  and Facilitating Peer Support Groups}, by The Community Tool Box
\item
  \href{http://www1.villanova.edu/content/villanova/artsci/vcle/resources/toolkit/\_jcr\_content/pagecontent/download\_8/file.res/FacilitationTips.doc}{Facilitation
  Tips}, by Villanova University
\item
  \href{http://pippabuchanan.com/2011/09/04/herding-passionate-cats-the-role-of-facilitator-in-a-peer-learning-process/}{Herding
  Passionate Cats: The Role of Facilitator in a Peer Learning}, by Pippa
  Buchanan
\item
  \href{http://webpages.sou.edu/~vidmar/SOARS2008/vidmar.ppt}{Reflective
  Peer Facilitation: Crafting Collaborative Self-Assessment}, by Dale
  Vidmar, Southern Oregon University Library
\item
  \href{http://www.umass.edu/ewc/ea/Facilitation\%20Skills/important\%20tips.doc}{Effective
  Co-Facilitation}, by Everywoman's Center, University of Massachussetts
\item
  ``\href{www.ncsu.edu/park\_scholarships/pdf/chris\_argyris\_learning.pdf?}{Teaching
  smart people how to learn}'' by Chris Argyris, Harvard Business Review
  69.3, 1991; also published in expanded form as a
  \href{http://www.amazon.com/Teaching-People-Harvard-Business-Classics/dp/1422126005}{book}
  with the same name.
\end{enumerate}
\subsection{Assessment}

\begin{enumerate}
\item
  Morgan, C. and M. O'Reilly. (1999).
  \href{http://www.amazon.com/Assessing-Distance-Learners-Flexible-Learning/dp/0749428783/ref=tmm\_pap\_title\_0?ie=UTF8\&qid=1388199564\&sr=1-1}{Assessing
  Open and distance learners.} London: Kogan Page Limited.
\item
  Schmidt, J. P., Geith, C., Håklev, S. and J. Thierstein. (2009).
  \href{http://www.irrodl.org/index.php/irrodl/article/view/641/1389}{Peer-To-Peer
  Recognition of Learning in Open Education}. \emph{International Review
  of Research in Open and Distance Learning}. Volume 10, Number 5.
\item
  L.S. Vygotsky:
  \href{http://books.google.com/books?id=RxjjUefze\_oC\&printsec=frontcover\&source=gbs\_atb\#v=onepage\&q\&f=false}{Mind
  in Society: Development of Higher Psychological Processes}
\item
  \href{http://org.sagepub.com/search?author1=Reijo+Miettinen\&sortspec=date\&submit=Submit}{Reijo
  Miettinen} and
  \href{http://org.sagepub.com/search?author1=Jaakko+Virkkunen\&sortspec=date\&submit=Submit}{Jaakko
  Virkkunen},
  \href{http://org.sagepub.com/content/12/3/437.abstract}{Epistemic
  Objects, Artifacts and Organizational Change}, \emph{Organization,}
  May 2005, 12: 437-456.
\end{enumerate}
\subsection{Technologies, Services, and Platforms}

\begin{enumerate}
\item
  Irene Greif and Sunil Sarin (1987): Data Sharing in Group Work, ACM
  Transactions on Office Information Systems, vol. 5, no. 2, April 1987,
  pp. 187-211.
\item
  Irene Greif (ed.) (1988): Computer-Supported Cooperative Work: A Book
  of Readings, San Mateo, CA: Morgan Kaufman.
\item
  Irene Greif (1988): Remarks in panel discussion on ``CSCW: What does
  it mean?'', CSCW `88. Proceedings of the Conference on
  Computer-Supported Cooperative Work, September 26-28, 1988, Portland,
  Oregon, ACM, New York, NY.
\item
  Kamnersgaard, 1988
\item
  Vessey \& Galletta, 1991
\item
  Norman, 2001, 2003
\item
  DeSanctis \& Pool, 2004
\end{enumerate}

\subsubsection{Real-Time Meetings}

\begin{enumerate}
\item Howard Rheingold's webconferencing \href{http://delicious.com/hrheingold/webconferencing}{bookmarks}
\end{enumerate}

\subsubsection{Additional Tips from an open source perspective}

Care of User:Neophyte on the Teaching Open Source wiki.

\begin{enumerate}
\item
  The Art of Community
\item
  Open Advice
\item
  The Open Source Way
\end{enumerate}
\subsection{Forums}

\begin{enumerate}
\item
  Rheingold, H. \href{http://blip.tv/file/1123048}{Why use forums?}
  \emph{Social Media Classroom}.
\item
  Rheingold, H. (1998).
  \href{http://www.rheingold.com/texts/artonlinehost.html}{The Art of
  Hosting Good Conversations Online}.
\item
  Gallagher, E. J. (2006).
  \href{http://www.lehigh.edu/~indiscus/doc\_guidelines.html}{Guidelines
  for Discussion Board Writing}. Lehigh University.
\item
  Gallagher, E.J.
  (2009).\href{http://www.academiccommons.org/2009/01/shaping-a-culture-of-conversation-the-discussion-board-and-beyond/}{Shaping
  a culture of conversation. The discussion board and beyond}. The
  Academic Commons.
\item
  Academic Technology Center. (2010).
  \href{http://www.wpi.edu/Academics/ATC/Collaboratory/Idea/boards.html}{Improving
  the Use of Discussion Boards}. Worcester Polytechnic Institute.
\end{enumerate}
\subsection{Still more recommended reading}

\subsubsection{On Paragogy}

\begin{enumerate}
\item
  Corneli, J. (2010). \href{http://metameso.org/~joe/docs/paragogy-lesson.pdf}{Implementing Paragogy}. Lesson plan.
\item
  Corneli, J. and C. Danoff. (2010/2013). \emph{Paragogy}. paragogy.net
\end{enumerate}
\subsubsection{On Learning vs Training}

\begin{enumerate}
\item
  Hart, J. (April 20th, 2012). \href{http://www.c4lpt.co.uk/blog/2012/04/20/is-it-time-for-a-byol-bring-your-own-learning-strategy-in-your-organization-byol/}{Is it time for a BYOL (Bring Your Own
  Learning) strategy for your organization?} \emph{Learning in the Social
  Space. Jane Hart's Blog.}
\end{enumerate}
\subsubsection{On PLNs}

\begin{enumerate}
\item
  Rheingold, H. (2010).
  \href{http://dmlcentral.net/blog/howard-rheingold/shelly-terrell-global-netweaver-curator-pln-builder}{Shelly
  Terrell: Global Netweaver, Curator, PLN Builder.} \emph{DML
  Central}.
\item
  Richardson, W. and R. Mancabelli. (2011).
  \href{http://www.amazon.com/Personal-Learning-Networks-Connections-Transform/dp/193554327X}{Personal
  Learning Networks: Using the Power of Connection to Transform
  Education}. Bloomington, IN: Solution Tree Press.
\item
  Howard Rheingold's PLN links on Delicious
\end{enumerate}
\subsubsection{A word list for your inner edu-geek}

You can read about all of these things on Wikipedia.

\begin{enumerate}
\item
  \href{http://en.wikipedia.org/wiki/Constructivism\_(philosophy\_of\_education)}{Constructivism}
\item
  \href{http://en.wikipedia.org/wiki/Social\_constructivism}{Social
  constructivism}
\item
  \href{http://www.english.iup.edu/mmwimson/Syllabi/803/721/Radical\%20Constructivism\%20\%20\%20721.htm}{Radical
  constructivism}
\item
  \href{http://en.wikipedia.org/wiki/Enactivism\_(psychology)}{Enactivism}
\item
  \href{http://en.wikipedia.org/wiki/Constructionism\_(learning\_theory)}{Constructionism}
\item
  \href{http://en.wikipedia.org/wiki/Connectivism}{Connectivism}
\end{enumerate}
