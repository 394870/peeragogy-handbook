In the introduction to ``Organizing a Learning Context'', we remarked
that a ``learning space'' is~\emph{only potentially}~less structured
than a ``course''. ~For example, a library tends to be highly
structured, with quiet rooms for reading, protocols for checking out
books, a cataloging and shelving system that allows people to find what
they are looking for, as well as rules that deter vandalism and theft.
(Digital libraries don't need to play by all the same rules, but are
still structured.)

But more structure does not always lead to better learning. In a 2010
Forbes article titled, ``The Classroom in 2020,'' George Kembel
describes a future in which ``Tidy lectures will be supplanted by messy
real-world challenges.'' The Stanford School of Design, (or ``d.school''
-- which Kemble co-founded and currently directs) is already well-known
for its open collaborative spaces, abundant supply of Post-It notes and
markers, and improvisational brainstorm activities -- almost the
opposite of traditional lecture-based learning.

One ``unexpected benefit'' of dealing with real-world challenges is that
we can change our approach as we go. ~This is how it works in peer
learning: peers can decide on different structures not just once (say,
at the beginning of a course), but throughout the duration of their time
together. This way, they are never ``stuck'' with existing structures,
whether they be messy or clean. At least\ldots{} that's the ideal.

In practice, ``bottlenecks'' frequently arise. ~For example, in a
digital library context, there may be bottlenecks having to do with
software development, organizational resources, community good will, or
access to funding -- and probably all of the above. ~In a didactic
context, it may be as simple as one person knowing something that others
do not.

While we can't eliminate scarcity in one stroke, we can design
activities for peer learning that are ``scarcity aware'' and that help
us move in the direction of adaptive learning structures.

\subsection{Planning Peer Learning
Activities}\label{planning-peer-learning-activities}

We begin with two simple questions:

\begin{itemize}
\itemsep1pt\parskip0pt\parsep0pt
\item
  How do we select an appropriate learning activity?
\item
  How do we go about creating a learning activity if we don't find an
  existing one?
\end{itemize}

``Planning a learning activity'' should mean planning
an~\emph{effective}~learning activity, and in particular that means
something that people can and will engage with. ~In short, an
appropriate learning activity may be one that you already do! ~At the
very least, current activities can provide a ``seed'' for even more
effective ones.

But when entering unfamiliar territory, it can be difficult to know
where to begin.~ And remember the bottlenecks mentioned above?~ When you
run into difficulty, ask yourself:
\href{http://peeragogy.org/patterns-and-heuristics/}{why is this hard?}~
You might try adapting
\href{http://learnpythonthehardway.org/book/intro.html\#comment-409972596}{Zed
Shaw's task-management trick}, and make a list of limiting factors,
obstacles, etc., then cross off those which you can find a strategy to
deal with (add an annotation as to why). ~For example, you might decide
to overcome your lack of knowledge in some area by hiring a tutor or
expert consultant, or by putting in the hours learning things the hard
way (Zed would particularly approve of this choice).~ If you can't find
a strategy to deal with some issue, presumably you can table it, at
least for a while.

Strategic thinking like this works well for one person. What about when
you're planning activities for someone else? ~Here you have to be
careful: remember, this is peer learning, not traditional ``teaching''
or ``curriculum design''. ~The first rule of thumb for \emph{peer
learning} is: don't plan activities for others unless you plan to to
take part as a fully engaged participant.~ Otherwise, you might be more
interested in the literature on \emph{collaborative learning}, which has
often been deployed to good effect within a standard pedagogical context
(see e.g.~Bruffee {{[}1{]}}).~ In a peer learning setting, everyone will
have something to say about~ ``what do you need to do'' and ``why is it
hard,'' and everyone is likely to be interested in everyone else's
answer as well as their own.

Furthermore, different participants will be doing different things, and
these will be ``hard'' for different reasons. Part of \emph{your} job is
to try to make sure that not only are all of the relevant roles covered,
but that the participants involved are getting enough support.

\subsection{One scenario: building activities for the Peeragogy
Handbook}\label{one-scenario-building-activities-for-the-peeragogy-handbook}

Adding a bunch of activities to the handbook won't solve all of our
usability issues, but more activities would help.~ We can think about
each article or section from this perspective:

\begin{enumerate}
\def\labelenumi{\arabic{enumi}.}
\item
  When looking at this piece of text, what type of knowledge are we (and
  the reader) trying to gain? ~ Technical skills, or abstract skills?~
  What's the point?
\item
  What's difficult here? ~What might be difficult for someone else?
\item
  What learning activity recipes or models might be appropriate? (See
  e.g.~{{[}2{]}}, {{[}3{]}}.)
\item
  What customizations do we need for this particular application?
\end{enumerate}

\textbf{\emph{~}As a quick example: designing a learning activity for
the current page}

\begin{enumerate}
\def\labelenumi{\arabic{enumi}.}
\item
  \emph{We want to be able to come up with effective learning activities
  to accompany a ``how to'' article for peer learners}
\item
  \emph{It might be difficult to ``unplug'' from all the reading and
  writing that we're habituated to doing.}
\item
  \emph{But there are lots and lots of ways to learn.}
\item
  \emph{Therefore, the proposed handbook activity is to simply step away
  from the handbook for a while.}
\item
  \textbf{Look for some examples of peer learning in everyday life.~
  When you've gained an insight about peer learning from your own
  experience, come back and create a related activity to accompany
  another handbook page!}
\end{enumerate}

\subsection{References}\label{references}

\begin{enumerate}
\def\labelenumi{\arabic{enumi}.}
\item
  Bruffee, Kenneth A. (1984). ``Collaborative learning and the
  conversation of mankind.'' \emph{College English} 46.7, 635-652
\item
  \href{http://www.kstoolkit.org/KS+Methods}{KS ToolKit} from
  kstoolkit.org.
\item
  \href{http://serc.carleton.edu/NAGTWorkshops/coursedesign/tutorial/strategies.html}{Designing
  Effective and Innovative Sources}~(particularly the section on
  ``Teaching Strategies for Actively Engaging Students in the
  Classroom'')
\end{enumerate}
