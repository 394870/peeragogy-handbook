\noindent\textbf{Welcome to the Peeragogy Handbook!}

\bigskip

This book presents a range of techniques that self-motivated learners
can use to connect with each other and develop stronger communities and
collaborations. The book is addressed to everyone who is interested in
how learning works, whether you're an educator, a hobbyist, an artist, a
home-school student, an employee, a parent, an activist, an archivist, a
mathematician, or a tennis player. The book was written by a bunch of
people who think learning is cool.

Over the course of working on the book, we practiced peeragogy --
another word for ``peer learning'' -- and we learned a lot.

Our experience within this project has been that flattened hierarchies
do not necessarily mean decisions go by consensus -- people often take
the ball and run with it. The handbook includes co-edited pages as well
as single-author works: often the lines and voices are blurred. One
constant throughout the book is our interest in making something useful.
To this end, the book is available under non-restrictive
\href{http://peeragogy.org/resources/license/}{legal terms}, which allow
you to reuse portions of it however you see fit it. Among other things,
we include instructions on
\href{http://peeragogy.org/resources/how-to-get-involved/}{how to join
us in further developing this resource}.

\bigskip

Sincerely, \href{http://peeragogy.org/resources/meet-the-team/}{The
Peeragogy Team}
