\textbf{Welcome to the Peeragogy Handbook!}

Peeragogy is a collection of techniques for collaborative learning and
collaborative work. By learning how to ``work smart'' together, we hope
to leave the world in a better state than it was when we arrived.

Indeed, humans have always learned from each other. But for a long time
-- until the advent of the Web and widespread access to digital media --
schools have had an effective monopoly on the business of learning. Now,
with access to open educational resources and free or inexpensive
communication platforms, groups of people can learn together outside as
well as inside formal institutions. All of this prompted us to
reconsider the meaning of ``peer learning.''

The \emph{Peeragogy Handbook} isn't a normal book; it is an example of
the kind of work that's only just now possible. The book is an evolving
guide, and it tells a collaboratively written story. In fact, it's one
that \emph{you} can help write. Using this book, you will develop new
norms for the groups you work with --- whether online, offline, or both.
Every section includes exercises and research methods that you can apply
to build and sustain strong and exciting collaborations. When you read
the book, you will get to know the authors and will see how we have
applied these ideas: in classrooms, in research, in business, and more.

You'll meet Julian, who put the ideas to work as one of the directors of
a housing association; Roland, a professional journalist and
change-maker; Charlie, a language teacher and writer who works with
experimental media for fun and profit; and Charlotte, an indie publisher
who wants to become better at what she does by helping others learn how
to do it well too --- as well as many other contributors from around the
world.

The book focuses on techniques for
\href{http://peeragogy.org/convene/}{building a strong group},
\href{http://peeragogy.org/organize/}{organizing a learning space},
\href{http://peeragogy.org/cowork/}{doing cooperative work}, and
\href{http://peeragogy.org/assessment/}{conducting effective peer
assessment}. These major sections are complemented by a catalog of
design patterns and notes on relevant technologies. You should, for this
reason, think of the book as first and foremost a practical guide. As
you work through these chapters, you will begin mastering these
techniques and developing the way you think about getting things done.

The following brief section is a guide to using the book itself. Then,
we provide a succinct overview of the book's contents which will help
you begin thinking like a peeragogue. Once again, future editions
\href{http://peeragogy.org/resources/how-to-get-involved/}{can include
your voice}, so don't hesitate to
\href{http://peeragogy.org/contact/}{get in touch} with comments or
questions.

%\textbf{\href{hhttp://youtu.be/\_UMAZbBOOCA}{LIVE SESSION}} 10/7 1PM
%GMT-5

