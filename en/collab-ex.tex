\attrib{Peter Taylor \& Teryl Cartwright}
%
\section*{Part I (Peter).}
Collaborative Exploration invites participants
to shape their own directions of inquiry and develop their skills as
investigators and teachers (in the broadest sense of the word). The
basic mode of a Collaborative Exploration centers on interactions over a
delimited period of time in small groups. Engagement takes place either
online, for instance via Google+, or face-to-face. The aim is to create
an experience of re-engagement with oneself as an avid learner and
inquirer. This section combines practical information about how to run
Collaborative Explorations as well as ideas and questions about how to
make sense of what happens in them. A companion entry conveys one
participant's experience with several Collaborative Explorations
(hereafter, ``CE'').

\subsection{Overview and contrast to cMOOCs}

The tangible goal of any CE is to develop contributions to the topic
defined by the ``case'', which is written by the host or originator of
the CE in advance, and which is intended to be broad and
thought-provoking (some examples are given below). We aim for a parallel
experiential goal, which is that we hope participants will be impressed
at how much can be learned with a small commitment of time using this
structure. The standard model for an online CE is to have four sessions
spaced one week apart, in which the same small group interacts in real
time via the internet, for an hour per session. Participants are asked
to spend at least 90 minutes between sessions on self-directed inquiry
into the case, and to share their inquiries-in-progress with their small
group and a wider community. Reflection typically involves shifts in
participants' definition of what they want to find out and how. Any
participants wondering how to define a meaningful and useful line of
inquiry are encouraged to review the scenario for the CE, any associated
materials, posts from other participants, and think about what they
would like to learn more about or dig deeper into. Everyone is left, in
the end, to judge for themselves whether what interests them is
meaningful and useful. During the live sessions, participants can expect
to do a lot of listening, starting off in the first session with
autobiographical stories that make it easier to trust and take risks
with whoever has joined that CE, and a lot of writing to gather their
thoughts, sometimes privately, sometimes shared. There is no assumption
that participants will pursue the case beyond the limited duration of
the CE. This said, the tools and processes that the CE employs for
purposes of inquiry, dialogue, reflection, and collaboration are
designed to be readily learned by participants, and to translate well
into other settings -- for instance, where they can be used to support
the inquiries of others. In short, online CEs are moderate-sized open
online collaborative learning. It remains to be seen whether the CE
``movement'' will attract enough participants to scale up to multiple
learning communities around any given scenario, each hosted by a
different person and running independently. A MOOC (massive open online
course) seeks to get masses of people registered, knowing that a tiny
fraction will complete it, while CE best practices focus on establishing
effective learning in small online communities, and then potentially
scale up from there by multiplying out. CEs aim to address the needs of
online learners who want to:

\begin{itemize}
\item
  dig deeper, make ``thicker'' connections with other learners
\item
  connect topics with their own interests
\item
  participate for short periods of time
\item
  learn without needing credits or badges
\end{itemize}

Currently, even the most high-profile MOOCs do not appear to be
conducive to deep or thick inquiry. For example, while link-sharing is
typical in ``connectivist'' or ``cMOOCs'', annotation and discussion of
the contents is less common. By contrast, CEs are structured to elicit
participants' thoughtful reflections and syntheses. The use of the
internet for CEs, in contrast, is guided by two principles of online
education (Taylor 2007).

\begin{itemize}
\itemsep1pt\parskip0pt\parsep0pt
\item
  Use computers first and foremost to teach or learn things that are
  difficult to teach or learn with pedagogical approaches that are not
  based on computers
\item
  Model computer use, at least initially, on known best practices for
  teaching/learning without computers.
\end{itemize}

Thus, CEs bring in participants from a distance, make rapid connections
with informants or discussants outside the course, and contribute to
evolving guides to materials and resources. At the same time,
participants benefit from the support of instructors/facilitators and
peers who they can trust, and integrate what they learn with their own
personal, pedagogical, and professional development.

\subsection{Example scenarios or ``cases''}

\subsubsection{Connectivist MOOCs: Learning and collaboration,
possibilities and limitations}

The core faculty member of a graduate program at a public urban
university wants help as they decide how to contribute to efforts at the
university program to promote open digital education. It is clear that
the emphasis will not be on xMOOCs, i.e., those designed for
transmission of established knowledge, but on cMOOCs. In other words,
the plan is to emphasize connectivist learning and community development
emerges around, but may extend well beyond, the materials provided by
the MOOC hosts (Morrison 2013; Taylor 2013). What is not yet clear about
is just how learning works in cMOOCs. What are the possibilities and
limitations of this educational strategy? How do they bear on themes
like creativity, community, collaboration, and openness? The program is
especially interested in anticipating any undesirable
consequences\ldots{}

\subsubsection{Science and policy that would improve responses to
extreme climatic events}

Recent and historical climate-related events shed light on the social
impact of emergency plans, investment in and maintenance of
infrastructure, as well as investment in reconstruction. Policy makers,
from the local level up, can learn from the experiences of others and
prepare for future crises. The question for this case is how to get
political authorities and political groups---which might be anywhere
from the town level to the international, from the elected to the
voluntary---interested in learning about how best to respond to extreme
climatic events. Changes might take place at the level of policy,
budget, organization, and so on. It should even be possible to engage
people who do not buy into the idea of human-induced climate
change---after all, whatever the cause, extreme climatic events have to
be dealt with\ldots{}.

\subsection{The structure}

Independent of the topic, we've found the following common structure
useful for our online CEs. \emph{Before the first live session}:
Participants review the scenario, the expectations and mechanics, join a
special-purpose Google+ community and get set up technically for the
hangouts.

\smallskip
\noindent\textbf{Session 1}: \emph{Participants getting to know each
other}. After freewriting to clarify thoughts and hopes, followed by a
quick check-in, participants take 5 minutes each to tell the story of
how they came to be a person who would be interested to participate in a
Collaborative Exploration on the scenario. Other participants note
connections with the speaker and possible ways to extend their
interests, sharing these using an online form.

\smallskip
\noindent \emph{Between-session
work}: Spend at least 90 minutes on inquiries related to the case,
posting about this to google+ community for the CE, and reviewing the
posts of others.

\smallskip
\noindent \textbf{Session 2}: \emph{Clarify thinking and
inquiries}. Freewriting on one's thoughts about the case, followed by a
check-in, then turn-taking ``dialogue process'' to clarify what
participants are thinking about their inquiries into the case. Session
finishes with gathering and sharing thoughts using an online form.

\smallskip
\noindent \emph{Between-session work}: Spend at least 90 minutes (a) on inquiries
related to the case and (b) preparing a work-in-progress presentation.

\smallskip
\noindent \textbf{Session 3}: \emph{Work-in-progress presentations}. 5 minutes for
each participant, with ``plus-delta'' feedback given by everyone on each
presentation.

\smallskip
\noindent\emph{Between-session work}: Digest the feedback on one's
presentation and revise it into a self-standing product (i.e., one
understandable without spoken narration). 

\smallskip
\noindent\textbf{Session 4}:
\emph{Taking Stock}. Use same format as for session 2 to explore
participants' thinking about (a) how the Collaborative Exploration
contributed to the topic (the tangible goal) and to the experiential
goal, as well as (b) how to extend what has emerged during the CE.

\smallskip
\noindent\emph{After session 4 (optional)}: Participants share on a public
Google+ community not only the products they have prepared, but also
reflections on the Collaborative Exploration process.

\subsection{How to make sense of what happens in CEs}

\mbox{(Re-)}engagement with oneself as an avid learner and inquirer in CEs is
made possible by the combination of:

\begin{itemize}
\itemsep1pt\parskip0pt\parsep0pt
\item
  Processes and tools used for inquiry, dialogue, reflection, and
  collaboration;
\item
  Connections made among the diverse participants who bring to bear
  diverse interests, skills, knowledge, experience, and aspirations; and
\item
  Contributions from the participants to the topics laid out in scenarios.
\end{itemize}

\noindent The hope is that through experiencing a \mbox{(re-)}engagement with learning,
participants will subsequently transfer experience with this triad
into their own inquiries and teaching-learning interactions, the ways
that they support inquiries of others; other practices of critical
intellectual exchange and cooperation; and that they will be more
prepared to challenge the barriers to learning that are often
associated with expertise, location, time, gender, race, class, or
age.

\subsection{Acknowledgements}

The comments of Jeremy Szteiter and the contributions of the
participants of the 2013 Collaborative Explorations have helped in the
preparation of this article.

\section*{Part II (Teryl).}

As a May graduate of the Master's program in
Critical and Creative Thinking (CCT) at UMass Boston, I owe my gratitude
to Professors Peter Taylor and Jeremy Szteiter for inviting me to
informally continue my education less than a month later. It is a
tribute to them that I would then take four consecutive CEs without
stopping. They can best share how to run a CE, but as a ``student,'' it
is how to creatively take a CE that inspires what I'd like to share.

\paragraph{June 2013 CE: Scaffolding Creative Learning}

I was grateful participants took the time to post links and ideas to
support my inquiries, yet something else intrigued me about the
potential of Collaborative Exploration. Luanne Witkowski, an artist and
one of the CCT instructors, took our ideas and made a diagram
incorporating our scaffolding concepts together; she changed her own
original drawing to include all of ours. I wanted to pay forward and
back my learning too, so I combined the ideas of all the participants,
adapted and taught a lesson outside the CE and then shared the results.
From this jumping into someone else's scaffolding, I went into even more
experimental learning in the next CE.

\paragraph{July 2013 CE: Design in Critical Thinking}

In~a second CE, I took the title literally and developed a design IN
critical thinking. To try out my triangle tangent thinking model, during
a lesson on leadership in church, I suddenly stopped teaching a
classroom of older professional adults halfway in and asked them to
participate in ``design as you go'' curriculum---by taking over the
class. Since I wanted to be fair, along with my lesson outline I had
already given them a supposed ``icebreaker'' activity~that they could
teach from, although they also had the option of my continued teaching.
Results? My triangle drawing works as a lesson plan; the class took the
tangent, but surprisingly, I wasn't just relegated to moderator, it
became a true co-facilitation,a model of change at the midpoint for both
the individual and community in the choices and direction.

\paragraph{September 2013 CE: Everyone Can Think Creatively}

This CE had to be commended for its participants humoring my project and
allowing the exploration of testing a CE itself. Was it possible to be a
Creative Failure in a Creativity CE? To evaluate ``Creative Failure in a
Creativity CE,'' I used a simple test. If creative success (unknowingly
given by my CE community) was a product both ``novel AND useful,'' any
post without a comment was a failure (``not useful'') to my readers. Any
post that a reader commented was similar to something else already done
was ``useful,'' but not novel. Failure had me posting again. Did I
mention what nice people these were when they didn't know what I was
doing? It would have been easy for them to ignore my continued posting,
yet the community of a CE cannot be praised enough. They were supportive
of me and finding academic colleagues who have a sense of humor is
mercifully not novel, but extremely useful in this experience.

\paragraph{October 2013 CE: Stories to Scaffold Creative Learning}

In this CE I gave myself the challenge of indirect teaching. Could I be
a story ``shower'', not teller? I took concepts important to me about
teaching with story, yet also tried to leave space for others'
interpretations. Ironically, in some ways creative failure
continued---again I was not as helpful as I had wished. This CE also had
a twist---no hero stories allowed, so my creative and personal stories
had to be ambiguous or use other connecting structures based on the
participants' preferences. It was interesting which stories worked
best---fiction worked more with humor, real experience worked if I
shared about someone other than myself and other kinds worked with
visuals.Collaborative Explorations provide a safe space for the joint
learning and teaching to occur. The diversity blends well into a
community that is curious, courageous and creative. Although I have my
M.A. as the first completely online CCT student, I found almost a
face-to-face learning ``feel'' in their deeply connected CE community as
well. It does require time, openness and commitment to each other during
the intense focus together on a topic. Yet seeing where the
participant-directed `design as you go' curriculum ends up is worth
investing in and sharing with others. After all, there are many other
ways still out there to try out CEs.

\paragraph{Postscript}

I also ran a CE~for the Susquehanna Conference of the UMC for 10 days,
working with a group of~professionals exploring a call into ordained
ministry.~Going in cold, I had to work harder to do community
building~without the Google hangout meetings and recommend~their
inclusion to increase~the comfort level and participation of the group
members.

\subsection{Resources}

\noindent Further examples of CE scenarios can be viewed at
{\center
\url{http://cct.wikispaces.com/CEt}.\par}

\bigskip
\noindent Recommended readings below convey some of the sources for the CE
processes.~ Ideas about possible extensions of CEs can be viewed in the
full prospectus at
{\center
\url{http://cct.wikispaces.com/CEp}.\par}

\subsection{References}

\begin{enumerate}
\itemsep1pt\parskip0pt\parsep0pt
\item
  Morrison, D. (2013). ``\href{http://bit.ly/164uqkJ}{A tale of two
  MOOCs @ Coursera: Divided by pedagogy}''.
\item
  Taylor, P. J. (2007) ``Guidelines for ensuring that educational
  technologies are used only when there is significant pedagogical
  benefit,'' International Journal of Arts and Sciences, 2 (1): 26-29,
  2007 (adapted from
  \href{http://bit.ly/etguide}{http://bit.ly/etguide}).
\item
  Taylor, P. J. (2013). ``\href{http://wp.me/p1gwfa-vv}{Supporting
  change in creative learning}''.
\end{enumerate}

\subsection{Recommended Reading}

\begin{enumerate}
\itemsep1pt\parskip0pt\parsep0pt
\item
  Paley, V. G. (1997). The Girl with the Brown Crayon. Cambridge, MA,
  Harvard University Press.
\item
  Paley, V.G. (2010). The Boy on the Beach: Building Community by Play.
  Chicago, University of Chicago Press.
\item
  Taylor, P. J. and J. Szteiter (2012). Taking Yourself Seriously:
  Processes of Research and Engagement Arlington, MA, The Pumping
  Station.
\item
  White, M. (2011). Narrative Practice: Continuing the Conversation. New
  York, Norton.
\end{enumerate}

