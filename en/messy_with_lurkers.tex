\begin{quote}
\textbf{Gigi Johnson}\emph{: (1) Co-learning is Messy. It needs time,
patience, confusion, re-forming, re-norming, re-storming, etc. Things go
awry and part of norms needs to be how to realign. (2) Co-learning is a
VERY different experience from traditional teacher-led learning in terms
of time and completion. It is frustrating, so many people will lurk or
just step in and out, the latter of which is very different from what is
acceptable in traditonal learning. Online learning programs are painted
with the brush now of an ``unacceptable'' 50\% average non-completion
rate. Stanford's MOOC AI class, which started out with +100,000 people,
had 12\% finish.~ If only 12\% or 50\% of my traditional class finished,
I'd have a hard time getting next quarter's classes approved!}
\end{quote}

The second point is similar to the earlier Anti-pattern
``\href{http://socialmediaclassroom.com/host/peeragogy/forum/anti-patterns-concerns-complaints-and-critiques\#comment-1854}{Misunderstanding
Power (Laws)}``. ~People have to join in order to try, and when joining
is low-cost, and completion low-benefit, it is not surprising that many
people will''dissipate" as the course progresses. The ``messiness'' of
co-learning is interesting because it points to a sort of ``internal
dissipation'', as contributors bring their multiple different
backgrounds, interests, and communication styles to bear.
~In~\href{http://www.altchi.org/submissions/submission\_wmt\_0.pdf}{Tomlinson
et al.}, we observed:

\begin{quote}
\emph{More authors means more content, but also more words thrown away.
Many of the words written by~authors were deleted during the ongoing
editing~process. The sheer mass of deleted words might raise~the
question of whether authoring a paper in such a~massively distributed
fashion is efficient.}
\end{quote}

If we were to describe this situation in traditional subject/object
terms, we would say that peer production has a ``low signal to noise
ratio''. ~However, it may be more appropriate (and constructive) to
think of meanings as co-constructed as the process runs, and of
messiness (or meaninglessness) as symptomatic, not of peer
production~\emph{itself}, but of deficiencies or infelicities in shared
meaning-making and ``integrating'' features.
