\paragraph{Definition:}

\begin{quote}
\textbf{Gigi Johnson}: (1) Co-learning is Messy. It needs time, patience,
confusion, re-forming, re-norming, re-storming, etc. Things go awry and
part of norms needs to be how to realign. (2) Co-learning is a VERY
different experience from traditional teacher-led learning in terms of
time and completion. It is frustrating, so many people will lurk or just
step in and out, the latter of which is very different from what is
acceptable in traditonal learning. Online learning programs are painted
with the brush now of an ``unacceptable'' 50\% average non-completion
rate. Stanford's MOOC AI class, which started out with +100,000 people,
had 12\% finish. If only 12\% or 50\% of my traditional class finished,
I'd have a hard time getting next quarter's classes approved!
\end{quote}

\paragraph{Problem:}

\begin{quote}
\textbf{Tomlinson \emph{et al.}}: More authors means more content, but
also more words thrown away. Many of the words written by authors were
deleted during the ongoing editing process. The sheer mass of deleted
words might raise the question of whether authoring a paper in such a
massively distributed fashion is efficient.
\end{quote}

\paragraph{Solution:} People have to join in order to try, and when joining
is low-cost, and completion low-benefit, it is not surprising that many
people will ``dissipate'' as the course progresses. The ``messiness'' of
co-learning is interesting because it points to a sort of ``internal
dissipation'', as contributors bring their multiple different
backgrounds, interests, and communication styles to bear.

\paragraph{Challenges:} If we were to describe this situation in the
traditional subject/object, sender/receiver terms of information theory,
we would say that peer production has a ``low signal to noise ratio'',
and we would tend to think of it as a highly inefficient process.
However, it may be more appropriate (and constructive) to think of
meanings as co-constructed as the process runs, and of messiness (or
meaninglessness) as symptomatic, not of peer production itself, but of
deficiencies or infelicities in shared meaning-making and
``integrating'' features.

\paragraph{What's Next:} What comes out of thinking about the anti-pattern
is that we need to be careful about how we think about ``virtues'' in a
peer production setting. It is not just a question of being a ``good
contributor'' to an existing project, but of continually improving the
methods that this project uses to make meaning.

\paragraph{References:}

\begin{enumerate}
\item
  Tomlinson, B., Ross, J., André, P., Baumer, E.P.S., Patterson, D.J.,
  Corneli, J., Mahaux, M., Nobarany, S., Lazzari, M., Penzenstadler, B.,
  Torrance, A.W., Callele, D.J., Olson, G.M., Silberman, M.S., Ständer,
  M., Palamedi, F.R., Salah, A., Morrill, E., Franch, X., Mueller, F.,
  Kaye, J., Black, R.W., Cohn, M.L., Shih, P.C., Brewer, J., Goyal, N.,
  Näkki, P., Huang, J., Baghaei, N., and Saper,
  C., \href{http://altchi.org/submissions/submission_wmt_0.pdf}{Massively
  Distributed Authorship of Academic Papers}, \emph{Proceedings of
  Alt.Chi}, Austin Texas, May 5--10 2012 (10 page extended abstract),
  ACM, 2012,
\item
  Yochai Benkler, and Helen Nissenbaum (2006). ``Commons-based Peer
  Production and Virtue.'' \emph{Journal of Political Philosophy} 14.4 :
  394-419.
\item
  Paul Kockelman (2010). ``Enemies, parasites, and noise: How to take up
  residence in a system without becoming a term in it'' \emph{Journal of
  Linguistic Anthropology} 20.2: 406-421
\end{enumerate}
