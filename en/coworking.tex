\attrib{Joseph Corneli}
%
\begin{quote}
 Interpersonal exchange and
collaboration to develop and pursue common goals goes further than
``learning'' or ``working'' in their mainstream definitions. ~This
article will look at examples drawn from Linux, Wikipedia, and my own
work on PlanetMath, with a few surprises along the way, leading us to
new ways of thinking about how to do co-design when build systems for
peer learning and peer production.
\end{quote}

\subsection{Co-working as the flip side of convening}

\begin{quote}
\textbf{Linus Torvalds}: The first mistake is thinking that you can
throw things out there and ask people to help. That's not how it works.
You make it public, and then you assume that you'll have to do all the
work, and ask people to come up with suggestions of what you should do,
not what they should do. Maybe they'll start helping eventually, but you
should start off with the assumption that you're going to be the one
maintaining it and ready to do all the work. The other thing--and it's
kind of related--that people seem to get wrong is to think that the code
they write is what matters. No, even if you wrote 100\% of the code, and
even if you are the best programmer in the world and will never need any
help with the project at all, the thing that really matters is the users
of the code. The code itself is unimportant; the project is only as
useful as people actually find it.
\end{quote}

In fact, we can think of contributors as a special class of ``user''
with a real time investment in the way the project works. We typically
cannot ``Tom Sawyer'' ourselves into leisure or ease just because we
manage to work collaboratively, or just because we have found people
with some common interests.~ And yet, in the right setting, many people
do want to contribute! For example, on ``Wikipedia, the encyclopedia
anyone can edit'' (as of
2011)~\href{http://\%20http://www.readwriteweb.com/archives/wikipedias_goal_1_billion_monthly_visitors_by_2015.php}{as
many as}~80,000 visitors make 5 or more edits per month. This is interesting to compare with the
\href{http://www.aaronsw.com/weblog/whowriteswikipedia}{empirical fact}
that (as of 2006) ``over 50\% of all the edits are done by just~.7\%~of
the users\ldots{} \emph{24 people}\ldots{} and in fact the most active 2\%,
which is 1400 people, have done 73.4\% of all the edits.''~ Similar
numbers apply to other peer production communities.

\subsection{A little theory}

In many natural systems, things are not distributed equally, and it is
not atypical for e.g.~20\% of the population to control 80\% of the
wealth (or, as we saw, for 2\% of the users to do nearly 80\% of the
edits). Many, many systems work like this, so maybe there's a good
reason for it.
Let's think about it in terms of ``coordination'' as understood by the
late Elinor Ostrom. She talked about ``local solutions for local
problems''. By definition, such geographically-based coordination
requires close proximity. What does ``close'' mean? If we think about
homogeneous space, it just means that we draw a circle (or sphere)
around where we are, and the radius of this circle (resp. sphere) is
small.

An interesting
\href{http://en.wikipedia.org/wiki/N-sphere\#Volume_and_surface_area}{mathematical
fact} is that as the dimension grows, the volume of the sphere gets
``thinner'', so the radius must increase to capture the same
\emph{d}-dimensional volume when \emph{d} grows! ~In other words, the
more different factors impact on a given issue, the less likely there
are to be small scale, self-contained, ``local problems'' or ``local
solutions'' in the first place.

As a network or service provider grows~ (like a
\href{http://peeragogy.org/organize/connectivism-in-practice-how-to-organize-a-mooc/}{MOOC}
as opposed to a
\href{http://peeragogy.org/case-study-collaborative-explorations/}{Collaborative
Exploration}, for example), they typically build many weak ties, with a
few strong ties that hold it all together.~ Google is happy to serve
everyone's web requests -- but they can't have just anyone walking in
off the street and connecting devices their network in Mountain View.

By the way, the 2006 article about Wikipedia quoted above was written
by Aaron Swartz (``over 50\% of all the edits are done by\ldots{} 24
people'', etc.), who achieved considerable
\href{http://www.wired.com/threatlevel/2011/07/swartz-arrest/}{notoriety}
for downloading lots and lots of academic papers with a device plugged
into MIT's network.  His suicide while under federal prosecution for
this activity caused considerable shock, grief, and dismay among online
activists.  One thing we could potentially take away from the experience is that
there is a tremendous difference between a solo effort and the
distributed peer-to-peer infrastructures like the ones that underly
the PirateBay, which, despite raids, fines, jail sentences,
nation-wide bans, and server downtime, has proved decidedly hard to
extinguish.  According to a recent press release: ``If they cut off
one head, two more shall take its place.''

\subsection{Co-working: what is an institution?}

As idealists, we would love to be able to create systems that are both
powerful and humane.~ Some may reflect with a type of sentimental
fondness on completely mythical economic systems in which a ``dedicated
individual could rise to the top through dint of effort.''~
But well-articulated systems like this \emph{do} exist: natural
languages, for example,~are so expressive and adaptive that most
sentences have never been said before.~ A well-articulated system lends
itself to ``local solutions to local problems'' -- but in the
linguistics case, this is only because all words are not created equal.

\begin{quote}
\textbf{Dr Seuss}: My brothers read a little bit. Little words like `If'
and `It.' My father can read big words, too, Like CONSTANTINOPLE and
TIMBUKTU.
\end{quote}

We could go on here to talk about Coase's theory of the firm, and
Benkler's theory of
``\href{http://www.yale.edu/yalelj/112/BenklerWEB.pdf}{Coase's
Penguin}''. We might continue
\href{http://www.aaronsw.com/weblog/perfectinstitutions}{quoting} from
Aaron Swartz. But we will not get so deeply into that here: you can
explore it on your own!~ For now, it is enough to say that an
institution is a bit like a language.~ This will help us a lot in the
next section.

\subsection{Designing a platform for peer learning\emph{}}

\begin{quote}
\href{planetmath.org}{PlanetMath} \emph{is a virtual community which
aims to help make mathematical knowledge more accessible.}
\end{quote}

In my PhD thesis {[}1{]}, I talk about my work to turn this
long-running website, which since 2001 had focused on building a
mathematics encyclopedia, into a peer produced peer learning
environment.  We wanted to retain
all of the old activities related to authoring, reviewing, and
discussing encyclopedia articles, but we would also add a bunch of new
features having to do with mathmatical problem solving, an activity
that is suitable for mathematical beginners.

My first translation of this idea into a basic interaction design
was as follows.~ People can continue to add articles to PlanetMath's
encyclopedia: they can connect one article to another
(A$\rightarrow$A) either by making one article the ``parent'' of
another, or, more typically, via an inline link. Like in the
old system, users can discuss any object (X$\rightarrow$T), but now
there is more structure: \emph{problems} can be connected to articles
(A$\rightarrow$P) and \emph{solutions} can be connected to problems
(P$\rightarrow$S).~ Instead of explicitly modeling ``goals,'' I
decided that problems and articles could be organized into
``collections,'' the same way that videos are organized into playlists
on YouTube, and that the user would get encouraging directed feedback as they
work their way through the problems in a given collection.~ I
described a few other types of objects and interactions, like questions and answers, groups, and
the ability to change the ``type'' of certain contributed objects.~

The next step was to do a complete overhaul of PlanetMath's software
system, to build something that could actually~\emph{do} all of that.~
After deploying the realized system and doing some studies with PlanetMath
users, I realized the design summarized above was not complete.~ Note
that this is very much along the lines of what Linus Torvalds said
above: I did the design, and me and a small group of collaborators
with their own vested interests built the system, then we put it out
there to get more ideas from users.

The main thing that was missing from the earlier design was the idea
of a \emph{project}.~ From interviewing users, it became clear to me
that it would be helpful to think of every object as being part of at
least one project: everything should have someone looking after it!~~
Importantly, getting back to the very beginning of this article, each
project can define its own purpose for existing.~ Here's how I put it
in my thesis:

\begin{quote}
\emph{Actions and artifacts are embedded within projects, which can be
modeled in terms of informal user experience and formal system features.
Project updates can be modeled with a language of fundamental actions.
Projects themselves model their outcomes, and are made ``viable'' by
features that connect to the motivations and ambitions of potential
participants.}
\end{quote}

The key point is that the evolving design describes a sort ``grammar''
for the kinds of things that can be done on PlanetMath.~ In the
updated design, projects are something like paragraphs that combine simple
sentences.~ The language can be extended further, and I hope that will
happen in further study.~ In particular, we need to understand more
about how the ``sub-language'' of project updates works (compare
the~\href{http://peeragogy.org/practice/roadmap/}{Roadmap} pattern described in this handbook).

\subsection{The discussion continues: Reliving the history of
mathematics as a peeragogical game?}

These notes have shown one approach to the design of spaces for
learning and knowledge building. Although the article has focused on
mathematics learning, similar reflections would apply to designing
other sorts of spaces for learning or working, for instance, to the
continued development of the Peeragogy project itself!  Perhaps it can contribute
to the development of a new kind of institution.

\begin{quote}
\textbf{Doug Breitbart}: It occurred to me that you could add a learning
dimension to the site that sets up the history of math as a series of
problems, proofs and theorems that, although already solved, could be
re-cast as if not yet solved, and framed as current challenges which
visitors could take on (clearly with links to the actual solutions, and
deconstruction of how they were arrived at, when the visitor decides to
throw in the towel).
\end{quote}

\subsection{References}

\begin{enumerate}
\itemsep1pt\parskip0pt\parsep0pt
\item
  Corneli, J. (2014).~
  \href{http://metameso.org/~joe/thesis-outline.html}{Peer Produced Peer
  Learning: A Mathematics Case Study}.~ Ph. D. thesis.~~ The
  Open University.
\end{enumerate}
