Early on, active peeragogue Charlie Danoff
\href{http://socialmediaclassroom.com/host/peeragogy/wiki/rolesdivision-labor}{suggested}
that someone take on the ``wrapper role'' -- do a weekly pre/post wrap,
so that new users would know the status the project is at any given
point in time. The project's
\href{http://socialmediaclassroom.com/host/peeragogy/}{wiki page} also
serves as another ``wrapper''. We check the public summaries of the
project from time to time to make sure that they accurately represent
the facts on the ground. Note that in its various forms, the ``wrapper
role'' plays an important integrative function. According to the theory
proposed by Yochai Benkler, for free/open ``commons-based'' projects to
work itis vital to have both (1) the ability to contribute small pieces;
(2) something that stitches those pieces together {[}1{]}. In the first
year of the Peeragogy project, the ``Weekly Roundup'' by Christopher
Tillman Neal served to engage and re-engage members. Peeragogues began
to eager watched for the weekly reports to see if our teams or our names
had been mentioned. When there was a holiday or break, Chris would
announce the hiatus, to keep the flow going. In the second year of the
project, we didn't routinely publish summaries of progress, and instead,
we've assumed that interested parties will stay tuned on Google+.
Nevertheless, we maintain internal and external summaries, ranging from
meeting agendas to press releases to quick-start guides.

\subsection{Reference}

\begin{enumerate}
\item
  Benkler, Y. (2002). Coase's Penguin, or Linux and the Nature of the
  Firm, \emph{Yale Law Journal} 112, pp. 369-446.
\end{enumerate}
