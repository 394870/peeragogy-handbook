Knowing how to make good use of ``weak ties'' is often seen as a
strength.

\begin{quote}
\textbf{Nancy Darling}: {[}S{]}trong and weak ties tend to serve
different functions in our lives. When we need a big favor or social or
instrumental support, we ask our friends. We call them when we need to
move a washing machine. But if we need information that we don't have,
the people to ask are our weak ties. They have more diverse knowledge
and more diverse ties than our close friends do. We ask them when we
want to know who to hire to install our washing machine. {[}1{]}
\end{quote}
The quote suggests that there is a certain trade-off between use of weak
ties and use of strong ties. The \emph{anti-pattern} in question then is
less to do with whether we are forming weak ties or strong ties, and
more to do with whether we are being honest with ourselves and with each
other about the nature of the ties we are forming -- and their potential
uses. We can be ``peers'' in either a weak or a strong sense. The
question to ask is whether our needs match our expectations! In the
peeragogy context, this has to do with how we interact.

\begin{quote}
\textbf{One of us}: I am learning about peeragogy, but I think I'm
failing to be a good peeragogue. I remember that Howard once told us
that the most important thing is that you should be responsible not only
for your own learning but for your peers' learning. {[}\ldots{}{]} So
the question is, are we learning from others by ourselves or are we
helping others to learn?
\end{quote}
If we are ``only'' co-consumers of information then this seems like a
classic example of a weak tie. We are part of the same ``audience''. On
the other hand, if we are actively engaging with other people, then this
is a foundation for strong ties. In this case of deep learning, our aims
are neither instrumental nor informational, but ``interactional''.
People who do not put in the time and effort will remain stuck at the
level of ``weak ties'', and will not be able to draw on the benefits
that ``strong ties'' offer.

\subsection{Reference}

\begin{enumerate}
\item
  Nancy Darling (2010).
  \href{http://www.psychologytoday.com/blog/thinking-about-kids/201005/facebook-and-the-strength-weak-ties}{Facebook
  and the Strength of Weak Ties}, Psychology Today.
\end{enumerate}
