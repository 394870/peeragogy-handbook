\textbf{Main actors}

The non-executive (Jim, Pamela, Julian) and executive (Clare, Malcolm,
Colin \& Jenny) directors of a housing association (a not-for-profit
organisation building and letting ``social'' housing for families in
housing need) \textbf{Main success scenario}

\begin{enumerate}
\item
  The board of the housing association need to set a strategy that takes
  account of significant changes in legislation, the UK {[}welfare{]}
  benefits system and the availability of long term construction loans.
\item
  Julian, eager to make use of his new-found peeragogical insights
  suggests an approach where individuals research specific factors and
  the team work together to draw out themes and strategic options. As a
  start he proposes that each board member researches an area of
  specific knowledge or interest.
\item
  Jim, the Chairman, identifies questions he wants to ask the Chairs of
  other Housing Associations. Pamela (a lawyer) agrees to do an analysis
  of the relevant legislation. Clare, the CEO, plans out a series of
  meetings with the local councils in the boroughs of interest to
  understand their reactions to the changes from central government.
  Jenny, the operations director, starts modelling the impact on
  occupancy from new benefits rules. Colin, the development director,
  re-purposes existing work on options for development sites to reflect
  different housing mixes on each site. Malcolm, the finance director,
  prepares a briefing on the new treasury landscape and the changing
  positions of major lenders.
\item
  Each member of the board documents their research in a private wiki.
  Julian facilitates some synchronous and asynchronous discussion to
  draw out themes in each area and map across the areas of interest.
  Malcolm, the FD, adapts his financial models to take differet options
  as parameters.
\item
  Clare refines the themes into a set of strategic options for the
  association, with associated financial modelling provided by Malcolm.
\item
  Individual board members explore the options asynchronously before
  convening for an all-day meeting to confirm the strategy.
\end{enumerate}
\textbf{Thoughts}

\begin{enumerate}
\item
  This may be a little close to the ``peer production'' end of
  peeragogy, but on the other hand, where (if anywhere) do we draw the
  line?
\item
  This probably needs to be made a little more abstract to be a useful
  use case, and in doing so I suspect will start to overlap with
  \href{http://socialmediaclassroom.com/host/peeragogy/forum/patterns-and-use-cases\#comment-1509}{Pæragogy
  helps solve complex problems}
\item
  It looks to me as if there may be some candidate patterns buried in
  this use case, e.g. Environment Scanning, Codifying Specialist
  Knowledge, Extracting Themes, Modelling Outcomes, Consensus Building
\end{enumerate}
