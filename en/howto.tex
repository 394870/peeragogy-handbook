\emph{Author}: Howard Rheingold

This document is a practical guide to online co-learning, a living
document that invites comment and invites readers to join the community
of editors; the document does not have to be read in linear order from
beginning to end.

If you and a group of other people want to use digital media and
networks to co-learn together, this handbook is a practical tool for
learning how to self-organize peer learning -- what we call
``peeragogy.'' Material about conceptualizing and convening co-learning
-- the stuff about getting started -- is located toward the top of the
table of contents. Material about assessment, resources, use cases is
located toward the bottom of the TOC. But you don't have to read it in
sequential order. Hop around if you'd like. We think -- and some
research seems to support -- that understanding how co-learning works
will help you do co-learning more effectively. So we've included
material about learning theories that support peer learning or that
reveal useful characteristics of successful peer learning. For those who
want to delve more deeply into the empirical research and scholarship,
we've linked to a sister document -- a literature review of learning
theory related to peeragogy. For those who want to study more deeply
about the aspects of peer learning we summarize in our articles, we
provide a list of links to related handbook articles, and a set of
resources for further study. Think of our pages as both places to start
and as jumping off points.

The short videos, most of them under one minute long, at the very
beginning of many articles are meant to convey a sense of what the
article and its supporting material is meant to convey.

This is a living document. If you want to join our community of editors,
contact howard@rheingold.com (If you want to see how we go about
creating a handbook entry, see our guide for newcomers.) If you don't
want to go as far as joining the community of editors, please feel free
to use the comment thread attached to each page to suggest changes
and/or additions.

\subsection{See also}

\begin{itemize}
\item
  \href{http://peeragogy.org/how-to-get-involved/}{Guide to getting
  started}
\item
  \href{http://peeragogy.org/table-of-contents/}{The Table of Contents}
\item
  \href{http://peeragogy.org/resources/}{Our list of resources}
\item
  \href{http://peeragogy.org/resources/literature-review-peeragogy/}{Our
  literature review}
\end{itemize}
