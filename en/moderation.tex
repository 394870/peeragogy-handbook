\begin{quote}
``Why is a fishbowl more productive than debate? The small group
conversations in the fishbowl tend to de-personalize the issue and
reduce the stress level, making people's statements more cogent. Since
people are talking with their fellow partisans, they get less caught up
in wasteful adversarial games.'' - the
\href{http://www.co-intelligence.org/y2k\_fishbowl.html}{Co-Intelligence
Institute}
\end{quote}
\href{http://peeragogy.org/organizing-a-learning-context/participation/}{Participation}
in online forums tends to follow a
"\href{http://en.wikipedia.org/wiki/Power\_law}{power law}," with
unequal engagement. One remedy for this is simply for the most active
participants to step back, and moderate how much they speak (see
\href{http://peeragogy.org/patterns-usecases/patterns-and-heuristics/carrying-capacity/}{Carrying
Capacity}). OWS uses a similar technique in their
"\href{http://en.wikipedia.org/wiki/Progressive\_stack}{progressive
stack}."
