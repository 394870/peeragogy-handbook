\begin{quote}
\textbf{The Co-Intelligence Institute}: Why is a fishbowl more
productive than debate? The small group conversations in the fishbowl
tend to de-personalize the issue and reduce the stress level, making
people's statements more cogent. Since people are talking with their
fellow partisans, they get less caught up in wasteful adversarial games.
\end{quote}
\href{http://peeragogy.org/organizing-a-learning-context/participation/}{Participation}
in online forums tends to follow a ``power law,'' with vastly unequal
engagement. If you want to counteract this tendency, one possibility
would simply be for the most active participants to step back, and
moderate how much they speak. This is related the the
\href{http://peeragogy.org/patterns-usecases/patterns-and-heuristics/carrying-capacity/}{Carrying
Capacity} pattern and the
\href{http://peeragogy.org/practice/antipatterns/misunderstanding-power/}{Misunderstanding
Power} anti-pattern: check those out before you proceed. Occupy Wall
Street used a related technique that they called the
``\href{http://en.wikipedia.org/wiki/Progressive\_stack}{progressive
stack}.'' There are lots of other strategies to try.
