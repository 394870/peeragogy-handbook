Authors: \href{http://peeragogy.org/resources/meet-the-team/}{The
Peeragogy Team}

\subsection{What is a \emph{pattern}?}

A \emph{pattern} is anything that happens over and over again. In the
context of peeragogy, we mean repeating things that we \emph{like}, or
that we think are useful for some purpose. Thing that happen a lot but
are not desirable are called \emph{anti-patterns}!

\subsection{What is a \emph{use case}?}

A \emph{use case} describes someone (or something) who uses a given
system or tool to achieve a goal. When writing a use case, it is
presented with a title (which serves as a brief summary), a main actor,
and a success scenario. Additional features can be added, such as
alternate interaction paths leading to a variation on the result.

\subsection{What do you get when you put these together?}

Combine patterns and use cases and you start to get something called a
\emph{pattern language}. See the section on
"\href{http://peeragogy.org/patterns-usecases/patterns-and-heuristics/}{Patterns
and Heuristics}" for one such representation. That page draws on the
relationships between the patterns we've found for organizing peer
learning, and some known problem-solving techniques. See the page
``\href{http://peeragogy.org/researching-peeragogy/}{Researching
peeragogy}'' for additional related discussion.

\subsection{Patterns of peeragogy}

Here is our index of the patterns we've found so far (described in more
detail after the jump):

\begin{itemize}
\item
  \href{http://peeragogy.org/patterns/heartbeat/}{Heartbeat}
\item
  \href{http://peeragogy.org/patterns/carrying-capacity/}{Carrying
  Capacity}
\item
  \href{http://peeragogy.org/patterns/creating-a-guide/}{Creating a
  Guide}
\item
  \href{http://peeragogy.org/patterns/discerning-a-pattern/}{Discerning
  a Pattern}
\item
  \href{http://peeragogy.org/patterns/moderation/}{Moderation}
\item
  \href{http://peeragogy.org/patterns/newcomer/}{Newcomer}
\item
  \href{http://peeragogy.org/patterns/pattern-language/}{Pattern
  Language}
\item
  \href{http://peeragogy.org/patterns/polling-for-ideas/}{Polling for
  Ideas}
\item
  \href{http://peeragogy.org/patterns/praxis-vs-poeisis/}{Praxis vs
  Poeisis}
\item
  \href{http://peeragogy.org/patterns/roadmap/}{Roadmap}
\item
  \href{http://peeragogy.org/patterns/roles/}{Roles}
\item
  \href{http://peeragogy.org/patterns/wrapper/}{Wrapper}
\end{itemize}

\subsection{Use cases for Peeragogy}

We also present a variety of hypothetical and not-so-hypothetical use
cases:

\begin{itemize}
\item
  \href{http://peeragogy.org/accounting/}{Accounting}
\item
  \href{http://peeragogy.org/use-cases/cest-la-vie/}{C'est la vie}
\item
  \href{http://peeragogy.org/use-cases/distributed-project-management/}{Distributed
  Project Management}
\item
  \href{http://peeragogy.org/use-cases/improved-adaptivity/}{Improved
  adaptivity}
\item
  \href{http://peeragogy.org/use-cases/improving-the-efficacy-of-research-funding/}{Improving
  the efficacy of research funding}
\item
  \href{http://peeragogy.org/use-cases/journalist-enters-the-whispering-gallery/}{Journalist
  enters the Whispering Gallery}
\item
  \href{http://peeragogy.org/judo/}{Judo}
\item
  \href{http://peeragogy.org/use-cases/living-the-oer-dream/}{Living the
  OER dream}
\item
  \href{http://peeragogy.org/use-cases/making-our-own-tools/}{Making our
  own tools}
\item
  \href{http://peeragogy.org/use-cases/peer-learning-on-the-technical-edge/}{Peer
  Learning on the Technical Edge}
\item
  \href{http://peeragogy.org/use-cases/from-peer-production-to-peer-learning/}{Peer
  production to peer learning}
\item
  \href{http://peeragogy.org/use-cases/prolegomena-to-any-future-math-learning-environment/}{Prolegomena
  to Any Future Math Learning Environment}
\item
  \href{http://peeragogy.org/use-cases/paeragogy-helps-solve-complex-problems/}{Pæragogy
  helps solve complex problems}
\item
  \href{http://peeragogy.org/use-cases/starting-a-company/}{Starting a
  Company}
\item
  \href{http://peeragogy.org/use-cases/steal-this-book/}{Steal This
  Book}
\item
  \href{http://peeragogy.org/use-cases/strategy-as-learning/}{Strategy
  as learning}
\item
  \href{http://peeragogy.org/use-cases/we-are-the-1-percent/}{We are the
  1 percent}
\item
  \href{http://peeragogy.org/use-cases/young-aspiring-blogger-wants-to-avoid-starvation/}{Young
  aspiring blogger wants to avoid starvation}
\end{itemize}

\subsection{Anti-patterns for Peeragogy}

And some ``anti-patterns'' (things to avoid):

\begin{itemize}
\item
  \href{http://peeragogy.org/antipatterns/isolation/}{Isolation}
\item
  \href{http://peeragogy.org/antipatterns/magical-thinking/}{Magical
  thinking}
\item
  \href{http://peeragogy.org/antipatterns/co-learning-messy-with-lurkers/}{Messy
  with Lurkers}
\item
  \href{http://peeragogy.org/antipatterns/misunderstanding-power/}{Misunderstanding
  Power}
\item
  \href{http://peeragogy.org/antipatterns/navel-gazing/}{Navel Gazing}
\item
  \href{http://peeragogy.org/antipatterns/stasis/}{Stasis}
\item
  \href{http://peeragogy.org/antipatterns/stuck-at-the-level-of-weak-ties/}{Stuck
  at the level of weak ties}
\end{itemize}

\subsubsection{Examples}

The above use cases and patterns make the ``story'' abstract -- but how
about some concrete examples of peeragogy in action? Consider:

\begin{itemize}
\item
  \href{http://openhatch.org/}{OpenHatch.org}, ``an open source
  community aiming to help newcomers find their way into free software
  projects.''
\item
  The
  \href{http://campus.ftacademy.org/wiki/index.php/Free\_Technology\_Guild}{Free
  Technology Guild} is a younger project with aspirations similar in
  some ways to those of OpenHatch, but in this case, oriented not just
  to pairing newcomers with mentors, but pairing clients with service
  providers. ``The idea is that we as a group will do useful projects
  for our members or external parties, and on-the-job we mentor and
  learn and get better.'' (Since this is a new project, the
  \href{http://campus.ftacademy.org/community/pg/groups/8500/free-technology-guild-working-group/}{project
  building phase} is itself a nacent example of paragogy.)
\item
  Many more examples on our
  \href{http://peeragogy.org/examples/}{examples} page!
\end{itemize}

\subsubsection{Further reading}

\begin{itemize}
\item
  \href{http://en.wikipedia.org/wiki/The\_Timeless\_Way\_of\_Building}{The
  Timeless Way of Building}, by Christopher Alexander.~An elegant work,
  like most of Alexander's writing. If you want to start out with
  something smaller, there's a pithy essay by Christopher Alexander
  called \href{http://www.rudi.net/pages/8755}{A City is Not a Tree},
  available online
\item
  \href{http://dreamsongs.net/Files/PatternsOfSoftware.pdf}{Patterns of
  Software}, by Richard Gabriel, who applies the ``pattern'' idea to
  software and programming languages.
\end{itemize}

~
