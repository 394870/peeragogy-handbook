\begin{quote}
Welcome to the Peeragogy Accelerator.
\end{quote}

The purpose of the \emph{Peeragogy Accelerator} is to use the power of
peer learning to help build great organizations.

We will do this by investing time and energy, rather than money,
building a distributed community of peer learners, and a strongly vetted
collection of best practices. Our project complements others' work on
sites like
\href{https://en.wikiversity.org/wiki/Wikiversity:Main_Page}{Wikiversity}
and \href{https://www.p2pu.org/en/}{P2PU}, but with an applied flavor.
It is somewhat similar to \href{https://www.ycombinator.com/}{Y
Combinator} and other start-up accelerators or incubators, but we're
doing it the
\href{https://en.wikipedia.org/wiki/Commons-based_peer_production}{commons
based peer production} way.

Here, we present \emph{Peeragogy in Action}, a project guide in four
parts. Each part relates to one or more sections of our handbook, and
suggests activities to try while you explore peer learning. These
activities are designed for flexible use by widely distributed groups,
collaborating via a light-weight infrastructure. Participants may be
educators, community organizers, designers, hackers, dancers, students,
seasoned peeragogues, or first-timers. The guide should be useful for
groups who want to build a strong collaboration, as well as to
facilitators or theorists who want to hone their practice or approach.
Together, we will use our various talents to build effective methods and
models for peer produced peer learning. We've labeled the phases as
Stage 1 through Stage 4, because that's the schedule we use, but if
you're working through this on your own, you can choose your own pace.
Let's get started!

\section*{Stage 1. Set the initial challenge and build a framework for
accountability among participants. (1-3
weeks)}\label{stage-1.-set-the-initial-challenge-and-build-a-framework-for-accountability-among-participants.-1-3-weeks}

\emph{Activity} -- Come up with a plan for your work and an agreement,
or informal contract, for your group. You can use the suggestions in
this document as a starting point, but your first task is to revise the
outline we've developed so that it suits your needs. It might be helpful
to ask: What are you interested in learning? What is your primary
intended outcome? What problem do you hope to solve? How collaborative
does your project need to be? How will the participants' expertise in
the topic vary? What sort of support will you and other participants
require? What problems won't you solve?

\emph{Technology} -- Familiarize yourself with the collaboration tools
you intend to use (e.g.~a public wiki, a private forum, a community
table, social media, or something else). Create something in text,
image, or video form introducing yourself and your project(s) to others
in the worldwide peeragogy community.

\emph{Suggested Resources} -- The Peeragogy Handbook, parts I
(`\href{http://peeragogy.org/}{Introduction}') and II
(`\href{http://peeragogy.org/motivation/}{Motivation}'). For a succinct
theoretical overview, please refer to our literature review, which we
have adapted into a
\href{http://en.wikipedia.org/wiki/Peer_learning}{Wikipedia page about
`Peer learning'}.

\emph{Observations from the Peeragogy project} -- We had a fairly weak
project structure at the outset, which yielded mixed results. One
participant said: ``I definitely think I do better when presented with a
framework or scaffold to use for participation or content development.''
Yet the same person wrote with enthusiasm about being ``freed of the
requirement or need for an entrepreneurial visionary.''

\emph{Further Reading} -- Boud, D. and Lee, A. (2005). \emph{`Peer
learning' as pedagogic discourse for research education}. Studies in
Higher Education, 30(5):501--516.

\emph{Further Questions}: \textbf{What subject or skill does YOUR group
want to learn?} OR \textbf{What product or service does YOUR group want
to produce?}

\begin{itemize}
\itemsep1pt\parskip0pt\parsep0pt
\item
  identify members \& subgroups
\item
  survey members: interests, motivations, skills, experience, time
\item
  other
\end{itemize}

\textbf{What learning theory and practice does the group need to know to
put together a successful peer-learning program?} OR \textbf{What
specific theory and research does the group need to meet production or
service goals?}

\begin{itemize}
\itemsep1pt\parskip0pt\parsep0pt
\item
  who has gone before?
  (\href{http://peeragogy.github.io/practice.html}{\textbf{Reduce,
  Reuse, Recycle}})
\item
  similar groups \& organizations
\item
  best \& worst practices
\item
  other similar products, for production
\item
  proven success strategies
\item
  other
\end{itemize}

\section*{Stage 2. Bring in other people to support your shared goals,
and make the work more fun too. (1-2
weeks)}\label{stage-2.-bring-in-other-people-to-support-your-shared-goals-and-make-the-work-more-fun-too.-1-2-weeks}

\emph{Activity} -- Write an invitation to someone who can help as a
co-facilitator on your project. Clarify what you hope to learn from them
and what your project has to offer. Helpful questions to consider as you
think about who to invite: What resources are available or missing? What
do you already have that you can build on? How will you find the
necessary resources? Who else is interested in these kinds of
challenges? Go through the these questions again when you have a small
group, and come up with a list of more people you'd like to invite or
consult with as the project progresses.

\emph{Technology} -- Identify tools that could potentially be useful
during the project, even if it's new to you. Start learning how to use
them. Connect with people in other locales who share similar interests
or know the tools. Find related groups, communities, and forums and
engage with others to start a dialogue.

\emph{Suggested resources} -- The Peeragogy Handbook, parts IV
(`\href{http://peeragogy.org/convening-a-group/}{Convening a Group}')
and V
(`\href{http://peeragogy.org/organizing-a-learning-context/}{Organizing
a Learning Context}').

\emph{Observations from the Peeragogy project} -- We used a strategy of
``open enrollment.'' New people were welcome to join the project at any
time. We also encouraged people to either stay involved or withdraw;
several times over the first year, we required participants to
explicitly reaffirm interest in order to stay registered in the forum
and mailing list.

\emph{Further Reading} -- Schmidt, J. Philipp. (2009). Commons-Based
Peer Production and education. Free Culture Research Workshop Harvard
University, 23 October 2009.

\emph{Further Questions}: \textbf{Identify and select the best learning
resources about your topic} OR \textbf{Identify and select the best
production resources for that product or service}

\begin{itemize}
\itemsep1pt\parskip0pt\parsep0pt
\item
  published resources
\item
  live resources (people)
\item
  other
\end{itemize}

\textbf{What is the appropriate technology and communications tools and
platforms your group needs to accomplish their learning goal?} OR
\textbf{How will these participants identify and select the appropriate
technology and communications tools and platforms to accomplish their
production goal or service mission?}

\begin{itemize}
\itemsep1pt\parskip0pt\parsep0pt
\item
  internal platforms \& tools including meeting spaces, connecting
  diverse platforms
\item
  external (public-facing) platforms \& tools
\item
  other
\end{itemize}

\section*{Stage 3. Solidifying your work plan and learning strategy
together with concrete measures for `success' to move the project
forward. (1-3
weeks)}\label{stage-3.-solidifying-your-work-plan-and-learning-strategy-together-with-concrete-measures-for-success-to-move-the-project-forward.-1-3-weeks}

\emph{Activity} -- Distill your ideas by writing an essay, making visual
sketches, or creating a short video to communicate the unique plans for
organization and evaluation that your group will use. By this time, you
should have identified which aspects of the project need to be refined
or expanded. Dive in!

\emph{Technology} -- Take time to mentor others or be mentored by
someone, meeting up in person or online. Pair up with someone else and
share knowledge together about one or more tools. You can discuss some
of the difficulties that you've encountered, or teach a beginner some
tricks.

\emph{Suggested resources} -- The Peeragogy Handbook, parts VI
(`\href{http://peeragogy.org/co-facilitation/}{Cooperation}'), VII
(`\href{http://peeragogy.org/assessment/}{Assessment}'), and at least
some of part II
(`\href{http://peeragogy.org/patterns-usecases/}{Peeragogy in
Practice}').

\emph{Observations from the Peeragogy project} -- Perhaps one of the
most important roles in the Peeragogy project was the role of the
`Wrapper', who prepared and circulated weekly summaries of forum
activity. This helped people stay informed about what was happening in
the project even if they didn't have time to read the forums. We've also
found that small groups of people who arrange their own meetings are
often the most productive.

\emph{Further Reading} -- Argyris, Chris. ``Teaching smart people how to
learn.'' Harvard Business Review 69.3 (1991); and, Gersick, Connie J.G.
``Time and transition in work teams: Toward a new model of group
development.'' Academy of Management Journal 31.1 (1988): 9-41.

\emph{Further Questions}: \textbf{What are your benchmarks for success
in your learning enterprise?} OR \textbf{What are your benchmarks for
success in your production enterprise or service organization?}

\begin{itemize}
\itemsep1pt\parskip0pt\parsep0pt
\item
  survey members
\item
  pilot testing
\item
  formal assessment
\item
  consensus
\item
  other
\item
  what's next?
\end{itemize}

\section*{Stage 4. Wrap up the project with a critical assessment of
progress and directions for future work. Share any changes to this
syllabus that you think would be useful for future peeragogues! (1-2
weeks).}\label{stage-4.-wrap-up-the-project-with-a-critical-assessment-of-progress-and-directions-for-future-work.-share-any-changes-to-this-syllabus-that-you-think-would-be-useful-for-future-peeragogues-1-2-weeks.}

\emph{Activity} -- Identify the main obstacles you encountered. What are
some goals you were not able to accomplish yet? Did you foresee these
challenges at the outset? How did this project resemble or differ from
others you've worked on? How would you do things differently in future
projects? What would you like to tackle next?

\emph{Writing} -- Communicate your reflection case. Prepare a short
written or multimedia essay, dealing with your experiences in this
course. Share the results by posting it where others in the broader
Peeragogy project can find it.

\emph{Suggested resources} -- The Peeragogy Handbook, parts VIII
(`\href{http://peeragogy.org/resources/technologies/}{Technologies,
Services, and Platforms}') and IX
(`\href{http://peeragogy.org/resources/}{Resources}').

\emph{Observations from the Peeragogy project} -- When we were deciding
how to license our work,~ we decided to use CC0, emphasizing~
`re-usability' and hoping that other people would come and remix the
handbook.~ At the moment, we're still waiting to see the first remix
edition, but we're confident that it will come along in due course.~
Maybe you'll be the one who makes it!

\emph{`Extra credit'} -- Contribute back to one of the other
organisations or projects that helped you on this peeragogical journey.
Think about what you have to offer. Is it a bug fix, a constructive
critique, pictures, translation help, PR, wiki-gnoming or making a cake?
Make it something special, and people will remember you and thank you
for it.

\emph{Further reading} -- Stallman, Richard.
``\href{http://www.gnu.org/philosophy/shouldbefree.html}{Why software
should be free}'' (1992).

\emph{Further Questions}: Write your own!

\subsection{{\small {Micro-}Case Study: The Peeragogy Project, Year
1}}\label{micro-case-study-the-peeragogy-project-year-1}

Since its conception in early 2012, the Peeragogy Project has collected
over 3700 comments in our discussion forum, and over 200 pages of
expository text in the handbook. It has given contributors a new way of
thinking about things together. However, the project has not had the
levels of engagement that should be possible, given the technology
available, the global interest in improving education, and the number of
thoughful participants who expressed interest. We hope that the handbook
and this accompanying syllabus will provide a seed for a new phase of
learning, with many new contributors and new ideas drawn from real-life
applications.

We began with these four questions:

\begin{enumerate}
\def\labelenumi{\arabic{enumi}.}
\item
  \emph{How does a motivated group of self-learners choose a subject or
  skill to learn?}
\item
  \emph{How can this group identify and select the best learning
  resources about that topic?}
\item
  \emph{How will these learners identify and select the appropriate
  technology and communications tools and platforms to accomplish their
  learning goal?}
\item
  \emph{What does the group need to know about learning theory and
  practice to put together a successful peer-learning program?}
\end{enumerate}

\subsection{{\small {Micro-}Case Study: The Peeragogy Project, Year
2}}\label{micro-case-study-the-peeragogy-project-year-2}

10 new handbook contributors joined in the project's second year. We've
begun a series of weekly Hangouts on Air that have brought in many
additional discussants, all key people who can help to fulfil
peeragogy's promise.~ The handbook has been considerably improved
through edits and discussion.~ The next step for us is putting this work
into action in the \emph{Peeragogy Accelerator}.

\subsection{{\small {Micro-}Case Study: The Peeragogy Project, Year
3}}\label{micro-case-study-the-peeragogy-project-year-3}

We published our plans as ``Building the Peeragogy Accelerator'',
presenting it at OER14 and inviting feedback. In the run up to this, we
had been very active creating additional abstracts and submitting them
to conferences. However, despite our efforts we failed to recruit any
newcomers for the trial run of the Accelerator. Even so, piloting the
Accelerator with some of our own projects worked reasonably
well,\footnote{For an overview, see
  \url{http://is.gd/up_peeragogy_accelerator}.} but we decided to focus
on the handbook in the second half of the year. As the project's line-up
shifted, participants reaffirmed the importance of having ``no camp
counsellors.'' In the last quarter of 2014, we created the workbook that
is now presented in Part I, as a quickstart guide to peeragogy. We also
revised the pattern catalog, and used the revised format to create a
``distributed roadmap'' for the Peeragogy project -- featured in Chapter
7 of the third edition of the handbook.
