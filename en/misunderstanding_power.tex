\begin{quote}
\emph{Zipf's law states that given some corpus of natural language
utterances, the frequency of any word is inversely proportional to its
rank in the frequency table. Thus the most frequent word will occur
approximately twice as often as the second most frequent word, three
times as often as the third most frequent word, etc.}
\end{quote}

Zipf's law (or other formulations of the same thing) govern
the~\href{http://www2.econ.uu.nl/users/marrewijk/geography/zipf/index.htm}{size
of cities}, and related formulations
describe~\href{http://pricetags.wordpress.com/2010/10/26/kleibers-law-growth-and-creativity-in-cities/}{energy
use}:~roughly speaking, an elephant has a lower metabolism than a mouse
and is more ``energy efficient''. ~At that same link, we see the
suggestion that creativity in large-scale environments\emph{speeds up!}
\emph{The anti-pattern}: how many times have we been at a conference or
workshop and heard someone say (or said ourselves) ``wouldn't it be
great if this energy could be sustained all year 'round?'' ~Or in a
classroom or peer production setting, wondered why it is that everyone
does not participate equally. ~``Wouldn't it be great if we could
increase participation?'' ~If you believe the result above, large-scale
participation would indeed tend to increase creativity! - But
nevertheless, participation does tend to fall off according
to~\emph{some}~power law (see Introduction to Power Laws
in~\href{http://www.theuncertaintyprinciple.danoff.org/v2i3.html}{The
Uncertainty Principle, Volume II, Issue 3}), and it would be a grand
illusion to assume that everyone is coming from a similar place with
regard to the various literacies and motivations that are conducive to
participation. ~Furthermore, a ``provisionist'' attitude (``If we change
our system we will equalize participation and access'') simply will not
work in general,~\emph{since}~\emph{power laws are inherently an
epiphenomenon of networks}. Note that participation in a given activity
often (but not always) falls off over~\emph{time} as well.~ This effect
seems related but is also not well understood (many people would like to
write a hit song / best selling novel / start a religion / etc., but few
actually do).~ See the anti-pattern
"\href{http://peeragogy.org/antipatterns/magical-thinking/}{Magical
Thinking}" for more on that. \emph{About the title}: Note that those
agents who do post the most in a given collaboration (respectively, the
words or ideas that are most common in a given language) will tend to
influence the space the most.~ In this way, we can see some parallels
between the
\href{http://en.wikipedia.org/wiki/Linguistic\_relativity}{Sapir-Whorf
Hypothesis} and Bourdieu's notion of
"\href{http://en.wikipedia.org/wiki/Symbolic\_violence}{symbolic
violence}``.~ Much as Paul Graham wrote about programming languages --
programmers are typically''satisfied with whatever language they happen
to use, because it dictates the way they think about programs" -- so too
are people often ``satisfied'' with their social environments, because
these tend to dictate the way they think and act in life.
