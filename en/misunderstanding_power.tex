\begin{quote}
\textbf{Wikipedia}: Zipf's law states that given some corpus of natural
language utterances, the frequency of any word is inversely proportional
to its rank in the frequency table. Thus the most frequent word will
occur approximately twice as often as the second most frequent word,
three times as often as the third most frequent word, etc. {[}1{]}
\end{quote}
Zipf's law (or other formulations of the same thing) govern the
\href{http://www2.econ.uu.nl/users/marrewijk/geography/zipf/index.htm}{size
of cities}, and related formulations describe
\href{http://pricetags.wordpress.com/2010/10/26/kleibers-law-growth-and-creativity-in-cities/}{energy
use}: roughly speaking, an elephant has a lower metabolism than a mouse
and is more ``energy efficient''. At that same link, we see the
suggestion that creativity and other social network effects speed up as
population grows! \emph{The anti-pattern}: how many times have we been
at a conference or workshop and heard someone say (or said ourselves)
``wouldn't it be great if this energy could be sustained all year
'round?'' Or in a classroom or peer production setting, wondered why it
is that everyone does not participate equally. ``Wouldn't it be great if
we could increase participation?'' But participation in a given
population is going falls off according to \emph{some} power law (see
Introduction to Power Laws in
\href{http://www.theuncertaintyprinciple.danoff.org/v2i3.html}{The
Uncertainty Principle, Volume II, Issue 3}). It would be a grand
illusion to assume that everyone is coming from a similar place with
regard to the various literacies and motivations that are conducive to
participation. Furthermore, a ``provisionist'' attitude (``If we change
our system we will equalize participation and access'') simply will not
work in general. Power laws are an inherent epiphenomenon of network
flow. Certainly, if you can
\href{http://peeragogy.org/practice/moderation/}{moderate} the way the
network is shaped, you can change the ``exponent'' -- for example, by
helping more people develop relevant literacies. But even so,
``equality'' remains a largely abstract notional. Note, as well, that
participation in a given activity tends to fall off over time. Many
people would like to write a hit song or a best selling novel or start a
religion, etc., but few actually do, because it takes sustained effort
over time. See the anti-pattern
``\href{http://peeragogy.org/antipatterns/magical-thinking/}{Magical
Thinking}'' for more on this. Our ability to develop new literacies is
limited. Much as Paul Graham wrote about programming languages --
programmers are typically ``satisfied with whatever language they happen
to use, because it dictates the way they think about programs'' -- so
too are people often ``satisfied'' with their social environments,
because these tend to dictate the way they think and act in life.

\subsubsection{Reference}

\begin{enumerate}
\item
  \href{http://en.wikipedia.org/w/index.php?title=Zipf\%27s\_law\&oldid=575709945}{Zipf's
  law}. (2013). In \emph{Wikipedia, The Free Encyclopedia}.
\item
  Graham, P. (2001). \href{http://www.paulgraham.com/avg.html}{Beating
  the averages}.
\end{enumerate}
