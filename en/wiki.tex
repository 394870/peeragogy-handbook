\attrib{Régis Barondeau}

\begin{quote}
In the context of P2P-learning, a wiki platform can be a useful and
powerful collaboration tool. This section will help you understand
what a wiki is and what it is not, why you should use it, how to
choose a wiki engine and finally how you could use it in a P2P
context. Some examples of P2P-learning projects run on wikis will help
you see the potential of the tool.
\end{quote}

\subsection{What is a wiki?}

For \href{http://en.wikipedia.org/wiki/Ward_cunningham}{Ward Cunningham}
father of the wiki, ``a wiki is a freely expandable collection of
interlinked Web `pages', a hypertext system for storing and modifying
information - a database, where each page is easily editable by any user
with a forms-capable Web browser client'' {[}1{]}.

According to Wikipedia : ``a wiki is a website whose users can add,
modify, or delete its content via a web browser using a simplified
markup language or a rich-text editor'' {[}2{]}.

You can watch this CommonCraft video
\href{http://www.youtube.com/watch?v=-dnL00TdmLY}{wiki in plain
english}~to better understand what a wiki is.

\subsection{What differentiates the wiki from other co-editing tools?}

The previous definitions show that a wiki is a ``website,'' in other
words it is composed of pages that are connected together by
hyperlinks.In additiont every authorized person (not all wikis are
totally open like Wikipedia) can edit the pages from a web browser,
reducing time and space constrains. In case one saves a mistake or for
any other reason would like to go back to a previous version, a feature
called ``history'' allows users to see previous versions and to roll
back any of them. This version history allows also to compare versions
avoiding the cluttered of the ``commentaries rainbow'' we are used too
in popular Word processors. For example if you work on a wiki page, and
come back later on, you will be able to catch up by comparing your last
version with the lastest version of someone else.

Tools like~\href{https://docs.google.com/}{Google
Docs}~or~\href{http://en.wikipedia.org/wiki/Etherpad}{Etherpad}~are
design to enable co-editing on a single document. This can be seen as a
``wiki way'' of working on a document as it is web based and includes
versioning. But it is not a wiki because a single document is not a
website. Those tools offer realtime collaboration which wikis do not and
are so far easier to use for beginners as they work
in~\href{http://en.wikipedia.org/wiki/WYSIWYG}{WYSIWYG}~mode, which many
wikis do not support. ~However, the advanced
features~\href{http://en.wikipedia.org/wiki/Wiki_syntax}{wiki markup
language}~make it a more powerful tool.~In summary, tools like Googles
Docs or Etherpad are a great way to quickly collaborate (synchronously,
asynchronously, or a mixture of both) on a single document for free,
with a low barrier to entry and no technical support. (Note that
Etherpad does have a ``wiki-links'' plugin that can allow it to be used
in a more wiki-like way;~\href{https://hackpad.com/}{Hackpad}~is another
real-time editing tool that prominently features linking -- and it
claims to be ``the best wiki ever''.)

Using a real wiki engine is more interesting for bigger projects and
allows a huge number of users to collaborate on the same platform. A
wiki reduces the coordination complication as e-mails exchanges are no
more needed to coordinate a project. On the other hand it can help us
deal with complexity ({[}3{]}, {[}4{]}) especially if you put basic
simple rules in place~like the
Wikipedia's~\href{http://en.wikipedia.org/wiki/NPOV}{neutral point of
view}~to allow every participant to share her or his ideas.

Going back to the continuum we talked about before, some tools like
Moodle, SharePoint, WordPress, Drupal or others have build in wiki
features. Those features can be good but will typically not be as good
for wiki-building purposes as a well-developed special-purpose wiki
engine. In other words, those tools main focus is not the wiki, which is
only a secondary feature. When you choose a real wiki engine
like~\href{http://www.mediawiki.org/}{Mediawiki},~\href{http://www.tiki.org/}{Tiki},~\href{http://foswiki.org/}{Foswiki},
etc., the wiki will be your platform, not a feature of it. For example
if you start a wiki activity in a Moodle course, this wiki will be only
visible to a specific group of students and searchable only to those
students. On the other hand if your learning platform is a wiki, the
whole platform will be searchable to all members regarding their
permissions. We are not saying here that a wiki is better than other
tools but if you need a wiki engine to address your needs you may
consider going with a strong wiki engine rather than a ``micro-wiki''
engine embedded in an other tool.

\subsection{Why use a wiki?}

Those are the main reasons you should consider a wiki for your peer
learning projects :

\begin{itemize}
\itemsep1pt\parskip0pt\parsep0pt
\item
  To reduce coordination complication by having a central and always up
  to date place to store your content. You will reduce e-mail usage
  drasticly, and have access to your content from everywhere using any
  operating system.
\item
  To keep track of the evolution of your project and be able to view or
  roll back any previous version of a wiki page using the history
  feature.
\item
  To make links between wiki pages to connect ideas and people but also
  make links to external URL's. This last possibility is very handy to
  cite your sources.
\item
  To deal with complexity. As a wiki allows anyone to contribute, if you
  set some easy rules like Wikipedia's NPOV (Neutral Point of View), you
  will be able to catch more complexity as you will allow everyone to
  express his or her opinion. Wikis also integrate a forum or comment
  feature that will help you solve editing conflicts.
\item
  To deal with work in progress. A wiki is a great tool to capture an on
  going work.
\item
  To support transparency by letting~every members of the community see
  what others are doing.
\item
  To support a network structure as a wiki is by essence an horizontal
  tool.
\end{itemize}

Using a hyperlinks you can\ldots{}

\begin{quote}
\textbf{Gérard Ayache}:~ ``\ldots{}jump by a single click from a network
node to the other, from a computer to an other, from one information to
the other, from one univers to the other, from one brain to the
other.''~ (Translated from {[}5{]}.)
\end{quote}

\subsection{How to choose a wiki engine?}

You will find more than a hundred different wiki engines.

The first main distinction is between open source ones that are free to
download and commercial ones you will have to pay for. You will find
powerful engines on both sides open source and commercial. Sometimes the
open source ones look less polished at first sight but are backed by a
strong community and offer a lot of customization possibilities. The
commercial are sold like a package, they are nicely presented but often
they offer less customization on the user side and additional feature or
custom made tools will cost you an extra fee.

The second distinction that we can make is between wiki farms and
self-hosted wikis.
The~\href{http://en.wikipedia.org/wiki/Wiki_hosting_service}{wiki
farm}~is a hosting service you can find for both open source or
commercial wikis. The goal of those farms is to simplify the hosting of
individual wikis. If you don't want to choose a wiki farm hosting, you
will have to host the wiki on your own server. This will give you more
latitude and data privacy but will require more technical skills and
cost you maintenance fees.

The~\href{http://www.wikimatrix.org/}{Wikimatrix}~web site will help you
choose the best wiki for your needs. It allows you to compare the
features of more than a hundred wiki
engines.~\href{http://c2.com/cgi/wiki?TopTenWikiEngines}{Here}~is the
top ten list of the best wiki engines by Ward Cunningham.

\subsection{How can a wiki be useful in a peeragogy project?}

A wiki is a good tool collaborative projects and a specially suited for
work in progress as you can easily track changes using the history,
compare those version and if necessary roll back a previous versions. In
other words, nothing gets lost.

Here are some ideas about how to use a wiki in a peeragogy project :

\begin{itemize}
\item
  \textbf{Use a wiki as your learning platform}.~It can also
  support~\href{http://socialmediaclassroom.com/host/peeragogy/wiki/connectivism-practice-how-organize-a-mooc}{Massive
  Open Online Courses (MOOCs)}.~A wiki will help you organize your
  \href{http://socialmediaclassroom.com/host/peeragogy/wiki/organizing-a-learning-context}{learning
  context}. You can choose to give access to your wiki only to the
  project participants or open it to the public
  like~\href{http://www.wikipedia.org/}{Wikipedia}. Using hyperlinking,
  you will operationalize the theory
  of~\href{http://en.wikipedia.org/wiki/Connectivism}{connectivism}~by
  connecting nodes together.~As a learning platform wikis are powerful
  because you can easily see what others are doing, share with them, get
  inspired, merge ideas or link to ideas. In other words, it creates
  emulation between learners.~For additional ressources about wiki in
  education follow this
  Diigo~\href{http://www.diigo.com/user/regisb/wiki\%20education}{link}.
\end{itemize}

\begin{itemize}
\item
  \textbf{Manage your peeragogy project}. A wiki is an excellent tool
  for project collaboration. Above all, the wiki can be a central place
  for peer learners to write or link to content. Even if you use several
  technologies to run your project as we did to write this handbook, at
  the end of the day, all the content can be centralized on a wiki using
  direct writing on wiki pages or hyperlinks. This way members~can
  access the content from anywhere and from any device connected to the
  internet using any platform or application and they will always see
  the most recent version while being able to browse through the
  versions history to understand what has changed since their last
  visit.
\end{itemize}

\begin{itemize}
\itemsep1pt\parskip0pt\parsep0pt
\item
  \textbf{Publish your project}. As a wiki is a website you can easily
  use it to show your work to the world. Regarding web design, don't
  forget that a wiki can look way better than a Wikipedia page if you
  customize it
\end{itemize}

\subsection{Examples of peeragogy projects run on wikis}

\href{http://www.appropedia.org/Welcome_to_Appropedia}{Appropedia}~is a
wiki site~for collaborative solutions
in~\href{http://www.appropedia.org/Sustainability}{sustainability},~\href{http://www.appropedia.org/Poverty}{poverty}~reduction
and~\href{http://www.appropedia.org/International_development}{international
development}~through the use of
sound~\href{http://www.appropedia.org/Principles}{principles}~and~\href{http://www.appropedia.org/Appropriate_technology}{appropriate
technology}~and the sharing of wisdom and
\href{http://www.appropedia.org/Project}{project}~information. The site
is open to stakeholders~to find, create and improve scalable and
adaptable solutions.

\href{http://en.wikipedia.org/wiki/Wikipedia:Teahouse}{Teahouse}~is a
peeragogy project run on a wiki that gives newcomers a place to learn
about Wikipedia culture and get feedback from experienced Wikipedians.

\subsection{What are the best practices when using a wiki?}

\begin{itemize}
\itemsep1pt\parskip0pt\parsep0pt
\item
  \textbf{Cofacilitation}~-- help each other learn, help each other
  administer
\item
  \textbf{Self-election}~-- enable people to choose what they want to
  work on, at their own pace, in their own way
\item
  \textbf{Communication}~-- use comment threads and talk pages to
  discuss wiki changes
\item
  \textbf{Documenting changes}~-- most wikis enable editors to write
  very brief descriptions of their edits
\item
  \textbf{Rules}~-- keep rules at a minimum level to avoid chaos without
  constraining creativity
\item
  \textbf{Fun}~-- make it fun for people to contribute
\end{itemize}

\subsection{References}

\begin{enumerate}
\itemsep1pt\parskip0pt\parsep0pt
\item
  Leuf, Bo, et Ward, Cunningham. 2001. The Wiki way : quick
  collaboration on the Web. Boston: Addison-Wesley, xxiii, 435 p. p.14
\item
  \href{http://en.wikipedia.org/wiki/Wiki}{Wiki} on Wikipedia
\item
  Andrus, Calvin D. 2005.~\href{http://ssrn.com/abstract=755904}{Toward
  a complex adaptative intelligence community - The wiki and the blog}.
  Studies in Intelligence. vol. 49, no 3. Online :
\item
  Barondeau, Régis.
  2010.~\href{http://www.regisbarondeau.com/Chapitre+4\%3A+Analyse+du+cas\#Synth_se}{La
  gestion de projet croise le wiki}. École des Sciences de la Gestion,
  Université du Québec à Montréal, 180 pp.
\item
  Ayache, Gérard. 2008. Homo sapiens 2.0 : introduction à une histoire
  naturelle de l'hyperinformation. Paris: Milo, 284 p. p.179
\end{enumerate}
