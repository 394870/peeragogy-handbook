Discerning patterns helps us build our vocabulary or repertoire for
peer-learning projects.

\begin{quote}
\emph{{[}W{]}e saw that language use is typically what we have to go on,
from an analytical perspective. Generally, if we are not starting with
language, we arrive at it soon enough. Language becomes something to pay
attention to, in much the same way in which Buddhist practitioners have
for centuries spent time watching their breath. --- From
"}\href{http://paragogy.net/ParagogicalPraxisPaper}{Paragogical
Praxis}\emph{" by Joe Corneli}
\end{quote}
The challenge of discerning a peeragogical pattern revolves around a
meta-awareness of language. For example, in building a peer learning
profile, a participant might identify an interest such as organic
gardening. We notice this is a pattern when it repeats; when organic
gardening is frequently listed among interests listed by participants in
their introductions. The classic example of a pattern is ``a place to
wait'' \emph{---}a type of space found in many architectural and urban
design projects. Once a pattern is detected, give it a title and write
down how the pattern works in a peer learning context. What does this
pattern say about the self-selection process of the group? Without
jumping to conclusions, consider that an interest in organic gardening,
for example, \emph{might} indicate the participants are oriented to
cooperation, personal health, or environmental activism (emphasis on
\emph{might!}).
