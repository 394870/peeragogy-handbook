Actually, of course, living beings are never~\emph{really}~in stasis.~
It just sometimes feels that way.~ Different anti-patterns
like~\href{http://socialmediaclassroom.com/host/peeragogy/forum/anti-patterns-concerns-complaints-and-critiques\#comment-2267}{Isolation}~or~\href{http://socialmediaclassroom.com/host/peeragogy/forum/anti-patterns-concerns-complaints-and-critiques\#comment-1808}{Navel-Gazing}~have
described different aspects of the~\emph{experience}~of feeling like one
is in stasis.~ Typically, what is happening in such a case is that one
or more dimensions of life are moving very slowly.

For instance, it seems we are not able to get programming support to
improve this version of the Social Media Classroom, for love or money,
since all developer energy is going into the next version.~ This isn't
true stasis, but it can feel frustrating when a specific small feature
is desired, but unavailable. The solution?~ Don't get hung up on small
things, and find the dimensions where movement~\emph{is}~possible.~ In a
sense this is analogous to eating a balanced diet.~ You probably
shouldn't only eat grilled cheese sandwiches, even if you like them a
lot.~ You should go for something different once in a while. This is
also related to the pattern that talks about
"\href{http://socialmediaclassroom.com/host/peeragogy/forum/patterns-and-use-cases\#comment-2320}{Carrying
Capacity}".~ There is always some dimension on which you can make
progress -- it just might not be the same dimension you've recently
over-harvested!
