\subsubsection{Main Actors}

Pierre and Marie - recently married.

\subsubsection{Main Success Scenario}

\begin{enumerate}
\item
  They furnished off an apartment from a Sears \& Roebuck sale. Their
  coolerator was crammed with TV dinners and ginger ale. (She couldn't
  cook.)
\item
  But when Pierre found work, the little money coming in worked out
  well. They got a hi-fi phono, and boy, did they let it blast -- Seven
  hundred little EPs, all rock, rhythm and jazz.
\item
  When the sun went down, the rapid tempo of the music sort of fell (for
  various reasons).
\item
  They bought a souped up Mercedes -- a cherry red '53 -- and drove it
  down to New Orleans to celebrate their anniversary.
\item
  ``C'est la vie,'' say the old folks, ``It goes to show you never can
  tell!''
\end{enumerate}

(Après Chuck Berry.)

\subsubsection{Thoughts}

I tried to use the familiar song to suggest that pæragogy works in
personal relationships, too.~ Compare the above story with this quote
from Leopold von Sacher-Massoch\ldots{}:

\begin{quote}
"That woman, as nature has created her and as man is at present
educating her, is his enemy. She can only be his slave or his despot,
but \emph{never his companion}. This she can become only when she has
the same rights as he, and is his equal in education and work."
\end{quote}

I don't know if Sacher-Massoch is particularly reliable as a feminist.~
But it~\emph{is}~interesting to look at ``companionship'' (along with
membership in the same age cohort) as a criterion for a peer-like and
working relationship in the story.~ It's unclear as to whether Pierre \&
Marie have ``equal'' roles (he found work, but it's not in any way
implied that she was working\ldots{} so how did she spend her time?~
Etc.).
