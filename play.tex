Authors: Bryan Alexander, Anna Keune What does play have to do with peer
learning? We can answer that question by thinking it through on
different levels. \textbf{1. The individual plays and learns} There are
deep links between play and learning for a single person. Consider, for
instance, the way we learn the rules of a game through playing it. The
first times we play a card game, or a physical sport, or a computer
simulation we test out rule boundaries as well as our understanding.
Actors and role-players learn their roles through the dynamic process of
performance. The resulting learning isn't absorbed all at once, but
accretes over time through an emergent process, one unfolding further
through iterations. In other words, the more we play a game, the more we
learn it. In addition to the rules of play, we learn about the subject
which play represents, be it a strategy game (chess, for example) or
simulation of economic conflict (\href{http://www.eveonline.com/}{Eve
Online}). First, this is a function of art. We learn about an historical
event through looking at paintings of it, watching representation
movies, reading historical novels or nonfiction books, or - considering
games as aesthetic objects - playing games about a topic. This
representational function has been a significant aspect of gaming from
Kriegspiel on; arguably, as far back as the conception of chess as
statecraft tutorial. Second, the process of play reveals new dimensions
to the subject, as different approaches and combinations display more
subject content. Consider the end game of chess contrasted with the
opening moves, or how successful video game players must learn to cope
with ever-increasing challenges and capacities. At another level, beyond
a game's rules and subject matter, play teaches us about ourselves.
Playing games teaches us about our personal nature; play, in fact, can
initiate change in a person (Vygotsky, 1966). When playing with others,
be it Fantasy Football or World of Warcraft, we also learn social rules,
from cultural knowledge to personality to teamwork mechanics. Think of
the way young children play their way into knowledge of social rules and
the physical world. Good games echo good teaching practice, too, in that
they structure a single player's experience to fit their regime of
competence (cf Vygotsky's zone of proximal learning, a la Gee (70)).
That is to say a game challenges players at a level suited to their
skill and knowledge: comfortable enough that play is possible, but so
challenging as to avoid boredom, eliciting player growth. A video game
example is the deep structure of levels, where players advance in stages
of increasing difficulty, rising only as their competency grows.
Role-playing in theater lets performers explore and test out concepts
(Boal 1979). Further, adopting a playful attitude helps individuals meet
new challenges with curiousity, along with a readiness to mobilize ideas
and practical knowledge. Indeed, the energy activated by play can take a
person beyond an event's formal limitations, as players can assume that
play can go on and on. (Bereiter \& Scardamalia, 1993) ``All systems of
play are, at base, learning systems.'' (Thompson and Brown, 97)
\textbf{2. The group plays and learns together} Games have always had a
major social component (Caillois, 2001; Huizinga, 1955), and learning
plays a key role in that interpersonal function. Using games to build
group cohesion is an old practice, actually a triusm in team sports. A
similar truism is the way one learns an opponent by playing against them
in chess. More recently we've seen rapid growth in learning simulations,
such as the market in business games. Using play, or, more narrowly,
games to build group learning follows naturally. This is a key form of
peer learning. Vygotsky's zone of proximal development is based on
groups, in that a learner is capable of greater performance with the
support of more accomplished people. In fact, play can
activatedevelopmental processes that are only internalized by an
individual through peer interaction (see also e.g., Engeström, 2001).
\textbf{3. Play and learning in the world} It is important to locate our
peeragogical moment in a world where gaming is undergoing a renaissance.
Not only has digital gaming become a large industry, but gaming has
begun to infiltrate non-gaming aspects of the world, sometimes referred
to as ``gamification.'' Putting all three of these levels together, we
see that we can possibly improve co-learning by adopting a playful
mindset. Such a playful attitude can then mobilize any or all of the
above advantages. For example,

\begin{itemize}
\item
  Two friends are learning the Russian language together. They invent a
  vocabulary game: one identifies an object in the world, and the other
  must name it in Russian. They take turns, each challenging the other,
  building up their common knowledge.
\item
  A middle-aged man decides to take up hiking. The prospect is somewhat
  daunting, since he's a very proud person and is easily stymied by
  learning something from scratch. So he adopts a ``trail name'', a
  playful pseudonym. This new identity lets him set-aside his
  self-importance and risk making mistakes. Gradually he grows
  comfortable with what his new persona learns.
\item
  A girl decides she wants to care for animals, but doesn't have access
  to critters. She plays with a virtual pet, learning some of the
  concepts - feeding, care, monitoring. She then spots those concepts in
  play elsewhere in the world, through watching movies, her parents, and
  adults in her neighborhood. As her game play escalates in complexity,
  she finds these caring concepts in other systems, gradually extending
  her thinking and abilities. Eventually a family friend gifts her with
  a dog, and she is well positioned to start practicing what she's
  simulated in play.
\end{itemize}
Shifting ground, we can also consider the \textbf{design} field as a
useful kind of playful peeragogy. It's an appropriate area for our
purposes, as the design field has long been considered as a form of
play, starting from roles (Schön, 1983). The person \emph{playing the
role} of the designer can select the contextual frame within which the
design is performed. This frame can be seen as the \emph{rules}
governing the design, the artifact and the process. These rules, as with
some games, may change over time. Therefore the possibility to adapt, to
tailor one's activities to changing context is important when designing
playful learning activities. This returns us to our third level of
game-learning, noted above, ``play and learning in theworld''. As the
``cultural memory'' of a person grows in broad social contexts, along
with it the forms of memory that can be accessed (Luria, 1930), learning
activities need to adapt. They need to be open enough to extend
side-by-side with changing practice (Fischer \& Scharff 2000) in order
to accommodate its advancing needs (Fischer, 2003). All of this
potential creativity naturally elicits the question of
\textbf{assessment}. How can we measure and validate learning in such
unconventional settings? The contrast between metrics and the ludic
impulse is strong. And yet play often already contains assessment's
seeds through rules and structures. Most games, after all, provide
metrics for measuring game progress: points, position on a board,
markers of status, and so on. We can repurpose these structures for
broader assessment since they provide clear and meaningful feedback to
players, as the gamification movement has argued (Mcgonigal). Moreover,
insofar as play involves the use and/or creation of media, we can assess
play as media work. Other players can look for argument and ideas in
selected works, or trace another player's growth through mediated play
over time. Balancing the needs of assessment and play requires some
finesse and tact. The spirit of each appears diametrically opposed to
the other, at times. Moreover, play can elicit competition, which does
not necessarily redound to mutual benefit for players. Griefing,
aggression, cut-throat tactics, defections (in the sense of game theory)
can lead to players not only exceeding colleagues' play, but actively
undermining their performance. Perhaps these problems of comparison and
measurement would benefit from a gamification approach. Assessment often
seems deliberate, opposed to the spontaneity we know imbues play;
creating a subgame of assessment might satisfy both sides of this
opposition. \textbf{Exercises}

\begin{itemize}
\item
  Use the \href{http://www.rtqe.net/ObliqueStrategies/}{Oblique
  Strategies} card deck (Brian Eno and Peter Schmidt, 1st edition 1975)
  to spur playful creativity. Each card advises players to change up
  their creative process, often in surprising directions.
\item
  Take turns making and sharing videos. This online collaborative
  continuous video storytelling involves a group of people creating
  short videos, uploading them to YouTube, then making playlists of
  results. Similar to \href{http://clipkino.info/}{Clip Kino}, only
  online.
\item
  Engage in theater play using Google+ Hangout. e.g. coming together
  with a group of people online and performing theatrical performances
  on a shared topic that are recorded.
\item
  Pick a computer game which embodies some part of what you want to
  learn.
\item
  " " non-digital game.
\item
  Adopt a wildly creative persona as your learning identity. See how
  their biography grows.
\end{itemize}
\subsection{References}

Beardon, C., 2002. Digital Bauhaus: aesthetics, politics and technology.
Digital Creativity, 13 4{]}, pp.169-179. Boal, A., 1979. Theatre of the
oppressed. 3rd ed. London: Pluto Press. Bereiter, C. and Scadamalia, M.,
1993. Surpassing ourselves, an inquiry into the nature and implications
of expertise. Peru, Illinois: Open Court. Roger Caillois, \emph{Man,
Play, and Games}. Trans. Meyer Barash. Urbana: University of Illinois
Press, 2001 Engestroem, Y., 2001. Expansive Learning at Work: toward an
activity theoretical reconceptualization. Journal of Education and Work,
14{[}1{]}, pp.133-156. doi:10.1080/13639080020028747 Fischer, G. and
Schaff, E., 2000. Meta-Design: design for designers. In: 3rd conference
on designing interactive systems {[}DIS '00{]}. New York:ACM.
pp.396-405. doi:10.1145/347642.347798 Fischer, G., 2003. Meta-Design:
beyond user-centered and participatory design. In: HCI International
2003. Mahwah: Lawrence Erlbaum Associates, pp.88-92 Available at:
http://citeseerx.ist.psu.edu/viewdoc/summary?doi=10.1.1.6.8238 Johan
Huizinga, \emph{Homo Ludens: A Study of the Play Element in Culture}.
Boston: Beacon, 1955 Luria, A. R., 1930. Mastery of tools. In: Ape,
primitive man, and child: essays in the history of behaviour, A.R. Luria
and L.S. Vygotsky. Translated by E. Rossiter, 1992. New York: Harvester
Wheatsheaf. Available at:
http://www.marxists.org/archive/luria/works/1930/child/ch07.htm Jane
Mcgonigal, Reality is Broken. New York: Penguin, 2011.
\href{http://www.sffaudio.com/?p=38223}{One recent story} about Mitch
Resnick and the role of play in children's learning
(\href{http://www.legobuildersoftomorrow.com/podcast1.mp3}{direct link
to mp3}). Schoen, D., 1983. The reflective practitioner, how
professionals think in action. New York, NY. Basic Books. Douglas Thomas
and John Seely Brown, A New Culture of Learning: Cultivating the
Imagination for a World of Constant Change. CreateSpace, 2011. Vygotsky,
L. S.,1978. The interaction between learning and development. In: M.
Cole, V. John-Steiner, S. Scribner and E. Souberman, eds. 1978. Mind in
society: the development of higher psychological processes. Cambridge,
MA: Harvard University Press, pp.79-91. Vygotsky, L. S., 1933. Play and
its role in the mental development of the child. Translated by C.
Mulholland, 1966. In: Voprosy psikhologii, 6. {[}online{]} Available at:
http://www.marxists.org/archive/vygotsky/works/1933/play.htm {[}Accessed
30 October 2011{]}.

\begin{center}\rule{3in}{0.4pt}\end{center}

\href{http://socialmediaclassroom.com/host/peeragogy/forum/initial-rough-outline\#comment-1618}{Link
to forum discussion}.
\href{http://socialmediaclassroom.com/host/peeragogy/wiki/initial-outline-source-book}{Link
to outline page}.
