\documentclass[article]{twoside}

\usepackage{hyperref}

\begin{document}

\title{Roadmaps in Peer Learning for Sustainable Learning Design}

\numberofauthors{6} %  in this sample file, there are a *total*
% of EIGHT authors. SIX appear on the 'first-page' (for formatting
% reasons) and the remaining two appear in the \additionalauthors section.
%
\author{
% You can go ahead and credit any number of authors here,
% e.g. one 'row of three' or two rows (consisting of one row of three
% and a second row of one, two or three).
%
% The command \alignauthor (no curly braces needed) should
% precede each author name, affiliation/snail-mail address and
% e-mail address. Additionally, tag each line of
% affiliation/address with \affaddr, and tag the
% e-mail address with \email.
%
% 1st. author
\alignauthor
Joe Corneli\\
\affaddr{Knowledge Media Institute}\\
\affaddr{The Open University}\\
\affaddr{Milton Keynes, UK}\\
\email{joseph.corneli@open.ac.uk}
% 2nd. author
\alignauthor
Charles Jeffrey Danoff\\
\affaddr{Mr Danoff's Teaching Laboratory}\\
\affaddr{Chicago, IL}\\
\email{charles@danoff.org}
% 3rd. author
\alignauthor
Fabrizio Terzi \\
\affaddr{Bergamo HUB}\\
\affaddr{Bergamo, IT}\\
\email{fabrizio.terzi@gmail.com}
\and  % use '\and' if you need 'another row' of author names
% 4th. author
\alignauthor Charlotte Pierce \\
\affaddr{Pierce Press} \\
\affaddr{Arlington, MA}\\
\email{charlotte.pierce@gmail.com}
% 5th. author
\alignauthor John Graves \\
\affaddr{Auckland University of Technology}\\
\affaddr{Auckland, NZ}\\
\email{jgraves@aut.ac.nz}
% 6th. author
\alignauthor R\'egis Barondeau \\
\affaddr{Universit\'e du Qu\'ebec \`a Montr\'eal}\\
\email{regis.barondeau@mac.com}
}

% There's nothing stopping you putting the seventh, eighth, etc.
% author on the opening page (as the 'third row') but we ask,
% for aesthetic reasons that you place these 'additional authors'
% in the \additional authors block, viz.
% \additionalauthors{Additional authors: John Smith (The Th{\o}rv{\"a}ld Group,
% email: {\texttt{jsmith@affiliation.org}}) and Julius P.~Kumquat
% (The Kumquat Consortium, email: {\texttt{jpkumquat@consortium.net}}).}
% \date{30 July 1999}

% Just remember to make sure that the TOTAL number of authors
% is the number that will appear on the first page PLUS the
% number that will appear in the \additionalauthors section.

\maketitle
\begin{abstract}

Following a year of productive learning and work culminating in the first edition of The Peeragogy Handbook we reflect here on lessons learned and patterns uncovered. In the second half of the paper we outline our goal: to transition from an innovative theoretical project to a sustainable, easily replicable peer project problem solving accelerator dynamically measuring assessment.

\end{abstract}

%
%
% New Section note for visual aid while editing
%
%

\section*{Categories and Subject Descriptors}
[{\bf Applied computing}]: {Education}, {Collaborative learning};
[{\bf Applied computing}]: {Document management and text processing}, {Document preparation}---\emph{Hypertext / hypermedia creation};
[{\bf Information systems}]: {Information systems applications}, {Collaborative and social computing systems and tools}---\emph{Wikis}

\terms{Human Factors}

\keywords{Peer learning, peer-to-peer, print on demand} % NOT required for Proceedings

%
%
% New Section note for visual aid while editing
%
%

\section{Introduction - Purpose of the Peeragogy Project}

The Peeragogy Project is a group of learners trying to uncover the most effective ways for peers to collaboratively learn. Our methodology centers on examining and recording how we learn and work as well as studying examples of others (e.g., Wikipedia, P2PU, the Invisible College, Greek Agoras, Guilds, etc.)

Participants must bring self-knowledge and no small measure of honesty to the peer-learning project in order to accurately enunciate their motivations. If everyone in your peer learning project asks “What brings me here?” “How can I contribute?” and “How can I contribute more effectively?” things will really start percolating. Test this suggestion by asking these questions of yourself and taking action on the answers!

The primary motivators reported by participants in the Peeragogy project include:

Acquisition of training or support in a topic or field;
Building relationships with interesting people;
Finding professional opportunities through other participants;
Creating or bolstering a personal network;
More organized and rational thinking through dialog and debate;
Feedback about their own performance and understanding of the topic.
Each of those motivators can affect the vitality of the peeragogical process and the end result for the individual participant.

Wikipedia-like project for learning
but with a rather different orientation from Wikiversity
Analogy to the historical university (as a guild)
Differences between this, P2PU, Wikiversity, etc.

    1 or more example(s) from each would be great!:

        working through the book to find one or two papers per section

        Christopher's Peeragogy Weekly Summaries

        The Forums

        Google+

        SMC Videos \& Chats

        Twitter

    Public Relations?? $\rightarrow$ Who has written about us in the last year?\url{https://cultivatingchange.wp.d.umn.edu/community/the-peeragogy-handbook/}


\section{2012 Problem Solved - Peeragogy Handbook}

Peeragogy is replicable techniques and patterns for effective peer learning and problem solving. Without any formal institution or pay, a group of volunteer peers came together and wrote a cohesive preliminary treatise on peeragogy, far richer than what any one of these people could have produced alone.

We utilized a variety of technologies to get work done: live meetings and loose bureaucracy via team organizations and leaders of different sub-projects. Another approach can be seen in Figure 1 by Amanda Lyons, condensing 4 paragraphs on how peer learners can get started on their endeavors into 1 pleasing image. 

Figure 1: Work by Amanda Lyons from \url{http://peeragogy.org/peeragogy-in-action/}

To facilitate asynchronous collaboration: Social Media Classroom, Social Bookmarking, Google Docs, Google+, WordPress, LaTeX, Lulu, Twitter, Blackboard Collaborate, Git, the Creative Commons Zero Public Domain Dedication, Google Hangouts and more. Obviously each offered its own pros and cons, but peeragogy is not bound to particular digital tools nor the digital world at all. For example, a physical interpretation was taken by Anna, Paola, Gigi and Amanda who held a workshop at the Open Knowledge Conference 2012 in Helsinki.\footnote{ \url{http://okfestival.org/peeragogy-handbook-workshop/} and \url{http://www.youtube.com/watch?v=P2UJrN58MVI} leading to the peeragogy project participating in the Open Book. Also, Vanessa and Charlie collaborated to share the Handbook at the Open Education Conference 2012 in Canada as part of its Remixathon \url{http://openedconference.org/2012/happenings/remixathon-info/}}

A different angle on peer learning that is also peeragogical is taken from A. T. Ariyaratne's essay on Rural Self Help[7], one of the foundational writings of the Sarvodaya Shramadana movement in Sri Lanka[fn 4], begins:

Nobody needs to teach rural communities about ``group effort'' and ``self-help''. [...] The real question, therefore, is to examine what are the constraints that exist inhibiting the expression of their group effort and self-help qualities designed to improve food and nutrition levels, clothing, shelter, health, sanitation, education and cultural life?

We approach peer learning in a similar spirit: it is something we all know how to do, but can't always do well. Intuitively, there are bound to be difficulties for a group of peers studying a subject together, outside a traditional classroom or without a teacher. Indeed, peer learning is different from other forms of group effort, the proverbial ``barnraising'' for example, in which the persons involved can be presumed to know how to build barns - or at least to know someone who knows, and stand ready to take orders.

Typically, peers are not experts in learning, didactics, or in the subject they are studying, and are faced with multiple difficulties associated with putting together knowledge about the subject, assembling a suitable learning strategy, and communicating with one another.\footnote{After A.T. \ldots to here is viz from \url{http://www.paragogy.net/wiki/ParagogyPaper2}}

As of early 2013 we are translating the handbook into multiple languages, and the Italian effort is focused on the MediaWiki powered Wikibooks platform; additionally team members have experimented with the Federated Wiki for translation and other possibilities. Our hub for archival and linkage purposes remains the Social Media Classrom; thus, with each new tool the question of connecting back into the overall flow of work and data coherently poses new challenges and solutions are usually, but not always found.

Specifically we seek to transition from an innovative theoretical project to a sustainable, easily replicable peer project problem solving accelerator dynamically measuring assessment.

For context all of this was instigated by Howard Rheingold adapting the Corneli and Danoff (2011) work on Paragogy, their word for conscious practice of peer learning, along a less academic paper, more practical, DIY tract convened a group of us including Bryan Alexander, Paul Allison, R\'egis Barondeau, Doug Breitbart, Suz Burroughs, Jay Cross, Julian Elve, Mar\'ia Fernanda, James Folkestad, Kathy Gill, Gigi Johnson, Anna Keune, Roland Legrand, Christopher Neal, Ted Newcomb, Stephanie Parker, Charlotte Pierce, David Preston, Paola Ricaurte, Stephanie Schipper, Fabrizio Terzi, and Geoff Walker together with the goal of producing a handbook explaining how other peers could learn effectively with other peers on a given project. Producing the book itself was our self-fulfilling method of researching our ideas about peer learning.

The current beginning of the book illustrates its purpose, philosophy and where it can potentially take readers:

This book presents a range of techniques that self-motivated learners can use to connect with each other and develop stronger communities and collaborations. The book is addressed to everyone who is interested in how learning works, whether you’re an educator, a hobbyist, an artist, a home-school student, an employee, a parent, an activist, an archivist, a mathematician, or a tennis player.  The book was written by a bunch of people who think learning is cool.

\subsection{What did we learn in 2012?}

In the Peeragogy project, we've identified several basic and more elaborated patterns that describe these effects. The central pattern is that of a Roadmap, which can apply at the individual level (as a personal learning plan\footnote{see What are Learning Analytics? by George Siemens \url{http://www.elearnspace.org/blog/2010/08/25/what-are-learning-analytics/} for more on individual learner profiles as a context individual learning plans and author Joe Corneli's example here \url{http://en.wikiversity.org/w/index.php?title=User:Arided/Paragogical_Profile&oldid=1003221}}) or at a project level. The roadmap might include a ``reason why'', thoughts about the goal, indicators of progress, a section for future work, and so forth. 

Group culture around maintaining and updating the roadmap define project governance. Other patterns flesh out the project's emergent properties as a sort of ``agora'':

Convene:

    Focusing on a Specific Project - Lightbulb moment: Most specific projects involve learning!

    Creating a Guide - Overviews expose the lay of the land collecting content and stories.

Organize:

    Roles - Specialize and mix it up. Play to participants strengths and skills.

    Newcomer - Create a guide for ``beginner's mind'' and help avoid need to bring new members up to speed each ``meeting.''

    Wrapper - Consolidate and contain. Front end appearance to participants.

Cooperate:

    Heartbeat - The ``heartbeat'' of the group keeps energy flowing.

    Carrying Capacity - Know your limits, find ways to get other people involved.

    Moderation - When leaders step back, dynamics can improve; moderator serves as champion and editor.

    Polling for Ideas - Invite brainstorming, collecting ideas, questions, and solutions.

    Patient - Do not hold grudges and do not make pithy comments about team members or other institutions.

    Sacrifice - Participants must be willing to sacrficice credit.

Believe:

    Participants need to buy in to the idea or philisophy behind a project.

Assess:

    Use or Make? - Repurposing, tinkering, or creating from scratch?

    Discerning a Pattern - Found a pattern? Give it a title and example.


These patterns provide a natural framework for participatory design and research in peer learning projects.

\subsection{Peer Survey}

In an effort to document ``the paths in the grass'' (NEEDS REFERENCE/CONTEXT/LINK) that come from unexpected links between different things in our successful publication of last year's Handbook we prepared a short survey for Peeragogy project participants. We asked people how they had participated (e.g Signing up for access to the Social Media Classroom and mailing list, Joining the Google+ Community, Authoring articles, etc.), and what goals or interests motivated their participation. We asked them to describe the Peeragogy project itself in terms of its aims and to evaluate its progress over the first year of its existence. As another measure of ``investment'' in the project, we asked, with no strings attached, whether the respondent would consider donating to the Peeragogy project. This survey was circulated to 223 members of the Peeragogy Google+ community (both en masse, and with individual +Name call-outs), as well as to the currently active members of the Peeragogy mailing list.

The outline of the project's purpose ranged from the general: "How to make sense of learning in our complex times" -- to much more specific:

"Push education further, providing a toolbox and [techniques] to self-learners. In the peeragogy.org introduction page we assume that self-learners are self-motivated, that may be right but the Handbook can also help them to stay motivated, to motivated others and to face obstacles that may erode motivation."

Considering motivation as a key factor, it is interesting to observe how various understandings of the project's aims and its flaws intersected with personal motivations. For example, one respondent (who had only participated by joining the Google+ community) was: "[Seeking] [i]nformation on how to create and engage communities of interest with a shared aim of learning."

More active participants justified their participation in terms of what they get out of taking an active role, for instance:

Contributing to the project allows me to co-learn, share and co-write ideas with a colourful mix of great minds. Those ideas can be related to many fields, from communication, to technology, to psychology, to sociology, and more.

The most active participants justified their participation in terms of beliefs or a sense of mission:

"Currently we are witnessing many efforts to incorporate technology as an important tool for the learning process. However, most of the initiatives are reduced to the technical aspect (apps, tools, social networks) without any theoretical or epistemological framework. Peeragogy is rooted in many theories of cooperation and leads to a deeper level of understanding about the role of technology in the learning process. I am convinced of the social nature of learning, so I participate in the project to learn and find new strategies to learn better with my students."

"I wanted to understand how ``peer production'' really works. Could we create a well-articulated system that helps people interested in peer production get their own goals accomplished, and that itself grows and learns? Peer production seems linked to learning and sharing - so I wanted to understand how that works."

They also expressed criticism of the project, implying that they may feel rather powerless to make the changes that would correct course:

"Sometimes I wonder whether the project is not too much 'by education specialists for education specialists'. I have the feeling peer learning is happening anyway, and that teens are often amazingly good at it. Do they need `learning experts' or 'books by learning experts' at all? Maybe they are the experts. Or at least, quite a few of them are."

Another respondent was more blunt:

"What problems do you feel we are aiming to solve in the Peeragogy project? We seem to not be sure. How much progress did we make in the first year? Some..got stuck in theory."

But, again, it's not entirely clear how the project provides clear pathways for contributors to turn their frustrations into changed behavior or results. Additionally we need to be entirely clear that we are indeed paving new ground with our work. If there are proven peer learning methods out there we have not examined and included in our efforts, we need to find and address them. Peeragogy is not about reinventing the wheel.

It's also not entirely clear whether excited newbies will find pathways to turn their excitement into shared products or process. For example, one respondent (who had only joined the Google+ community) had not yet introduced their fascinating projects publicly:

"I joined the Google+ community because I am interested in developing peer to peer environments for my students to learn in. We are moving towards a community-based, place-based program where we partner with community orgs like our history museum for microhistory work, our local watershed community and farmer's markets for local environmental and food issues, etc. I would love for those local efforts working with adult mentors to combine with a peer network of other HS students in some kind of cMOOC or social media network."

Responses such as this highlights our need to make ourselves available to hear about exciting new projects from interested peers, simultaneously giving them easier avenues to share. Our work on developing a peeragogy accelerator in the next section is an attemt to address this situation.

\subsection{Survey Analysis}

Clearly, many participants will have intermediate levels of investment -- basically ``social consumers'' of the project as a ``product''. However, if we think about the metaphor of a college or university, this description also applies to most members of the student body, who are physically present only for a brief portion of the institution's history, and who may not join the student government etc. A blueprint for a distributed university or peers we will discuss briefly in the conclusion perhaps named ``Peeragogy.EDU''\footnote{See \url{http://peeragogy.org/knight-foundation-prototype-fund-proposal-unfunded/}} should include plenty of room for people who take a less involved role. Nevertheless, integrating some of these light-weight contributions (like blog posts about the ideas) would be an important role for more active participants to take on.

Some of the big challenges can be parsed out using our high-level patterns. First, in our work, we uncovered pretty much everyone needs a team. Building off that, these issues could define new project ``roadmaps'' and ``teams'':

    Cooperate ~ How to build a really strong collaboration?

        A team is not a group of people who work together. A team is a group of people who trust each other...

    Convene ~ How to build a more practical focus?

        The insight that the project will thrive if people are working hard on their individual problems and sharing feedback on the process seems like the key thing going forward. This feels valuable and important.

    Organize ~ How to connect with motivated newcomers?

        I just came on board a month ago. [...] I am designing a self organizing learning environment (SOLE) or PLE/PLN that I hope will help enable communities of life long learners to practice digital literacies.

    Assess ~ How to be effective and relevant?

        I am game to also explore ways to attach it to spaces where funding can flow based on real need in communities.

The basic workflow in the project -- discussing ideas, condensing them into written handbook sections, annotating them with new resources in informal discussion, reevaluating and rewriting -- seems relatively sustainable, as long as we have community members and editors who are willing to do the work. However, the long-term relevance of the project will come from building workflows that are less self-referential and more applied. With the current foundation in theory and examples from literature and day-to-day activities of participants, we are prepared to be a more effective peer-produced accelerator for peer learning and peer production projects. The kinds of critical questions elaborated above would tend to apply to other projects as well.

\subsection{More possible ideas for this section}

We can also draw on the answers to our initial round of questions (a ``pilot study'') -- that info is summed up in http://peeragogy.org/organize/

We can also form a brief survey\footnote{\url{https://docs.google.com/forms/d/1wVFbrLmZG6PNGtJZykbIgxubfTNChAUxbt-LMFpYLmk/viewform}} recycling material from the book, e.g. the stuff from Connie Gersick that Gigi wrote about\footnote{\url{http://peeragogy.org/convene/}} -- or the reference from Ted that we reused later\footnote{\url{http://peeragogy.org/motivation/}}

%
%
% New Section note for visual aid while editing
%
%

\section{2013 Problem - Peeragogy Accelerator}

Having piloted the peeragogy patterns as a research method for understanding the peeragogy project itself, work is now underway to apply them in another peer learning and peer production community, PlanetMath, in the context of evaluating socio-technical change associated with a public beta of new software for the decade-old site.

I would emphasize the transition from ``innovative project'' in ``sustainable learning design'' in short, we've been doing ``peer support'' and ``critical thinking'' all along
Example: Joe and Charlie made a simple peer support pact outside of P2PU ---this led to--$\rightarrow$ papers ---which led to--$\rightarrow$ some of the seeds for the Peeragogy project

we're not just offering content
we're also not offering ``classes'' (or a place to organize classes)
We're working at a higher level, more strategic
We're focusing on people instead of topics/subjects
And I would also say ``Leadership led team skills''

Charlie
Fabrizio
John
Regis

\subsection{Pilot Accelerator Project: Independent Publishers of New England’s “PeerPub-U”}

Drawing on the experience and skills of its 108+ members, IPNE plans to build an open learning & support network (a.k.a. PeerPubU, or “Peer Publishing University”) to address issues in independent publishing and provide education to its members - part of its stated mission. IPNE facilitators hope that this network will dovetail with other planned membership-building efforts and help raise the standards and maximize success of indie publishers in New England in this fast-changing business sector.

On Jan. 26, 2013, regional independent publishers & authors attended the IPNE.org greater Boston branch meeting in Arlington, MA, billed as a “plenary session” for PeerPubU. In addition to the live in-person meeting, Peeragogy Handbook team members Gigi Johnson (Los Angeles, US), Roland Legrand (Antwerp, Belgium) and Anna Keune (Helsinki, Finland) joined via Google+ Hangout.
Plenary meeting observations:

At the meeting, IPNE President Tordis Isselhardt suggested a Peeragogy-style effort might more likely meet with success if the organization's members rallied around a specific project, perhaps an "Independent Publishing Handbook," (like the Peeragogy Handbook) in addition to creating resource repositories and sharing of expertise as individuals’ challenges arise.

Brainstorming ways to sustain motivation, members suggested that the 108 members of the association could earn authorship credit for contributing articles and editor credit for working on the manuscript; and could spin off their own chapters as stand-alone, profit-making publications.
Members agreed to set up the project in IPNE’s Basecamp content-management platform. Members who express interest at the branch meetings are regularly added. The Basecamp tally was 12 self-selected participants by March, 2013. 

Peeragogues attending noted that there are facilitators, but not an overall leader, of peer-learning efforts, and suggested that potential contributors and facilitators prepare by visiting peeragogy.org and extracting practices and patterns that might work best for IPNE.

IPNE officers at the meeting perked up at the suggestion that "PeerPubU" might become a case study in future editions of the Peeragogy Handbook.

Live, in-person development sessions of the pilot Peeragogy project take place at IPNE Greater Boston Regional Branch meetings on a monthly basis, and an online video meeting platform is in development.

\subsection{PlanetMath}

\subsection{The Uncertainty Principle}

\subsection{Bergamo}

Team member Fabrizio Terzi is seeking to put Peeragogy in Action via a HUB\url{http://www.the-hub.net/} Laboratory Project Art & Open Technology Incubator. He is drawing on peeragogical patterns and his peer learning network to develop an application to submit to the Bergamo, Italy Art & Tourism Council.


\subsection{More possible ideas for this section}

Also, how will we use what we learned to make 2013 better? (use
Peeragogical discoveries to improve the Peeragogy Handbook, Project,
spin-off and partner projects, etc.)

As we move into year two and an updated Handbook, we focus on translations and formalizing peeragogical patterns for problem solving; i.e., how can we (and other peers) use our ideas to better solve problems facing us directly.

In the course of this paper we will examine peeragogical applications to an online Math web cornucopia, Chicago based zine and a New England Independent Publishers association. The main problem we in 2013 and here in this paper are trying to solve is seeing if peeragogical patterns of transparency, accountability and efficiency can help us individually with pressing project problems. Specifically we seek to transition from an innovative theoretical project to a sustainable, easily replicable (peer project) problem solving accelerator.

%
%
% New Section note for visual aid while editing
%
%

\section{Meta paper section}

This section is optional at this stage

\subsection{Invited co-authors and initial contributions}

    Fabrizio Terzi Y- I'd like to contribute trying to combine and clarify more points in a visual way for ``research paper'' \& ``research presentation'' submissions . [ONE-PAGER]

    Joe Corneli Y - I'd like to contribute to structuring the argument (e.g. mindmapping) and making this into a proper, publishable, research paper (e.g. writing, editing, reference checking). I think it would be useful for me to re-use the ideas here in the ``Discussion'' section of my thesis. [ONE-PAGER]

    R\'egis Barondeau Y- ``Uses and impact of wikis and other open resources, tools, and practices in fields and application areas.'' ** [ONE-PAGER]

    John Graves Can contribute my research into the literature, most of which is listed here. Also, SlideSpeech interactivity has been put into production so delivery of the paper (the presentation of it) could be collaboratively authored in the SlideSpeech system. Here is the SlideSpeech interactivity tutorial: [ONE-PAGER]

    Samuel Rose Y - I am interested in these areas: Innovative development and/or implementation of wiki applications, Building open systems and tools, Open knowledge and information production, Uses and impact of wikis and other open resources, tools, and practices in fields and application areas \ldots and I will commit to coming up with a cohesive contribution that covers at least some of this. [ONE-PAGER]

        R\'egis: Samuel, please check, modify, complete what I started. 

    Charlotte Pierce Y [ONE-PAGER]

    Charlie Danoff Y [ONE-PAGER] Interested in Open Ed, Wikis, Sharing OER, lately been using GitHub to remix and share seconday school syllabi \& lesson plans ... interested in how peeragogical patterns might make my efforts more effective \& efficient

    Paola Ricaurte Y I'm interested in the social and cultural aspects of open collaboration: social and cultural capitals (Bourdieu): language, digital competences, transnational networks, and so on. Is there anyone else interested in these topics? [ONE-PAGER]

    Anna Keune Y [ONE-PAGER]

    Samuel Zwaan 

\subsection{Overview of Contributed Ideas - Invisible College}

Given the digital medium is actually going to look nothing like our traditional concept of space ...

If we're imagining a ``virtual college'' that instantiates the new invisible college of distributed networks, we might begin with a familiar image above like this one:

Figure 1. An abstraction of the familiar college campus

Once we understand the virtual versions of these fixtures of the campus, we can start to look for the ``paths in the grass'' -- those emergent patterns that show how people want to use the space. We can then go beyond this picture and ask how it is wrong or misleading, when we consider the nature of technologies like wikis, and the role they can play in coordinating learning and research in a globally-connected environment.

On pp. 4-5 of The New Invisible College, Caroline Wagner introduces ``five forces'' that match a familiar schema. Networks become the laboratory. The circulation of people and ideas gives a new meaning to the forum. Access to the inevitably ``sticky'' features of geography and rare artifacts generalize the library. Distributed teams working in both local communities and with global interconnection provide a space for informal collaboration, socialization, and learning. But the key feature is emergence -- what happens when we put these other features together.

Just starting with the titles or main ideas from the contributed one-page papers for now, please reorder, add detail, and bridges or context as needed -- and move points into the more FORMAL paper outline below. Each of the top-level headings should probably have 3 or 4 main sub-points -- that's a, b, c, d in the format that Google Docs uses :-). Please also refer to the outline of the Peeragogy Handbook 2.0 -- maybe this paper can be a ``warm-up'' for that. We can also draw writing directly from the Handbook 1.1 if we need/want.

        What is the one idea that connects our various contributions and thoughts...? The thread that will connect everything together... the thesis statement, as it were?

    Designs for a distributed university

        New Invisible College: becoming concrete through self-reflection of research on wikis for research (and education): I can provide a short ``book report'' on Wagner's monograph...

        I want to talk about wikis and wiki-like systems as places for capturing the ``externalities'' from other value-creation systems -- a way to document ``the paths in the grass'' that come from unexpected links between different things

        I'm trying to think about ways to do a tiny (10 day) study on PlanetMath along the lines of the themes I just mentioned...

    Creative confrontation for co-leading

        This seems connected to the Forum that in the Notes section that accompanies my write-up -- it also seems that with the various ``contradictory'' ends, the proposal from Fabrizio moves out of the institution of the forum, and into the ``no-man's land'' of OER communities. As such, it gives a nice statement of ``the problem to be solved'' in our research collaboration -- maybe the ``grand challenge''.

\subsection{Eliminating 90/9/1 and More Possibilities}

    Benefits of wikis in learning

        Stats and charts drawn from the first year of the Peeragogy project in the SMC -- but what stats and charts? What do we want to show or learn about the project from a quantitative point of view? What can we do to connect this to qualitative matters drawn from our own experience with the project?

    An experiment in the efficiency of learning

        Power laws...? Note that the Peeragogy project will also exhibit some power-law like features (no doubt). What can we learn from this? When I think about power laws, I often ask ``what's the exponent?''. What makes the difference between rapid fall off and less-rapid fall off? How does this relate to ``learning''?

*** John mentioned something about learning/hour metric\ldots can we incorporate something like that into our assessment??

        What is our ``contribution to knowledge'' in this paper? (High-level summary of the key findings and ideas.)\cite{paragogy}
\cite{Origins}
\cite{Tales}
\cite{Peeragogy-2}
\cite{College}
\cite{Bridges}
\cite{peer}
\cite{Gluing}
\cite{GroupInformatics}
\cite{Why}
\cite{PeeragogyinAction}

%
%
% New Section note for visual aid while editing
%
%

\section{Conclusion}

Going forward we need to record data better -> (discussion with R\'egis about this)\footnote{\url{https://plus.google.com/u/0/101437188321463196206/posts/eUxkDqEAmvG}} -> can we move towards some sort of learning achieved by hour metric?

    The main point, I think, is, this one: What are the methods we've identified that will help us make progress towards our goals?

                Hypothesis 1 : in a peeragogy environment the 90-9-1 principle doesn't make sense

                Hypothesis 2 : the long tail rule applies - This may validate an other point in the section ``managing participation''

                Hypothesis 3 : most contributors influence the structure of the site not only the content (meaning re-organize the links over time) - I my wiki experience I see often a bunch of architectes create the structure and users follow it

                Hypothesis 4 : peer learning projects need to find the right balance of freedom, interest and bureaucracy

        I had an interesting idea about the power law of participation. The 90/10/1 breaks down when you work with small groups (like 1 person) -- and I'm guessing that the ``long tail'' is much longer (and skinnier) in G+ than in SMC. However, it's not necessarily that fat in SMC either.

        Also, again, ``quality'' seems at least as important as ``quantity''... 

    A ``phenomenological approach'': what is it like to participate or to have participated in the project?


            We can start filling in ideas about these points asynchronously, as Charlie has started below. (Thanks Charlie!)

        The points of the PAR are as follow:

            Review what was supposed to happen

            What happened / is happening?

            What is right or wrong with what we're doing / have done?

            What did we learn or change?

            How can we learn this experience to improve next time?

                Charlie: can we move to using a higher percent of Free Software tools? (more likely our work can be shared, remixed \& properly stored?)

        This aspect of the paper would echo the Paragogy paper that Joe and Charlie presented at OKCon 2011, which would provide a good point of comparison -- have we learned anything since ``our P2PU days''?

Could this be useful to programmers at WikiSym as a way for them to crowdsource their documentation?

This paper can be thought of as a moment in time of the Peeragogy Project, sort of like a financial statement (e.g. Balance Sheet) we want to understand where the money/learning's been going over the past year to budget learning for next year

Future steps
Outreach

Individuals have a lot of motivation, which is great, but they may need some support even if they are ``self-starters''.

``If you want to go fast, go alone, if you want to go far go together.''
We tend to need clear outputs, some form of assessment (e.g. a paper,
a talk, a thesis, a ``masterpiece'')
Documenting the more distributed knowledge process
e.g. you can watch a talk online

%
% The following two commands are all you need in the
% initial runs of your .tex file to
% produce the bibliography for the citations in your paper.
% (Uncomment to reproduce the .bbl file if needed!)

%% \bibliographystyle{abbrv}
%% \bibliography{bib}  % sigproc.bib is the name of the Bibliography in this case

\begin{thebibliography}{10}

\bibitem{Origins}
C.~Alexander.
\newblock The origins of pattern theory, the future of the theory, and the
  generation of a living world,, 1996.

\bibitem{paragogy}
J.~Corneli and C.~J. Danoff.
\newblock Paragogy.
\newblock In S.~Hellmann, P.~Frischmuth, S.~Auer, and D.~Dietrich, editors,
  {\em Proceedings of the 6th Open Knowledge Conference}, Berlin, Germany,
  2011.

\bibitem{PeeragogyinAction}
J.~Corneli, A.~Keune, C.~J. Danoff, and A.~Lyons.
\newblock Peeragogy in action.
\newblock In {\em The Open Book}, pages 80--87. The Finnish Institute in
  London, 2013.

\bibitem{peer}
C.~J. Danoff and J.~Corneli.
\newblock The {P}aragogical {A}ction {R}eview.
\newblock {\em submitted to African Journal of Information Systems}, 2013.

\bibitem{Gluing}
A.~Dougherty.
\newblock {G}luing the {W}eb {T}ogether: {A}n {I}nterview with {L}arry {W}all.
\newblock {\em ZD Internet User}, 1998.

\bibitem{Tales}
R.~P. Gabriel.
\newblock {\em Patterns of Software}.
\newblock Oxford University Press, New York, 1996.

\bibitem{GroupInformatics}
S.~Goggins, C.~Mascaro, and G.~Valetto.
\newblock Group informatics: A methodological approach and ontology for
  understanding socio-technical groups.
\newblock {\em Journal of the American Society for Inforamation Science and
  Technology}, 2013.

\bibitem{Why}
D.~S. Hugo~Mercier.
\newblock Why do humans reason? arguments for an argumentative theory.
\newblock {\em Behavioral and Brain Sciences}, pages 34,57--111, 2011.

\bibitem{Bridges}
D.~McGavran.
\newblock {\em The Bridges of God}.
\newblock World Dominion Press, 1955.

\bibitem{Peeragogy-2}
H.~Rheingold et~al.
\newblock {\em The Peeragogy Handbook version 1.1}.
\newblock Chicago: PubDomEd Press, 2013.

\bibitem{College}
C.~S. Wagner.
\newblock {\em The New Invisible College}.
\newblock The Brookings Institution, Washington D.C., paperback edition
  edition, 2005.

\end{thebibliography}

\balancecolumns


\end{document}
