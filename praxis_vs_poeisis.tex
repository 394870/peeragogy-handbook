\begin{quote}
``Praxis, a noble activity, is always one of use, as distinct from
poesis which designates fabrication. Only the former, which plays and
acts, but does not produce, is noble.'' {[}1{]} (p. 101)
\end{quote}

There is a tension between ``making stuff'' and ``using stuff''. Peer
\emph{production}, as the name indicates, is about ``making stuff,''
or~\href{http://en.wikipedia.org/wiki/Poiesis}{poesis}. And stuff is, at
least in theory, kinda cool.~ Furthermore, some of the most familiar
examples of peeragogy in practice come from the craft and maker
movements.~ However, we can also try to be aware of just how much
``learning'' is really ``remix'' -- re-use and recycling of other
people's ideas and techniques.~ Understanding and negotiating the
tension between reuse and creativity is key to the art of remix.

\subsubsection{Reference}

\begin{enumerate}
\item
  Baudrillard, J. (1975). \emph{The mirror of production}. Telos Press
\end{enumerate}
