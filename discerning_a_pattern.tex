\begin{quote}
\emph{{[}W{]}e saw that language use is typically what we have to go on,
from an analytical perspective. Generally, if we are not starting with
language, we arrive at it soon enough. Language becomes something to pay
attention to, in much the same way in which Buddhist practitioners have
for centuries spent time watching their breath.} --
"\href{http://paragogy.net/ParagogicalPraxisPaper}{Paragogical Praxis}"
by Joe Corneli
\end{quote}

The challenge of discerning a pæragogical pattern typically comes down
to the question ``What are we doing with language?'' For example, in
building a peer learning profile someone might identify an interest
(e.g. gardening, puns). We notice this is a pattern when it keeps
happening (most participants have included some interests in their
self-introductions). The classic example of a pattern from architecture
is \textbf{A Place to Wait} -- something that comes up in a lot of
architectural contexts. Once a (suspected) pattern is found, we give it
a title and write down how using the pattern works in a peer learning
context. In the current case, Discerning a Pattern helps us build our
peer learning ``vocabulary'' or ``repertoire'' for peer learners.
