Author: Maria Arenas, with contributions by Charlie Danoff

\subsubsection{\textbf{Summary}}

\textbf{Co-facilitating~emerges when people have to work together in
order to complete a task, in environments like schools, universities,
shelters, churches, workplaces.}

\subsection{Co-facilitating in peer-to-peer learning}

Facilitation~is the process of enabling groups to work cooperatively
and~effectively. Peers co-facilitate by taking and sharing leadership
roles~to move the peer learning process along faster and/or more
efficiently.~The main purpose of \emph{co-facilitation} is to offer and
receive support from a cohort who is invested in the project.
Co-facilitation~commonly can be found in specific collaborations between
two or more~people who need each other to complete a task, for example,
learn about a~given subject, author a technical report, resolve a
problem, or conduct~research Dr. Fink writes in \emph{Creating
Significant Learning Experiences~}(Jossey Bass, 2003) that ``in this
process, there has to be some kind of~change in the learner. No change,
no learning''. Significant learning~requires that there be some kind of
lasting change that is important in terms of the learner's life;
therefore a way to measure the~effectiveness of co-facilitation is if
there's been a change in the peer~group.

\subsection{Which roles, competences and skills do we need to
co-facilitate?}

Co-facilitation~roles can be found in groups/teams like basketball,
health, Alcoholics~Anonymous, spiritual groups, etc. For example,
self-help groups are~composed of people who gather to share common
problems and experiences~associated with a particular problem,
condition, illness, or personal~circumstance. ``Freedom to Learn'' is
among the learning theories~Carl Rogers was known for. Commenting on
Rogers' related work,~Barrett-Lennard remarked: ``\ldots{}he offered
several~hypothesized general principles. These included: We cannot~teach
another person directly; we can only facilitate his learning. The
structure and organization of the self appears to~become more rigid
under threat; to relax its boundaries when~completely free from
threat\ldots{}. The educational~situation which most effectively
promotes significant learning is one in which 1) threat to the self of
the learner is reduced~a minimum, and 2) differentiated perception of
the field of experience is facilitated.'' Part of the facilitator's
role~is creating a safe place for learning to take place; but
they~should also challenge the participants. As John Wooden said of
coaching:~``Be quick, but don't hurry.''
\href{http://en.wikipedia.org/wiki/John\_Heron}{John Heron} articulated
this nature of~facilitation well:

\begin{quote}
`'Too much hierarchical control, and~participants become passive and
dependent or hostile and resistant. They~wane in self-direction, which
is the core of all learning. Too much~cooperative guidance may
degenerate into a subtle kind of nurturing~oppression, and may deny the
group the benefits of totally autonomous~learning. Too much autonomy for
participants and laissez-faire on your~part, and they may wallow in
ignorance, misconception, and chaos.''
\end{quote}

\subsection{Co-facilitating discussion forums}

If~peers are preparing a forum discussion, here are some ideas from
``The~tool box'', that can be helpful as guidelines for running this
type of~meetings:

\begin{itemize}
\item
  Explain the importance of collaborative group work and make it a
  requirement.
\item
  Establish how you will communicate in the forum
\item
  Be aware of mutual blind spots in facilitating and observing others
\item
  Watch out for different rhythms of intervention''.
\end{itemize}

\subsection{Co-facilitating~wiki workflows}

A good place to begin for any co-facilitators working~with a wiki is
Wikipedia's famous ``5 Pillars.''

\begin{itemize}
\item
  Wikipedia is an encyclopedia
\item
  Wikipedia writes articles from a neutral point-of-view
\item
  Wikipedia is free content that anyone can edit, use, modify, and
  distribute.
\item
  Editors should interact with each other in a respectful and civil
  manner.
\item
  Wikipedia does not have firm rules.
\end{itemize}

\subsection{Co-facilitating live sessions}

Learning~experiences in Live Sessions which include Social Media and
co~facilitating exercise is described in the article'' Learning
Re-imagined:~Participatory, Peer, Global, Online`` by Howard Rheingold,
we have taken~inspiration from his points and re-mixed them slightly.

\begin{itemize}
\item
  Establish roles for co facilitators and participants (moderator,
  technical recorder, writer to take notes, etc..).
\item
  Provide a reading list -- indicating what is really important and what
  is more ``nice to know''.
\item
  Ideally before, or when the session begins, take some time to allow
  participants to familiarize themselves with the tools.
\item
  Introduce yourself and your peers (co-facilitators) and ask the
  members to make a brief introduction of themselves.
\item
  Review the agenda for the session, both to make sure there \emph{is}
  an agenda (at the start) and to make sure everything was covered (at
  the end).
\item
  Online~tools like: Mumble, Diigo, Etherpad and chat can be used to
  communicate~and interact in the session. However, consider
  whether~participants are interested in experimenting with lots of
  tools. Often~more tools (and some content) can end up making tasks
  harder.
\item
  Keep~it Simple Stupid, or KISS: Remember you came together with your
  peers~to accomplish something not to discuss an agenda or play with
  online~tools; keep everything as easily accessible as possible to
  ensure you~realize your peer goals.
\end{itemize}

\subsection{Paragogical Action Review}

Following~any co-facilitating session it is essential that the
co-facilitators~come together and review what happened. A useful
framework is
the~\href{http://www.africom.mil/WO-NCO/DownloadCenter/\%5C40Publications/Training\%20the\%20Force\%20Manual.pdf}{Paragogical
Action Review} (PAR), based on the U.S. Army's After Action~Review,
which has four components, to which we have added a fifth. A further
difference in the Paragogical Action Review is that it need not take
place ``after'' the action, but can be integrated into the action
(accordingly, we use a present tense phrasing).

\begin{itemize}
\item
  Review what was supposed to happen (training plans)
\item
  Establish what is happening
\item
  Determine what's right or wrong with what's happening
\item
  Determine how the task should be done differently in the future
\item
  Share your notes with your other peers for feedback and to improve
  things going forward
\end{itemize}

\subsubsection{Experiences and experiments in co-facilitating}

\begin{itemize}
\item
  \href{http://dmlcentral.net/blog/howard-rheingold/learning-reimagined-participatory-peer-global-online}{Learning
  Reimagined: Participatory, Peer, Global, Online}, by Howard Rheingold
\item
  \href{http://www.researchgate.net/}{Research~Gate} is a network
  dedicated to science and research, in which members
  connect,~collaborate and discover scientific publications, jobs
  and~conferences.
\item
  \href{http://ctb.ku.edu/en/tablecontents/section\_1180.aspx}{Creating
  and Facilitating Peer Support Groups}, by The Community Tool Box
\item
  \href{http://www1.villanova.edu/content/villanova/artsci/vcle/resources/toolkit/\_jcr\_content/pagecontent/download\_8/file.res/FacilitationTips.doc}{Facilitation
  Tips}, by Villanova University
\item
  \href{http://pippabuchanan.com/2011/09/04/herding-passionate-cats-the-role-of-facilitator-in-a-peer-learning-process/}{Herding
  Passionate Cats: The Role of Facilitator in a Peer Learning}, by~Pippa
  Buchanan
\item
  \href{http://webpages.sou.edu/~vidmar/SOARS2008/vidmar.ppt}{Reflective
  Peer Facilitation: Crafting Collaborative Self-Assessment}, by Dale
  Vidmar, Southern Oregon University Library
\item
  \href{http://www.umass.edu/ewc/ea/Facilitation\%20Skills/important\%20tips.doc}{Effective
  Co-Facilitation}, by Everywoman´s Center, University of Massachussetts
\end{itemize}

\subsubsection{Resources}

\begin{enumerate}
\item
  \href{http://www.scribd.com/doc/54544925/51/TRAINING-TOPIC-Co-facilitation-skills}{Peer
  Education: Training of Trainers Manual};~UN Interagency Group on Young
  Peoples Health
\item
  \href{http://www.breakoutofthebox.com/Co-FacilitatingPfeifferJones.pdf}{Co
  Facilitating}:~Advantages \& Potential Disadvantages. J. Willam
  Pfeifer and John E Johnes
\item
  A
  \href{http://reviewing.co.uk/archives/art/13\_1\_what\_do\_facilitators\_do.htm\#8\_WAYS\_OF\_FACILITATING\_ACTIVE\_LEARNING}{summary}
  of John Heron's model on role of facilitators
\item
  \href{http://www.infed.org/thinkers/et-rogers.htm}{C}\href{http://www.infed.org/thinkers/et-rogers.htm}{arl
  Rogers, Core Conditions and Education},~Encyclopedia of Informal
  Education
\item
  \href{http://www.studygs.net/peermed.htm}{Peer Mediation}, Study
  Guides and Strategies
\item
  \href{http://sk.cupe.ca/updir/cofacilitation-handouts.doc}{Co-Facilitation:
  The Advantages and Challenges}, Canadian Union of Public Employees
\item
  \href{http://community.bistudio.com/wiki/Bohemia\_Interactive\_Community:Guidelines}{Bohemia
  Interactive Community Wiki Guidelines}
\item
  Barrett-Lennard, G. T. (1998)
  \href{http://openlibrary.org/works/OL2014352W/Carl\_Rogers'\_Helping\_System}{Carl
  Roger's Helping System. Journey and Substance}, London: Sage
\item
  \href{http://en.wikipedia.org/w/index.php?title=Wikipedia:Five\_pillars\&oldid=501472166}{5
  Pillars of Wikipedia}, from Wikipedia
\item
  \emph{\href{http://www.africom.mil/WO-NCO/DownloadCenter/\%5C40Publications/Training\%20the\%20Force\%20Manual.pdf}{Training
  the Force}, (2002)~}US Army Field Manual \#FM 7-0 (FM 25-100)
\end{enumerate}
