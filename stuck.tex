Remember this from our article on
\href{http://peeragogy.org/organizing-a-learning-context/}{organizing a
learning context}?

\begin{quote}
\emph{There is a certain irony here: we are studying ``peeragogy'', and
yet many respondents did not feel they were really getting to know one
another ``as peers''.~ Several remarked that they learned less from
other individual participants, and more from ``the collective''.~ Those
who did have a ``team'', or who knew one another from previous
experiences, felt more peer-like in those relationships.}
\end{quote}

\subsubsection{Are weak ties ``strong''?}

``Weak ties'' are often deemed a strength: see for
example~\href{http://www.psychologytoday.com/blog/thinking-about-kids/201005/facebook-and-the-strength-weak-ties}{this
article}~in Psychology Today, which says:

\begin{quote}
"\emph{{[}S{]}trong and weak ties tend to serve different functions in
our lives. When we need a big favor or social or instrumental support,
we ask our friends.~ We call them when we need to move a washing
machine.~ But if we need information that we don't have, the people to
ask are our weak ties.~ They have more diverse knowledge and more
diverse ties than our close friends do. We ask them when we want to know
who to hire to install our washing machine.}"
\end{quote}

The quote suggests that there is a certain trade-off between use of weak
ties and use of strong ties.~ The anti-pattern in question then is less
to do with whether we are forming weak ties or strong ties, and more to
do with whether we are being \emph{honest with ourselves and with each
other about the nature of the ties we are forming} -- and their
potential uses.~ We can be ``peers'' in either a weak or a strong
sense.~ The question to ask is whether our needs match our expectations!

In the peeragogy context, this has to do with how we interact.~ One of
the participants in this project wrote:

\begin{quote}
"\emph{I am learning about peeragogy, but I think I'm failing {[}to
be{]} a good peeragog{[}ue{]}.~ I remember that Howard {[}once{]} told
us that the most important thing is that you should be responsible not
only for your own learning but for your peers' learning.~ {[}\ldots{}{]}
So the question is, are we learning from others by ourselves or are we
{[}\ldots{}{]} helping others to learn?}"
\end{quote}

If we are ``only'' co-consumers of information (which happens to
``produced'' live, by some of the participants), then this seems like a
classic example of a weak tie.~ We are part of the same ``audience'' --
or anyway, in the same ``theater'' (even if separated from each other by
continuous ``4th walls'').~ On the other hand, actively engaging with
other people (whether with ``my'' learning, with ``their'' learning, or
with the co-production of knowledge) seems to be the foundation for
strong ties.~ In this case our aims (or needs) are more instrumental,
and less informational.

People who do not put in the time and effort will remain stuck at the
level of ``weak ties'', and will not be able to draw on the benefits
that ``strong ties'' offer.
